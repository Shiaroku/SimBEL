\documentclass[a4paper]{book}
\usepackage[times,inconsolata,hyper]{Rd}
\usepackage{makeidx}
\usepackage[utf8,latin1]{inputenc}
% \usepackage{graphicx} % @USE GRAPHICX@
\makeindex{}
\begin{document}
\chapter*{}
\begin{center}
{\textbf{\huge Package `SimBEL'}}
\par\bigskip{\large \today}
\end{center}
\begin{description}
\raggedright{}
\item[Type]\AsIs{Package}
\item[Title]\AsIs{Un package de calcul du best estimate epargne sous Solvabilite
2.}
\item[Version]\AsIs{0.3.1}
\item[Description]\AsIs{Un modele de simulation Monte-Carlo s'appuyant sur une projection
d'un canton (actif et passif) permettant l'evaluation des provisions best
estimate d'un contrat d'epargne francais en euros. Plusieurs chocs de la formule
standard peuvent etre effectues.}
\item[Author]\AsIs{Prim'Act}
\item[URL]\AsIs{}\url{http://primact.fr}\AsIs{}
\item[Maintainer]\AsIs{Quentin Guibert }\email{quentin.guibert@primact.fr}\AsIs{}
\item[Depends]\AsIs{R (>= 3.3.1)}
\item[Imports]\AsIs{rootSolve}
\item[Suggests]\AsIs{rootSolve}
\item[License]\AsIs{GPL-2}
\item[LazyData]\AsIs{TRUE}
\item[RoxygenNote]\AsIs{5.0.1}
\item[Collate]\AsIs{'Action\_class.R' 'Action\_buy.R' 'Action\_calc\_pmvl.R'
'Action\_calc\_vm.R' 'Action\_internal.R' 'Action\_revalo.R'
'Action\_sell.R' 'Action\_sell\_pvl.R' 'Action\_update\_dur\_det.R'
'Action\_update\_vm.R' 'AlmEngine\_class.R'
'AlmEngine\_create\_ptf\_bought\_action.R' 'Immo\_class.R'
'AlmEngine\_create\_ptf\_bought\_immo.R' 'Oblig\_class.R'
'AlmEngine\_create\_ptf\_bought\_oblig.R'
'AlmEngine\_do\_calc\_nb\_sold\_action.R'
'AlmEngine\_do\_calc\_nb\_sold\_immo.R'
'AlmEngine\_do\_calc\_nb\_sold\_oblig.R' 'AlmEngine\_internal.R'
'Treso\_class.R' 'RC\_class.R' 'PRE\_class.R' 'FraisFin\_class.R'
'PortFin\_class.R' 'AlmEngine\_reallocate.R' 'AlmEngine\_update.R'
'AutresPassifs-class.R' 'AutresPassifs-internal.R'
'AutresPassifs-load.R' 'AutresPassifs-proj\_annee.R'
'AutresReserves-class.R' 'AutresReserves-init\_debut\_pgg\_psap.R'
'AutresReserves-internal.R' 'AutresReserves-load.R'
'AutresReserves-update\_reserves.R' 'ESG\_class.R'
'ParamBe\_class.R' 'ParamRevaloEngine\_class.R'
'ParamAlmEngine\_class.R' 'HypCanton\_class.R' 'Ppb\_class.R'
'ModelPointESG\_class.R' 'TabEpEuroInd-class.R'
'EpEuroInd-class.R' 'TauxPB-class.R' 'FraisPassif-class.R'
'ParamRachDyn-class.R' 'ParamComport-class.R'
'ParamTableRach-class.R' 'ParamTableMort-class.R'
'HypTech-class.R' 'PortPassif-class.R' 'Canton\_class.R'
'Be\_class.R' 'Be-run\_be.R' 'Be-run\_be\_simu.R'
'Be-write\_results.R' 'Be\_internal.R' 'Canton\_calc\_fin\_proj.R'
'Canton\_calc\_result\_technique\_ap\_pb.R' 'Canton\_internal.R'
'Canton\_proj\_an.R' 'ParamChocSousc-class.R'
'ParamChocMket\_class.R' 'ChocSolvabilite2\_class.R'
'ChocSolvabilite2\_do\_choc\_action\_type1.R'
'ChocSolvabilite2\_do\_choc\_action\_type2.R'
'ChocSolvabilite2\_do\_choc\_frais.R'
'ChocSolvabilite2\_do\_choc\_immo.R'
'ChocSolvabilite2\_do\_choc\_longevite.R'
'ChocSolvabilite2\_do\_choc\_mortalite.R'
'ChocSolvabilite2\_do\_choc\_rachat\_down.R'
'ChocSolvabilite2\_do\_choc\_rachat\_up.R'
'ChocSolvabilite2\_do\_choc\_spread.R'
'ChocSolvabilite2\_do\_choc\_spread\_unitaire.R'
'ChocSolvabilite2\_do\_choc\_taux.R' 'ChocSolvabilite2\_internal.R'
'ChocSolvabilite2\_load.R' 'ESG-get\_choc\_inflation\_frais.R'
'ESG\_chargement.R' 'ESG\_extract\_mp.R' 'ESG\_internal.R'
'EpEuroInd-calc\_pm.R' 'EpEuroInd-calc\_prest.R'
'EpEuroInd-calc\_primes.R' 'EpEuroInd-calc\_revalo\_pm.R'
'EpEuroInd-calc\_tx\_cible.R' 'EpEuroInd-calc\_tx\_min.R'
'EpEuroInd-calc\_tx\_sortie.R' 'EpEuroInd-internal.R'
'EpEuroInd-vieilli\_mp.R' 'FraisFin\_calc.R'
'FraisFin\_internal.R' 'FraisFin\_load.R'
'FraisPassif-calc\_frais.R' 'FraisPassif-internal.R'
'FraisPassif-load.R' 'HypCanton\_internal.R' 'HypCanton\_load.R'
'HypTech-get\_choc\_rach.R' 'HypTech-get\_choc\_table.R'
'HypTech-get\_comport.R' 'HypTech-get\_qx\_mort.R'
'HypTech-get\_qx\_rach.R' 'HypTech-get\_rach\_dyn.R'
'HypTech-internal.R' 'Initialisation\_class.R'
'Initialisation\_load.R' 'HypTech-load.R' 'Immo\_buy.R'
'Immo\_calc\_pmvl.R' 'Immo\_calc\_vm.R' 'Immo\_internal.R'
'Immo\_revalo.R' 'Immo\_sell.R' 'Immo\_update\_dur\_det.R'
'Immo\_update\_vm.R' 'Initialisation\_create\_folder.R'
'Initialisation\_initSimBEL.R' 'Initialisation\_init\_scenario.R'
'Initialisation\_internal.R' 'Initialisation\_set\_architecture.R'
'ModelPointESG\_internal.R' 'Oblig\_buy.R' 'Oblig\_calc\_coupon.R'
'Oblig\_calc\_dur.R' 'Oblig\_calc\_nominal.R' 'Oblig\_calc\_pmvl.R'
'Oblig\_calc\_sur\_dec.R' 'Oblig\_calc\_vm.R' 'Oblig\_calc\_vnc.R'
'Oblig\_calc\_z\_spread.R' 'Oblig\_echeancier.R'
'Oblig\_flux\_annee.R' 'Oblig\_internal.R' 'Oblig\_sell.R'
'Oblig\_update\_cc.R' 'Oblig\_update\_dur.r'
'Oblig\_update\_mat\_res.R' 'Oblig\_update\_sd.r'
'Oblig\_update\_vm.r' 'Oblig\_update\_vnc.r' 'Oblig\_update\_zsp.r'
'Oblig\_yield\_to\_maturity.R' 'PRE\_calc.R'
'PRE\_do\_update\_val\_courante.R' 'PRE\_do\_update\_val\_debut.R'
'PRE\_internal.R' 'PRE\_load.R' 'ParamAlmEngine\_internal.R'
'ParamAlmEngine\_load.R' 'ParamBe\_internal.R'
'ParamChocMket\_internal.R' 'ParamChocSousc-internal.R'
'ParamComport-calc\_tx\_cible.R' 'ParamComport-internal.R'
'ParamRachDyn-calc\_rach\_dyn.R' 'ParamRachDyn-internal.R'
'ParamRevaloEngine\_internal.R' 'ParamRevaloEngine\_load.R'
'ParamTableMort-calc\_qx.R' 'ParamTableMort-internal.R'
'ParamTableRach-calc\_rach.R' 'ParamTableRach-internal.R'
'PortFin\_calc\_pmvl.R' 'PortFin\_calc\_rdt.R'
'PortFin\_calc\_resultat\_fin.R' 'PortFin\_calc\_tra.R'
'PortFin\_chargement.R' 'PortFin\_chargement\_reference.R'
'PortFin\_do\_update\_pmvl.R'
'PortFin\_do\_update\_vm\_vnc\_precedent.R' 'PortFin\_internal.R'
'PortFin\_print\_alloc.R' 'PortFin\_update.R'
'PortFin\_update\_reference.R' 'PortFin\_vieillissement\_action.R'
'PortFin\_vieillissement\_immo.R'
'PortFin\_vieillissement\_oblig.R'
'PortFin\_vieillissement\_treso.R'
'PortPassif-calc\_rdt\_marche\_ref.R' 'PortPassif-internal.R'
'PortPassif-load.R' 'PortPassif-proj\_annee\_av\_pb.R'
'PortPassif-vieillissement\_ap\_pb.R'
'PortPassif-vieillissement\_av\_pb.R' 'Ppb\_dotation\_reprise.R'
'Ppb\_init\_debut.R' 'Ppb\_internal.R' 'Ppb\_load.R' 'RC\_calc.R'
'RC\_do\_update\_val\_courante.R' 'RC\_do\_update\_val\_debut.R'
'RC\_internal.R' 'RC\_load.R' 'RevaloEngine\_base\_prod\_fin.R'
'RevaloEngine\_calc\_marge\_fin.R'
'RevaloEngine\_calc\_result\_technique.R'
'RevaloEngine\_calc\_revalo.R' 'RevaloEngine\_class.R'
'RevaloEngine\_finance\_cible\_marge.R'
'RevaloEngine\_finance\_cible\_pmvl.R'
'RevaloEngine\_finance\_cible\_ppb.R'
'RevaloEngine\_finance\_contrainte\_legale.R'
'RevaloEngine\_finance\_tmg.R' 'RevaloEngine\_internal.R'
'RevaloEngine\_pb\_contr.R' 'SimBEL.R' 'TabEpEuroInd-internal.R'
'TauxPB-internal.R' 'Treso\_calc\_vm.R' 'Treso\_internal.R'
'Treso\_revalo.R' 'Treso\_revenu.R' 'Treso\_update.R'
'taux\_period-function.R'}
\end{description}
\Rdcontents{\R{} topics documented:}
\inputencoding{utf8}
\HeaderA{Action}{La classe Action}{Action}
%
\begin{Description}\relax
Classe pour les actifs de type Action
\end{Description}
%
\begin{Section}{Slots}

\begin{description}

\item[\code{ptf\_action}] est un dataframe, chaque ligne represente un actif action du portefeuille d'action.

\end{description}
\end{Section}
%
\begin{Author}\relax
Prim'Act
\end{Author}
%
\begin{SeeAlso}\relax
Les operations d'achat vente action  \code{\LinkA{buy\_action}{buy.Rul.action}},
\code{\LinkA{sell\_action}{sell.Rul.action}} et \code{\LinkA{sell\_pvl\_action}{sell.Rul.pvl.Rul.action}}.
\end{SeeAlso}
\inputencoding{utf8}
\HeaderA{AlmEngine}{La classe ALMEngine}{AlmEngine}
%
\begin{Description}\relax
Classe ayant pour principal vocation de contenir des methodes de reallocation.
\end{Description}
%
\begin{Section}{Slots}

\begin{description}

\item[\code{journal\_achat\_vente}] outil permettant de memoriser l'ensemble des operations d'achat-vente.

\end{description}
\end{Section}
%
\begin{Author}\relax
Prim'Act
\end{Author}
%
\begin{SeeAlso}\relax
La fonction de reallocation du Portefeuille \code{\LinkA{reallocate}{reallocate}}
\end{SeeAlso}
\inputencoding{utf8}
\HeaderA{AutresPassifs}{La classe \code{AutresPassifs}}{AutresPassifs}
\keyword{classes}{AutresPassifs}
%
\begin{Description}\relax
Une classe pour la gestion des passifs hors modele.
\end{Description}
%
\begin{Section}{Slots}

\begin{description}

\item[\code{mp}] un objet \code{data.frame} au format fige contenant les flux des passifs hors modele.

\end{description}
\end{Section}
%
\begin{Author}\relax
Prim'Act
\end{Author}
%
\begin{SeeAlso}\relax
La lecture des flux d'une annee \code{\LinkA{proj\_annee\_autres\_passifs}{proj.Rul.annee.Rul.autres.Rul.passifs}}.
\end{SeeAlso}
\inputencoding{utf8}
\HeaderA{AutresReserves}{La classe AutreReserves}{AutresReserves}
\keyword{classes}{AutresReserves}
%
\begin{Description}\relax
Une classe de parametres permettant de gerer le stock de provision globale de gestion (PGG) et de
provision pour sinistres a payer (PSAP).
\end{Description}
%
\begin{Section}{Slots}

\begin{description}

\item[\code{pgg\_debut}] la valeur de la PGG en debut de periode.

\item[\code{psap\_debut}] la valeur de la PSAP en debut de periode.

\item[\code{pgg\_valeur}] la valeur courant de la PGG.

\item[\code{psap\_valeur}] la valeur courant de la PSAP.

\item[\code{tx\_pgg\_ep}] le taux de PGG applique sur l'epargne.

\item[\code{tx\_pgg\_autres}] le taux de PGG applique sur les autres passifs.

\item[\code{tx\_psap\_ep}] le taux de PGG applique sur l'epargne.

\item[\code{tx\_psap\_autres}] le taux de PGG applique sur les autres passifs.

\end{description}
\end{Section}
%
\begin{Author}\relax
Prim'Act
\end{Author}
%
\begin{SeeAlso}\relax
Le calcul et la mise a jour des autres reserves \code{\LinkA{update\_reserves}{update.Rul.reserves}} et
\code{\LinkA{init\_debut\_pgg\_psap}{init.Rul.debut.Rul.pgg.Rul.psap}}.
\end{SeeAlso}
\inputencoding{utf8}
\HeaderA{autres\_passif\_load}{Methode permettant de charger la valeur initiale des autres passifs.}{autres.Rul.passif.Rul.load}
\aliasA{AutresPassifs}{autres\_passif\_load}{AutresPassifs}
%
\begin{Description}\relax
\code{autres\_passif\_load} est une methode permettant de charger les donnees associees a un
objet de classe \code{\LinkA{AutresPassifs}{AutresPassifs}}.
\end{Description}
%
\begin{Usage}
\begin{verbatim}
autres_passif_load(file_autres_passif_address)
\end{verbatim}
\end{Usage}
%
\begin{Arguments}
\begin{ldescription}
\item[\code{file\_autres\_passif\_address}] est un \code{character} contenant l'adresse exacte
du fichier d'input utilisateur
permettant de renseigner un objet \code{\LinkA{AutresPassifs}{AutresPassifs}}.
\end{ldescription}
\end{Arguments}
%
\begin{Value}
L'objet de la classe \code{\LinkA{AutresPassifs}{AutresPassifs}} construit a partir des inputs renseignes par l'utilisateur.
\end{Value}
%
\begin{Author}\relax
Prim'Act
\end{Author}
%
\begin{SeeAlso}\relax
La classe \code{\LinkA{Initialisation}{Initialisation}} et sa methode \code{\LinkA{set\_architecture}{set.Rul.architecture}}
pour renseigner l’input.
\end{SeeAlso}
\inputencoding{utf8}
\HeaderA{autres\_reserves\_load}{Methode permettant de charger la valeur initiale de la PSAP et de la PGG.}{autres.Rul.reserves.Rul.load}
\aliasA{AutresReserves}{autres\_reserves\_load}{AutresReserves}
%
\begin{Description}\relax
\code{autres\_reserves\_load} est une methode permettant de charger les donnees associees a un
objet de classe \code{\LinkA{AutresReserves}{AutresReserves}}.
\end{Description}
%
\begin{Usage}
\begin{verbatim}
autres_reserves_load(file_autres_reserves_address)
\end{verbatim}
\end{Usage}
%
\begin{Arguments}
\begin{ldescription}
\item[\code{file\_autres\_reserves\_address}] est un \code{character} contenant l'adresse exacte
du fichier d'input utilisateur
permettant de renseigner un objet \code{\LinkA{AutresReserves}{AutresReserves}}.
\end{ldescription}
\end{Arguments}
%
\begin{Value}
L'objet de la classe \code{\LinkA{AutresReserves}{AutresReserves}} construit a partir des inputs renseignes par l'utilisateur.
\end{Value}
%
\begin{Author}\relax
Prim'Act
\end{Author}
%
\begin{SeeAlso}\relax
La classe \code{\LinkA{Initialisation}{Initialisation}} et sa methode \code{\LinkA{set\_architecture}{set.Rul.architecture}}
pour renseigner l’input.
\end{SeeAlso}
\inputencoding{utf8}
\HeaderA{base\_prod\_fin}{Calcule la base de produits financiers attribuables.}{base.Rul.prod.Rul.fin}
\aliasA{RevaloEngine}{base\_prod\_fin}{RevaloEngine}
%
\begin{Description}\relax
\code{base\_prod\_fin} est une methode permettant de calculer la base de produits financiers attribuables
pour la revalorisation des contrats.
\end{Description}
%
\begin{Usage}
\begin{verbatim}
base_prod_fin(tra, pm_moy, ppb)
\end{verbatim}
\end{Usage}
%
\begin{Arguments}
\begin{ldescription}
\item[\code{tra}] est une valeur \code{numeric} donnant le taux de rendement de l'actif.

\item[\code{pm\_moy}] est un vecteur \code{numeric} comprenant le montant de PM moyenne par produit.

\item[\code{ppb}] est un objet de la classe \code{\LinkA{Ppb}{Ppb}} qui renvoie l'etat courant de la PPB.
\end{ldescription}
\end{Arguments}
%
\begin{Value}
La valeur de la base de produit financier par produit et au total pour le portefeuille.
\end{Value}
%
\begin{Author}\relax
Prim'Act
\end{Author}
%
\begin{SeeAlso}\relax
\code{\LinkA{Ppb}{Ppb}}.
\end{SeeAlso}
\inputencoding{utf8}
\HeaderA{Be}{La classe \code{Be}.}{Be}
\keyword{classes}{Be}
%
\begin{Description}\relax
Une classe pour le calcul du best estimate d'un assureur.
\end{Description}
%
\begin{Section}{Slots}

\begin{description}

\item[\code{param\_be}] un objet \code{\LinkA{ParamBe}{ParamBe}} qui regroupe les parametres de base du calcul d'un best estimate.

\item[\code{canton}] un objet de type \code{\LinkA{Canton}{Canton}} correspond au canton parametre en date initiale.

\item[\code{esg}] un objet de type \code{\LinkA{ESG}{ESG}}.

\item[\code{tab\_flux}] une liste qui contient les flux moyens de best estimate et de ses composantes.

\item[\code{tab\_be}] est une liste qui contient la valeur du best estimate et de ses composantes.

\end{description}
\end{Section}
%
\begin{Author}\relax
Prim'Act
\end{Author}
%
\begin{SeeAlso}\relax
Le calcul d'un best estimate : \code{\LinkA{run\_be}{run.Rul.be}}.
Le calcul d'une simulation de best estimate : \code{\LinkA{run\_be\_simu}{run.Rul.be.Rul.simu}}.
L'initialisation d'un best estimate dans les situations centrales et choquees : \code{\LinkA{init\_scenario}{init.Rul.scenario}}.
La sortie des resultats au format ".csv" : \code{\LinkA{write\_be\_results}{write.Rul.be.Rul.results}}.
La classe \code{\LinkA{Canton}{Canton}}.
La classe \code{\LinkA{ESG}{ESG}}.
La classe \code{\LinkA{ParamBe}{ParamBe}}.
\end{SeeAlso}
\inputencoding{utf8}
\HeaderA{buy\_action}{Mise a jour de chaque composante d'un portefeuille action suite a un achat d'un autre portefeuille action.}{buy.Rul.action}
\aliasA{Action}{buy\_action}{Action}
%
\begin{Description}\relax
\code{buy\_action} est une methode permettant de mettre a jour le portefeuille action suite a l'achat d'un autre portefeuille action.
de chaque composante d'un portefeuille action.
\end{Description}
%
\begin{Usage}
\begin{verbatim}
buy_action(x, ptf_bought)
\end{verbatim}
\end{Usage}
%
\begin{Arguments}
\begin{ldescription}
\item[\code{x}] objet de la classe \code{Action} (decrivant le portefeuille action en detention).

\item[\code{ptf\_bought}] objet de la classe \code{Action} (decrivant le portefeuille action achete).
\end{ldescription}
\end{Arguments}
%
\begin{Value}
L'objet \code{x} complete des elements de \code{ptf\_bought}.
\end{Value}
%
\begin{Author}\relax
Prim'Act
\end{Author}
\inputencoding{utf8}
\HeaderA{buy\_immo}{Mise a jour de chaque composante d'un portefeuille immo suite a un achat d'un autre portefeuille immo.}{buy.Rul.immo}
\aliasA{Immo}{buy\_immo}{Immo}
%
\begin{Description}\relax
\code{buy\_immo} est une methode permettant de mettre a jour le portefeuille immo suite a l'achat d'un autre portefeuille immo.
de chaque composante d'un portefeuille immo.
\end{Description}
%
\begin{Usage}
\begin{verbatim}
buy_immo(x, ptf_bought)
\end{verbatim}
\end{Usage}
%
\begin{Arguments}
\begin{ldescription}
\item[\code{x}] objet de la classe \code{Immo} (decrivant le portefeuille immo en detention).

\item[\code{ptf\_bought}] objet de la classe \code{Immo} (decrivant le portefeuille immo achete).
\end{ldescription}
\end{Arguments}
%
\begin{Value}
L'objet \code{x} complete des elements de \code{ptf\_bought}.
\end{Value}
%
\begin{Author}\relax
Prim'Act
\end{Author}
\inputencoding{utf8}
\HeaderA{buy\_oblig}{Mise a jour de chaque composante d'un portefeuille obligataire suite a un achat d'un autre portefeuille obligataire.}{buy.Rul.oblig}
\aliasA{Oblig}{buy\_oblig}{Oblig}
%
\begin{Description}\relax
\code{buy\_oblig} est une methode permettant de mettre a jour le portefeuille obligataire suite a l'achat d'un autre portefeuille obligataire.
de chaque composante d'un portefeuille obligataire.
\end{Description}
%
\begin{Usage}
\begin{verbatim}
buy_oblig(x, ptf_bought)
\end{verbatim}
\end{Usage}
%
\begin{Arguments}
\begin{ldescription}
\item[\code{x}] objet de la classe \code{Oblig} (decrivant le portefeuille obligataire en detention).

\item[\code{ptf\_bought}] objet de la classe \code{Oblig} (decrivant le portefeuille obligataire achete).
\end{ldescription}
\end{Arguments}
%
\begin{Value}
L'objet \code{x} complete des elements de \code{ptf\_bought}.
\end{Value}
%
\begin{Author}\relax
Prim'Act
\end{Author}
\inputencoding{utf8}
\HeaderA{calc\_coupon}{Calcul le coupon des models points constituant le portefeuille obligataire.}{calc.Rul.coupon}
\aliasA{Oblig}{calc\_coupon}{Oblig}
%
\begin{Description}\relax
\code{calc\_coupon} est une methode permettant de calculer les valeurs de coupon de l'ensemble des obligations
composant un portefeuille obligataire.
\end{Description}
%
\begin{Usage}
\begin{verbatim}
calc_coupon(x)
\end{verbatim}
\end{Usage}
%
\begin{Arguments}
\begin{ldescription}
\item[\code{x}] un objet de la classe Oblig, dont on souhaite calculer le coupon annuel de chacune de ses composantes.
\end{ldescription}
\end{Arguments}
%
\begin{Value}
Un vecteur dont chaque element correspond a la valeur du coupon de l'obligation consideree : tx\_coupon * parite * nominal * nb\_unit.
Le vecteur renvoye a autant d'elements que le portefeuille obligataire en input a de lignes.
\end{Value}
%
\begin{Author}\relax
Prim'Act
\end{Author}
\inputencoding{utf8}
\HeaderA{calc\_dotation\_ppb}{Dote la valeur de la PPB}{calc.Rul.dotation.Rul.ppb}
\aliasA{Ppb}{calc\_dotation\_ppb}{Ppb}
%
\begin{Description}\relax
\code{calc\_dotation\_ppb} est une methode permettant de doter la PPB. La dotation est effectuee si les
limites de dotation de la PPB sur l'annee ne sont pas atteintes. La valeur de cette limite est mise a jour suite a la dotation.
\end{Description}
%
\begin{Usage}
\begin{verbatim}
calc_dotation_ppb(x, montant)
\end{verbatim}
\end{Usage}
%
\begin{Arguments}
\begin{ldescription}
\item[\code{x}] objet de la classe \code{\LinkA{Ppb}{Ppb}}.

\item[\code{montant}] une valeur \code{numeric} a doter.
\end{ldescription}
\end{Arguments}
%
\begin{Value}
\code{ppb} l'objet \code{x} mis a jour.

\code{dotation} le montnant de la dotation effectuee.
\end{Value}
%
\begin{Author}\relax
Prim'Act
\end{Author}
\inputencoding{utf8}
\HeaderA{calc\_fin\_proj}{calcule le flux et les resultats ajustes en fin de projection.}{calc.Rul.fin.Rul.proj}
\aliasA{Canton}{calc\_fin\_proj}{Canton}
%
\begin{Description}\relax
\code{calc\_fin\_proj} est une methode permettant de calculer au niveau du canton les resultats financier, technique,
brut et net d'impot, ainsi que le flux de passifs soldant une projection.
\end{Description}
%
\begin{Usage}
\begin{verbatim}
calc_fin_proj(x, resultat_fin, result_tech, pm_fin_ap_pb, tx_pb, tx_enc_moy)
\end{verbatim}
\end{Usage}
%
\begin{Arguments}
\begin{ldescription}
\item[\code{x}] est un objet de la classe \code{\LinkA{Canton}{Canton}}.

\item[\code{resultat\_fin}] est la valeur \code{numeric} du resultat financier avant fin de projection.

\item[\code{result\_tech}] est la valeur \code{numeric} du resultat technique avant fin de projection.

\item[\code{pm\_fin\_ap\_pb}] est un vecteur \code{numeric} par produit
correspond au PM de fin avant application de la fin de projection.

\item[\code{tx\_pb}] est un vecteur \code{numeric} par produit
correspond au taux de PB contractuel.

\item[\code{tx\_enc\_moy}] est un vecteur \code{numeric} par produit
correspond au taux chargement sur encours moyens.
\end{ldescription}
\end{Arguments}
%
\begin{Value}
\code{flux\_fin\_passif} un vecteur de flux de fin par produit.

\code{result\_tech} le montant de resultat technique en fin de projection.

\code{result\_fin} le montant de resultat finanacier en fin de projection.

\code{result\_brut} le montant de resultat brut d'impot en fin de projection.

\code{result\_net} le montant de resultat net d'impot en fin de projection.

\code{impot} le montant d'impot sur le resultat en fin de projection.
\end{Value}
\inputencoding{utf8}
\HeaderA{calc\_flux\_annee}{Calcul les flux percus dans l'annee du fait de la detention des obligations du portefeuille obligataire.}{calc.Rul.flux.Rul.annee}
\aliasA{Oblig}{calc\_flux\_annee}{Oblig}
%
\begin{Description}\relax
\code{calc\_flux\_annee} est une methode permettant de calculer les valeurs nominales de l'ensemble des obligations
composant un portefeuille obligataire.
\end{Description}
%
\begin{Usage}
\begin{verbatim}
calc_flux_annee(x)
\end{verbatim}
\end{Usage}
%
\begin{Arguments}
\begin{ldescription}
\item[\code{x}] un objet de la classe Oblig.
\end{ldescription}
\end{Arguments}
%
\begin{Value}
Une liste composee de deux vecteurs:
\begin{description}

\item[\code{tombee\_coupon} : ] Chaque element correspond aux tombees de coupon pour l'annee a venir. Ce vecteur a autant d'elements
que le portefeuille obligataire d'inputs a de lignes.
\item[\code{tombee\_echeance} : ] Chaque element correspond aux tombees d echeances pour l'annee a venir. Ce vecteur a autant d'elements
que le portefeuille obligataire d'inputs a de lignes.

\end{description}

\end{Value}
%
\begin{Author}\relax
Prim'Act
\end{Author}
\inputencoding{utf8}
\HeaderA{calc\_frais}{Calcule des frais de passif.}{calc.Rul.frais}
\aliasA{FraisPassif}{calc\_frais}{FraisPassif}
%
\begin{Description}\relax
\code{calc\_frais} est une methode generique permettant de calculer les frais sur prestations, sur primes
et sur encours.
\end{Description}
%
\begin{Usage}
\begin{verbatim}
calc_frais(x, type, nom_prod, nb, mt, coef_inf)
\end{verbatim}
\end{Usage}
%
\begin{Arguments}
\begin{ldescription}
\item[\code{x}] objet de la classe \code{\LinkA{FraisPassif}{FraisPassif}}.

\item[\code{type}] un \code{character} designant le type de frais applique.

\item[\code{nom\_prod}] est le nom de produit de type \code{character}.

\item[\code{nb}] correspond a un nombre de contrats, utilise comme assiette de frais fixe par contrat.

\item[\code{mt}] correspond a un montant, utilise comme assiette de frais variable.

\item[\code{coef\_inf}] correspond au coefficient d'inflation applique.
\end{ldescription}
\end{Arguments}
%
\begin{Details}\relax
Le type du contrat prend pour valeur \code{prime} pour les frais sur primes, \code{prest} pour les frais
sur prestations et \code{enc} pour les frais sur encours.
\end{Details}
%
\begin{Value}
Une liste contenant les montants de frais fixes et de frais variables.
\end{Value}
%
\begin{Author}\relax
Prim'Act
\end{Author}
\inputencoding{utf8}
\HeaderA{calc\_frais\_fin}{Calcul des frais financier.}{calc.Rul.frais.Rul.fin}
\aliasA{FraisFin}{calc\_frais\_fin}{FraisFin}
%
\begin{Description}\relax
\code{calc\_frais\_fin} est une methode permettant de calculer les frais financiers.
\end{Description}
%
\begin{Usage}
\begin{verbatim}
calc_frais_fin(x, vm_moy, coef_inflation)
\end{verbatim}
\end{Usage}
%
\begin{Arguments}
\begin{ldescription}
\item[\code{x}] est un objet de type \code{FraisFin} contenant les parametres des frais financiers associes a un canton.

\item[\code{vm\_moy}] est un objet de type \code{numeric} correspondant a la valeur moyenne de l'actif en valeur
de marche.

\item[\code{coef\_inflation}] est un objet de type \code{numeric} correspondant au coefficient d'inflation des frais.
\end{ldescription}
\end{Arguments}
%
\begin{Value}
La valeur des frais financiers : un reel de type \code{numeric}.
\end{Value}
%
\begin{Author}\relax
Prim'Act
\end{Author}
\inputencoding{utf8}
\HeaderA{calc\_marge\_fin}{Calcule la marge financiere de l'assureur.}{calc.Rul.marge.Rul.fin}
\aliasA{RevaloEngine}{calc\_marge\_fin}{RevaloEngine}
%
\begin{Description}\relax
\code{calc\_marge\_fin} est une methode permettant de
de calculer la marge financiere de l'assureur apres attribution d'un certain niveau de revalorisation.
\end{Description}
%
\begin{Usage}
\begin{verbatim}
calc_marge_fin(base_fin, rev_prest_nette, rev_stock_nette, contrib_tmg_prest,
  contrib_tmg_stock, contrib_ppb_tx_cible)
\end{verbatim}
\end{Usage}
%
\begin{Arguments}
\begin{ldescription}
\item[\code{base\_fin}] est un vecteur de type \code{numeric} comprenant par produit la base de produits financiers.

\item[\code{rev\_prest\_nette}] est un vecteur de type \code{numeric} comprenant par produit
la revalorisation nette sur prestations.

\item[\code{rev\_stock\_nette}] est un vecteur de type \code{numeric} comprenant par produit
la revalorisation nette sur stock.

\item[\code{contrib\_tmg\_prest}] est une valeur \code{numeric} comprenant par produit
la contribution de la PPB au financement des TMG sur prestations.

\item[\code{contrib\_tmg\_stock}] est une valeur \code{numeric} comprenant par produit
la contribution de la PPB au financement des TMG sur stock.

\item[\code{contrib\_ppb\_tx\_cible}] une valeur de type \code{numeric} comprenant par produit
la contribution de la PPB au financement au taux cible sur stock.
\end{ldescription}
\end{Arguments}
%
\begin{Value}
Le montant de la marge de l'assureur.
\end{Value}
%
\begin{Author}\relax
Prim'Act
\end{Author}
\inputencoding{utf8}
\HeaderA{calc\_nominal}{Calcul le nominal des models points constituant le portefeuille obligataire.}{calc.Rul.nominal}
\aliasA{Oblig}{calc\_nominal}{Oblig}
%
\begin{Description}\relax
\code{calc\_nominal} est une methode permettant de calculer les valeurs de nominal de l'ensemble des obligations
composant un portefeuille obligataire.
\end{Description}
%
\begin{Usage}
\begin{verbatim}
calc_nominal(x)
\end{verbatim}
\end{Usage}
%
\begin{Arguments}
\begin{ldescription}
\item[\code{x}] un objet de la classe Oblig.
\end{ldescription}
\end{Arguments}
%
\begin{Value}
Un vecteur dont chaque element correspond a la valeur du nominal de l'obligation consideree : parite * nominal * nb\_unit.
Le vecteur renvoye a autant d'elements que le portefeuille obligataire en input a de lignes.
\end{Value}
%
\begin{Author}\relax
Prim'Act
\end{Author}
\inputencoding{utf8}
\HeaderA{calc\_pm}{Calcul les PM pour des contrats epargne en euros.}{calc.Rul.pm}
\aliasA{EpEuroInd}{calc\_pm}{EpEuroInd}
%
\begin{Description}\relax
\code{calc\_pm} est une methode permettant de calculer les provisions mathematiques (PM)
de fin de periode avant application de la revalorisation au titre de la participation aux benefices.
\end{Description}
%
\begin{Usage}
\begin{verbatim}
calc_pm(x, tab_prime, tab_prest, tx_cible, tx_min, an, method, tx_soc)
\end{verbatim}
\end{Usage}
%
\begin{Arguments}
\begin{ldescription}
\item[\code{x}] un objet de la classe \code{\LinkA{EpEuroInd}{EpEuroInd}} contenant les model points epargne euros.

\item[\code{tab\_prime}] une liste contenant les flux de primes pour chaque ligne de model points.
Le format de cette liste correspond a la sortie \code{flux} de la methode \code{\LinkA{calc\_primes}{calc.Rul.primes}}.

\item[\code{tab\_prest}] est une liste contenant les flux de prestations pour chaque ligne de model points.
Le format de cette liste correspond a la sortie \code{flux} de la methode \code{\LinkA{calc\_prest}{calc.Rul.prest}}.

\item[\code{tx\_cible}] est une liste conteant les taux cible annuel et semestriel par model points.
Le format de cette liste correspond a la sortie de la methode \code{\LinkA{calc\_tx\_cible}{calc.Rul.tx.Rul.cible}}.

\item[\code{tx\_min}] une liste contenant le taux de revalorisation minimum associes a chaque ligne de model points.
Le format de cette liste correspond a la sortie de la methode \code{\LinkA{calc\_tx\_min}{calc.Rul.tx.Rul.min}}.

\item[\code{an}] une valeur \code{numeric} represantant l'annee de projection courante.

\item[\code{method}] un \code{character} prenant pour valeur \code{normal} pour le calcul
des flux avec application de la revalorisation au titre de la participation aux benefices,
et la valeur \code{gar} pour le calcul avec uniquement les flux garanti (calcul de la FDB).

\item[\code{tx\_soc}] est une valeur \code{numeric} correspondant au taux de prelevements sociaux.
\end{ldescription}
\end{Arguments}
%
\begin{Details}\relax
Cette methode permet de calculer les montants de PM de fin d'annee avec une revalorisation
minimale. Les chargements sur encours sont egalement preleves. Cette methode permet de gerer les contrats a taux de
revalorisation net negatif. Cette methode permet egalement de calculer le besoin de financement necessaire
pour atteindre les exigences de revalorisation des assures.
\end{Details}
%
\begin{Value}
Une liste contenant :
\begin{description}

\item[\code{method} : ] la valeur de l'argument \code{method}
\item[\code{flux} : ] une liste comprenant les flux de l'annee
\item[\code{stock} : ] une liste comprenant les nombres de sorties

\end{description}


Le format de la liste \code{flux} est :
\begin{description}

\item[\code{rev\_stock\_brut} : ] un vecteur contenant la revalorisation minimale
brute de l'annee appliquee au PM
\item[\code{rev\_stock\_nette} : ] un vecteur contenant la revalorisation minimale
nette de l'annee appliquee au PM
\item[\code{enc\_charg\_stock} : ] un vecteur contenant les chargement sur encours de l'annee,
calcules en prenant en compte la revalorisation minimale
\item[\code{enc\_charg\_base\_th} : ] un vecteur contenant les chargements sur encours theoriques
de l'annee, evalues sur la base de la PM non revalorisees
\item[\code{enc\_charg\_rmin\_th} : ] un vecteur contenant les chargements sur encours theoriques
de l'annee, evalues sur la seule base de la revalorisation minimale des PM
\item[\code{base\_enc\_th} : ] un vecteur contenant l'assiette de calcul des chargements sur encours de l'annee
\item[\code{soc\_stock} : ] un vecteur contenant le prelevements sociaux de l'annee
\item[\code{it\_tech\_stock} : ] un vecteur contenant les interets techniques sur stock de l'annee
\item[\code{it\_tech} : ] un vecteur contenant les interets techniques sur stock et
sur prestations de l'annee
\item[\code{bes\_tx\_cible} : ] un vecteur contenant le besoin de financement de l'annee pour
atteindre le taux cible de chaque assure.

\end{description}


Le format de la liste \code{stock} est :
\begin{description}

\item[\code{pm\_deb : }] un vecteur contenant le montant de PM en debut d'annee
\item[\code{pm\_fin : }] un vecteur contenant le montant de PM en fin d'annee, avec
revalorisation au taux minimum
\item[\code{pm\_moy : }] un vecteur contenant le montant de PM moyenne sur l'annee.

\end{description}

\end{Value}
%
\begin{Author}\relax
Prim'Act
\end{Author}
%
\begin{SeeAlso}\relax
\code{\LinkA{calc\_primes}{calc.Rul.primes}}, \code{\LinkA{calc\_prest}{calc.Rul.prest}}, \code{\LinkA{calc\_tx\_cible}{calc.Rul.tx.Rul.cible}},
\code{\LinkA{calc\_tx\_min}{calc.Rul.tx.Rul.min}}.
\end{SeeAlso}
\inputencoding{utf8}
\HeaderA{calc\_pmvl}{Mets a jour les sous totaux de d'actions et immobilier en plus ou moins value latente.}{calc.Rul.pmvl}
\aliasA{PortFin}{calc\_pmvl}{PortFin}
%
\begin{Description}\relax
\code{calc\_pmvl} est une methode permettant de calculer les valeurs de marche.
\end{Description}
%
\begin{Usage}
\begin{verbatim}
calc_pmvl(x)
\end{verbatim}
\end{Usage}
%
\begin{Arguments}
\begin{ldescription}
\item[\code{x}] objet de la classe PortFin.
\end{ldescription}
\end{Arguments}
%
\begin{Value}
L'objet PortFin dont la somme des composantes en PVL et en MVL a ete mise a jour
\end{Value}
%
\begin{Author}\relax
Prim'Act
\end{Author}
\inputencoding{utf8}
\HeaderA{calc\_pmvl\_action}{Calcul les valeurs de marches de chaque composante du portefeuille action.}{calc.Rul.pmvl.Rul.action}
\aliasA{Action}{calc\_pmvl\_action}{Action}
%
\begin{Description}\relax
\code{calc\_pmvl\_action} est une methode permettant de calculer les valeurs de marche.
\end{Description}
%
\begin{Usage}
\begin{verbatim}
calc_pmvl_action(x)
\end{verbatim}
\end{Usage}
%
\begin{Arguments}
\begin{ldescription}
\item[\code{x}] objet de la classe \code{Action} (decrivant le portefeuille d'action).
\end{ldescription}
\end{Arguments}
%
\begin{Value}
Une liste composee de deux elements \code{(pvl, mvl)} correspondant respectivement 
aux sommes des plus values latentes actions et aux sommes des moins values latentes action.
\end{Value}
%
\begin{Author}\relax
Prim'Act
\end{Author}
\inputencoding{utf8}
\HeaderA{calc\_pmvl\_immo}{Calcul les valeurs de marches de chaque composante du portefeuille immobilier.}{calc.Rul.pmvl.Rul.immo}
\aliasA{Immo}{calc\_pmvl\_immo}{Immo}
%
\begin{Description}\relax
\code{calc\_pmvl\_immo} est une methode permettant de calculer les valeurs de marche.
\end{Description}
%
\begin{Usage}
\begin{verbatim}
calc_pmvl_immo(x)
\end{verbatim}
\end{Usage}
%
\begin{Arguments}
\begin{ldescription}
\item[\code{x}] objet de la classe \code{Immo} (decrivant le portefeuille d'immobilier).
\end{ldescription}
\end{Arguments}
%
\begin{Value}
Une liste composee de deux elements \code{(pvl, mvl)} correspondant respectivement 
aux sommes des plus values latentes immobilieres et aux sommes des moins values latentes immobilieres.
\end{Value}
%
\begin{Author}\relax
Prim'Act
\end{Author}
\inputencoding{utf8}
\HeaderA{calc\_pmvl\_oblig}{Calcul les valeurs de marches de chaque composante du portefeuille d'obligations.}{calc.Rul.pmvl.Rul.oblig}
\aliasA{Oblig}{calc\_pmvl\_oblig}{Oblig}
%
\begin{Description}\relax
\code{calc\_pmvl\_oblig} est une methode permettant de calculer les valeurs de marche.
\end{Description}
%
\begin{Usage}
\begin{verbatim}
calc_pmvl_oblig(x)
\end{verbatim}
\end{Usage}
%
\begin{Arguments}
\begin{ldescription}
\item[\code{x}] objet de la classe \code{Oblig} (decrivant le portefeuille d'obligations).
\end{ldescription}
\end{Arguments}
%
\begin{Value}
Une liste composee de deux elements \code{(pvl, mvl)} correspondant respectivement 
aux sommes des plus values latentes obligations et aux sommes des moins values latentes obligations.
\end{Value}
%
\begin{Author}\relax
Prim'Act
\end{Author}
\inputencoding{utf8}
\HeaderA{calc\_PRE}{Calcul de la PRE.}{calc.Rul.PRE}
\aliasA{PRE}{calc\_PRE}{PRE}
%
\begin{Description}\relax
\code{calc\_PRE} est une methode permettant de calculer le montant de PRE.
\end{Description}
%
\begin{Usage}
\begin{verbatim}
calc_PRE(x, pmvl_action_immo)
\end{verbatim}
\end{Usage}
%
\begin{Arguments}
\begin{ldescription}
\item[\code{x}] objet de la classe \code{PRE}, necessaire pour connaitre le stock de PRE initial.

\item[\code{pmvl\_action\_immo}] est un \code{numeric} correspondant au montant global de plus ou moins values latentes des actifs actions et immobiliers.
En cas de moins value latente, la PRE est abondee.
En cas de plus value latente, la PRE est integralement reprise.
\end{ldescription}
\end{Arguments}
%
\begin{Value}
Le format de la liste renvoyee est :
\begin{description}

\item[\code{pre\_courante} : ]  valeur de la pre courante calculee a partir des inputs transmis
\item[\code{var\_pre} : ]  variation de la pre courante

\end{description}

\end{Value}
%
\begin{Author}\relax
Prim'Act
\end{Author}
\inputencoding{utf8}
\HeaderA{calc\_prest}{Calcul les flux de prestations pour des contrats epargne en euros.}{calc.Rul.prest}
\aliasA{EpEuroInd}{calc\_prest}{EpEuroInd}
%
\begin{Description}\relax
\code{calc\_prest} est une methode permettant de calculer les flux de prestations,
les chargements sur encours relatifs a ces prestations et les nombres de sorties sur une periode.
\end{Description}
%
\begin{Usage}
\begin{verbatim}
calc_prest(x, tx_sortie, tx_min, an, method, tx_soc)
\end{verbatim}
\end{Usage}
%
\begin{Arguments}
\begin{ldescription}
\item[\code{x}] un objet de la classe \code{\LinkA{EpEuroInd}{EpEuroInd}} contenant les model points epargne euros.

\item[\code{tx\_sortie}] une matrice contenant les taux de sortie associes a chaque ligne de model points.
Le format de cette matrice correspond a la sortie de la methode \code{\LinkA{calc\_tx\_sortie}{calc.Rul.tx.Rul.sortie}}.

\item[\code{tx\_min}] une liste contenant le taux de revalorisation minimum associes a chaque ligne de model points.
Le format de cette liste correspond a la sortie de la methode \code{\LinkA{calc\_tx\_min}{calc.Rul.tx.Rul.min}}.

\item[\code{an}] une valeur \code{numeric} represantant l'annee de projection courante.

\item[\code{method}] un \code{character} prenant pour valeur \code{normal} pour le calcul
des flux avec application de la revalorisation au titre de la participation aux benefices,
et la valeur \code{gar} pour le calcul avec uniquement les flux garanti (calcul de la FDB).

\item[\code{tx\_soc}] est une valeur \code{numeric} correspondant au taux de prelevements sociaux.
\end{ldescription}
\end{Arguments}
%
\begin{Details}\relax
Cette methode permet de calculer les flux de sortie en echeance, les flux de rachat totaux et partiels et
les flux de deces d'un contrat epargne en euros. Ces prestations font l'objet d'une relavorisation
au taux minimum contractuel. Les nombres de sortie sont egalement produits.
Des chargements sont appliques sur flux de rachats. Des prelevements sur encours sont appliques sur les
prestations revalorises au taux minimum contractuel. Cette methode permet de gerer les contrats a taux de
revalorisation net negatif.
\end{Details}
%
\begin{Value}
Une liste contenant :
\begin{description}

\item[\code{method} : ] la valeur de l'argument \code{method}
\item[\code{flux} : ] une liste comprenant les flux de l'annee
\item[\code{stock} : ] une liste comprenant les nombres de sorties

\end{description}


Le format de la liste \code{flux} est :
\begin{description}

\item[\code{ech} : ] un vecteur contenant les flux de sortie en echeance de l'annee
\item[\code{rach\_tot} : ] un vecteur contenant les flux de rachat totaux de l'annee
\item[\code{dc} : ] un vecteur contenant les flux de deces de l'annee
\item[\code{rach\_part} : ] un vecteur contenant les flux de rachat partiel de l'annee
\item[\code{prest} : ] un vecteur contenant les flux prestations de l'annee
\item[\code{rev\_ech} : ] un vecteur contenant la revalorisation des echeances de l'annee
\item[\code{rev\_rach\_tot} : ] un vecteur contenant la revalorisation des rachats totaux de l'annee
\item[\code{rev\_dc} : ] un vecteur contenant la revalorisation des deces de l'annee
\item[\code{rev\_rach\_part} : ] un vecteur contenant la revalorisation des rachats partiels de l'annee
\item[\code{rev\_prest} : ] un vecteur contenant la revalorisation brute des prestations de l'annee
\item[\code{rev\_prest\_nette} : ] un vecteur contenant la revalorisation des prestations nette de l'annee
\item[\code{enc\_charg} : ] un vecteur contenant les chargements sur l'encours de l'annee
\item[\code{rach\_charg} : ] un vecteur contenant les chargements sur les rachats de l'annee
\item[\code{soc\_prest} : ] un vecteur contenant les prelevements sociaux sur prestations de l'annee
\item[\code{it\_tech\_prest} : ] un vecteur contenant les interets techniques sur prestations de l'annee.

\end{description}


Le format de la liste \code{stock} est :
\begin{description}

\item[\code{nb\_ech : }] un vecteur contenant le nombre de sorties en echeance de l'annee
\item[\code{nb\_rach\_tot : }] un vecteur contenant le nombre de rachats totaux de l'annee
\item[\code{nb\_dc : }] un vecteur contenant le nombre de deces de l'annee
\item[\code{nb\_sortie : }] un vecteur contenant le nombre de sorties de l'annee
\item[\code{nb\_contr\_fin : }] un vecteur contenant le nombre de contrats en cours en fin d'annee
\item[\code{nb\_contr\_moy : }] un vecteur contenant la moyenne du nombre de contrats sur l'annee.

\end{description}

\end{Value}
%
\begin{Author}\relax
Prim'Act
\end{Author}
%
\begin{SeeAlso}\relax
\code{\LinkA{calc\_tx\_sortie}{calc.Rul.tx.Rul.sortie}}, \code{\LinkA{calc\_tx\_min}{calc.Rul.tx.Rul.min}}.
\end{SeeAlso}
\inputencoding{utf8}
\HeaderA{calc\_primes}{Calcul les flux de primes pour des contrats epargne en euros.}{calc.Rul.primes}
\aliasA{EpEuroInd}{calc\_primes}{EpEuroInd}
%
\begin{Description}\relax
\code{calc\_primes} est une methode permettant de calculer les flux de primes sur
une periode.
\end{Description}
%
\begin{Usage}
\begin{verbatim}
calc_primes(x)
\end{verbatim}
\end{Usage}
%
\begin{Arguments}
\begin{ldescription}
\item[\code{x}] un objet de la classe \code{\LinkA{EpEuroInd}{EpEuroInd}} contenant les model points epargne euros.
\end{ldescription}
\end{Arguments}
%
\begin{Details}\relax
Cette fonction permet de projeter uniquement des primes periodiques de contrats epargne en euros.
\end{Details}
%
\begin{Value}
\code{stock} : une liste contenent le nombre de versements \code{nb\_vers} associe a chaque model point.

\code{flux} : une liste contenant pour chaque model point les montants de primes brutes \code{pri\_brut},
les montants de primes nettes \code{pri\_net} et les chargemenets sur primes \code{pri\_chgt}.
\end{Value}
%
\begin{Author}\relax
Prim'Act
\end{Author}
\inputencoding{utf8}
\HeaderA{calc\_qx}{Calcule le taux de deces.}{calc.Rul.qx}
\aliasA{ParamTableMort}{calc\_qx}{ParamTableMort}
%
\begin{Description}\relax
\code{calc\_qx} est une methode permettant de calculer le taux de deces.
\end{Description}
%
\begin{Usage}
\begin{verbatim}
calc_qx(table_mort, age, gen)
\end{verbatim}
\end{Usage}
%
\begin{Arguments}
\begin{ldescription}
\item[\code{table\_mort}] un objet de la classe \code{\LinkA{ParamTableMort}{ParamTableMort}} contenant la table de mortalite.

\item[\code{age}] une valeur \code{numeric} correspondant a l'age.

\item[\code{gen}] une valeur \code{numeric} correspondant a la generation.
\end{ldescription}
\end{Arguments}
%
\begin{Value}
La valeur du taux de deces calcule.
\end{Value}
%
\begin{Author}\relax
Prim'Act
\end{Author}
\inputencoding{utf8}
\HeaderA{calc\_rach}{Calcule le taux de rachat.}{calc.Rul.rach}
\aliasA{ParamTableRach}{calc\_rach}{ParamTableRach}
%
\begin{Description}\relax
\code{calc\_rach} est une methode permettant de calculer le taux de rachat.
\end{Description}
%
\begin{Usage}
\begin{verbatim}
calc_rach(table_rach, age, anc)
\end{verbatim}
\end{Usage}
%
\begin{Arguments}
\begin{ldescription}
\item[\code{table\_rach}] un objet de la classe \code{\LinkA{ParamTableRach}{ParamTableRach}} contenant la table de rachat.

\item[\code{age}] une valeur \code{numeric} correspondant a l'age.

\item[\code{anc}] une valeur \code{numeric} correspondant a l'anciennete.
\end{ldescription}
\end{Arguments}
%
\begin{Value}
La valeur du taux de rachat calcule.
\end{Value}
%
\begin{Author}\relax
Prim'Act
\end{Author}
\inputencoding{utf8}
\HeaderA{calc\_rach\_dyn}{Calcule la composante rachats dynamique.}{calc.Rul.rach.Rul.dyn}
\aliasA{ParamRachDyn}{calc\_rach\_dyn}{ParamRachDyn}
%
\begin{Description}\relax
\code{calc\_rach\_dyn} est une methode permettant de calculer la composante rachat dynamique
selon la methodologie transmise dans le ONC de l'ACPR de 2013.
\end{Description}
%
\begin{Usage}
\begin{verbatim}
calc_rach_dyn(p, tx_cible, tx_serv)
\end{verbatim}
\end{Usage}
%
\begin{Arguments}
\begin{ldescription}
\item[\code{p}] un objet de la classe \code{\LinkA{ParamRachDyn}{ParamRachDyn}} contenant les parametres de rachats dynamiques.

\item[\code{tx\_cible}] une valeur \code{numeric} correspondant au taux de revalorisation cible.

\item[\code{tx\_serv}] une valeur \code{numeric} correspondant au taux de revalorisation servi.
\end{ldescription}
\end{Arguments}
%
\begin{Value}
La valeur du taux rachat.
\end{Value}
%
\begin{Author}\relax
Prim'Act
\end{Author}
\inputencoding{utf8}
\HeaderA{calc\_RC}{Calcul de la RC.}{calc.Rul.RC}
\aliasA{RC}{calc\_RC}{RC}
%
\begin{Description}\relax
\code{calc\_RC} est une methode permettant de calculer le montant de RC.
\end{Description}
%
\begin{Usage}
\begin{verbatim}
calc_RC(x, pmvr_oblig)
\end{verbatim}
\end{Usage}
%
\begin{Arguments}
\begin{ldescription}
\item[\code{x}] objet de la classe \code{RC}, necessaire pour connaitre le stock de RC initial.

\item[\code{pmr\_oblig}] est un \code{numeric} correspondant au montant global annuel de plus ou moins values realisees sur des actifs obligataires.
\end{ldescription}
\end{Arguments}
%
\begin{Value}
Le format de la liste renvoyee est :
\begin{description}

\item[\code{RC\_courante} : ]  valeur de la RC courante initiale augmentee des plus ou moins values annuelles realisees
\item[\code{var\_RC} : ]  variation de la RC courante.

\end{description}

\end{Value}
%
\begin{Author}\relax
Prim'Act
\end{Author}
\inputencoding{utf8}
\HeaderA{calc\_rdt}{Calcul les rendements de chacune des composante des sous-portefeuilles action et immobilier du portefeuille PortFin.}{calc.Rul.rdt}
\aliasA{PortFin}{calc\_rdt}{PortFin}
%
\begin{Description}\relax
\code{calc\_rdt} est une methode permettant de calculer les rendements des portfeuilles Action et Immo d'un objet PortFin.
\end{Description}
%
\begin{Usage}
\begin{verbatim}
calc_rdt(x, mp_ESG)
\end{verbatim}
\end{Usage}
%
\begin{Arguments}
\begin{ldescription}
\item[\code{x}] objet de la classe PortFin.

\item[\code{mp\_ESG}] objet de la classe ModelPointESG decrivant les conditions de l'annee n ( ainsi que l'annee n-1 pour les indices actions \& immo).
\end{ldescription}
\end{Arguments}
%
\begin{Value}
Un data frame compose de deux colonnes et autant de lignes que le portefeuille action a de lignes.
\end{Value}
%
\begin{Author}\relax
Prim'Act
\end{Author}
\inputencoding{utf8}
\HeaderA{calc\_rdt\_marche\_ref}{Calcul du taux de rendement de reference au niveau du marche}{calc.Rul.rdt.Rul.marche.Rul.ref}
\aliasA{PortPassif}{calc\_rdt\_marche\_ref}{PortPassif}
%
\begin{Description}\relax
\code{calc\_rdt\_marche\_ref} est une methode permettant de calculer un taux cible.
\end{Description}
%
\begin{Usage}
\begin{verbatim}
calc_rdt_marche_ref(x, mp_esg)
\end{verbatim}
\end{Usage}
%
\begin{Arguments}
\begin{ldescription}
\item[\code{mp\_esg}] est un objet de type \code{ModelPointESG}, qui represente la situation courante
en annee et simulations des valeurs de l'ESG.

\item[\code{param\_comport}] un objet de la classe \code{ParamComport}.
\end{ldescription}
\end{Arguments}
%
\begin{Value}
Une liste contenant les rendements de reference du marche.
\end{Value}
%
\begin{Author}\relax
Prim'Act
\end{Author}
\inputencoding{utf8}
\HeaderA{calc\_reprise\_ppb}{Reprend sur la valeur de la PPB}{calc.Rul.reprise.Rul.ppb}
\aliasA{Ppb}{calc\_reprise\_ppb}{Ppb}
%
\begin{Description}\relax
\code{calc\_reprise\_ppb} est une methode permettant de reprendre sur la PPB.
La reprise est effectuee si les limites de reprise de la PPB sur l'annee ne sont pas atteintes. La valeur de cette limite est mise a jour suite a la reprise
\end{Description}
%
\begin{Usage}
\begin{verbatim}
calc_reprise_ppb(x, montant)
\end{verbatim}
\end{Usage}
%
\begin{Arguments}
\begin{ldescription}
\item[\code{x}] un objet de la classe \code{Ppb}.

\item[\code{montant}] la valeur \code{numeric} de la reprise.
\end{ldescription}
\end{Arguments}
%
\begin{Value}
\code{ppb} l'objet \code{x} mis a jour

\code{reprise} le montnant de la reprise effectuee.
\end{Value}
%
\begin{Author}\relax
Prim'Act
\end{Author}
\inputencoding{utf8}
\HeaderA{calc\_result\_technique}{calcule le resultat technique}{calc.Rul.result.Rul.technique}
\aliasA{RevaloEngine}{calc\_result\_technique}{RevaloEngine}
%
\begin{Description}\relax
\code{calc\_result\_technique} est une methode permettant de calculer le resultat technique avant attribution de
participation aux benefices.
\end{Description}
%
\begin{Usage}
\begin{verbatim}
calc_result_technique(passif_av_pb, var_pre)
\end{verbatim}
\end{Usage}
%
\begin{Arguments}
\begin{ldescription}
\item[\code{passif\_av\_pb}] est une liste produit par la methode \code{\LinkA{viellissement\_av\_pb}{viellissement.Rul.av.Rul.pb}}
appliquee a un portefeuille de passif.

\item[\code{var\_pre}] est une valeur \code{numeric} correspondant a la variation de PRE.
\end{ldescription}
\end{Arguments}
%
\begin{Value}
Le resultat technique.
\end{Value}
%
\begin{Author}\relax
Prim'Act
\end{Author}
%
\begin{SeeAlso}\relax
\code{\LinkA{PRE}{PRE}}, \code{\LinkA{viellissement\_av\_pb}{viellissement.Rul.av.Rul.pb}}.
\end{SeeAlso}
\inputencoding{utf8}
\HeaderA{calc\_result\_technique\_ap\_pb}{calcule le resultat technique apres prise en compte de la participation aux benefices.}{calc.Rul.result.Rul.technique.Rul.ap.Rul.pb}
\aliasA{Canton}{calc\_result\_technique\_ap\_pb}{Canton}
%
\begin{Description}\relax
\code{calc\_result\_technique\_ap\_pb} est une methode permettant de calculer le resultat technique
apres attribution de participation aux benefices.
\end{Description}
%
\begin{Usage}
\begin{verbatim}
calc_result_technique_ap_pb(passif_av_pb, passif_ap_pb, ppb, var_pre)
\end{verbatim}
\end{Usage}
%
\begin{Arguments}
\begin{ldescription}
\item[\code{passif\_av\_pb}] est une liste produit par la methode \code{\LinkA{viellissement\_av\_pb}{viellissement.Rul.av.Rul.pb}}.

\item[\code{passif\_ap\_pb}] est une liste produit par la methode \code{\LinkA{viellissement\_ap\_pb}{viellissement.Rul.ap.Rul.pb}}.

\item[\code{ppb}] est un objet de la classe \code{\LinkA{Ppb}{Ppb}} qui renvoie l'etat courant de la PPB.

\item[\code{var\_pre}] est une valeur \code{numeric} correspondant a la variation de PRE.
\end{ldescription}
\end{Arguments}
%
\begin{Value}
Le resultat technique apres participation aux benefices.
\end{Value}
\inputencoding{utf8}
\HeaderA{calc\_revalo}{Applique la politique de revalorisation d'un canton.}{calc.Rul.revalo}
\aliasA{RevaloEngine}{calc\_revalo}{RevaloEngine}
%
\begin{Description}\relax
\code{calc\_revalo} est une methode permettant de
d'appliquer l'ensemble de la politique de revalorisation d'un assureur.
\end{Description}
%
\begin{Usage}
\begin{verbatim}
calc_revalo(x, passif_av_pb, tra, plac_moy_vnc, result_tech)
\end{verbatim}
\end{Usage}
%
\begin{Arguments}
\begin{ldescription}
\item[\code{x}] un objet de la classe \code{\LinkA{Canton}{Canton}}.

\item[\code{passif\_av\_pb}] est une liste produit par la methode \code{\LinkA{viellissement\_av\_pb}{viellissement.Rul.av.Rul.pb}}
appliquee a un portefeuille de passif.

\item[\code{tra}] est la valeur \code{numeric} du taux de rendement de l'actif.

\item[\code{plac\_moy\_vnc}] est la valeur \code{numeric} moyenne des actifs en valeur nette comptable.

\item[\code{result\_tech}] est la valeur \code{numeric} du resultat technique prise en compte avant distribution
de la PB.
\end{ldescription}
\end{Arguments}
%
\begin{Value}
\code{add\_rev\_nette\_stock} une liste avec la valeur de la revalorisation nette servie par
produit au titre de la participation aux benefices.

\code{pmvl\_liq} le montant de plus-values latentes en actions a realiser.

\code{ppb} un objet \code{\LinkA{Ppb}{Ppb}} correspondant a la PPB mise a jour.

\code{tx\_pb} un vecteur reprenant les taux de PB par produit renseigne dans l'objet \code{x}.

\code{tx\_enc\_moy} un vecteur reprenant les taux de chargement sur encours theoriques moyens
par produit.
\end{Value}
%
\begin{Author}\relax
Prim'Act
\end{Author}
%
\begin{SeeAlso}\relax
Le calcul du TRA : \code{\LinkA{calc\_tra}{calc.Rul.tra}}.
Le vieillissemennt des passifs avant PB : \code{\LinkA{viellissement\_av\_pb}{viellissement.Rul.av.Rul.pb}}.
Le calcul du resultat technique avant PB : \code{\LinkA{calc\_result\_technique}{calc.Rul.result.Rul.technique}}.
Le calcul de la base de produits financiers : \code{\LinkA{base\_prod\_fin}{base.Rul.prod.Rul.fin}}.
Le calcul de la PB contractuelle : \code{\LinkA{pb\_contr}{pb.Rul.contr}}.
Le financement des TMG par la PPB : \code{\LinkA{finance\_tmg}{finance.Rul.tmg}}.
Le financement du taux cible par la PPB : \code{\LinkA{finance\_cible\_ppb}{finance.Rul.cible.Rul.ppb}}
Le financement du taux cible par la realisation plus-values latentes actions : \code{\LinkA{finance\_cible\_pmvl}{finance.Rul.cible.Rul.pmvl}}
Le financement du taux cible par la compression de la marge de l'assureur : \code{\LinkA{finance\_cible\_marge}{finance.Rul.cible.Rul.marge}}
Le calcul de la marge de l'assureur : \code{\LinkA{calc\_marge\_fin}{calc.Rul.marge.Rul.fin}}
L'application de la contrainte legale de participation aux benefices : \code{\LinkA{finance\_contrainte\_legale}{finance.Rul.contrainte.Rul.legale}}
\end{SeeAlso}
\inputencoding{utf8}
\HeaderA{calc\_revalo\_pm}{Calcule et applique la revalorisation pour des PM pour des contrats epargne en euros.}{calc.Rul.revalo.Rul.pm}
\aliasA{EpEuroInd}{calc\_revalo\_pm}{EpEuroInd}
%
\begin{Description}\relax
\code{calc\_revalo\_pm} est une methode permettant de calculer la revallorisation des PM sur une annee.
\end{Description}
%
\begin{Usage}
\begin{verbatim}
calc_revalo_pm(x, rev_net_alloue, tx_soc)
\end{verbatim}
\end{Usage}
%
\begin{Arguments}
\begin{ldescription}
\item[\code{x}] un objet de la classe \code{\LinkA{EpEuroInd}{EpEuroInd}} contenant les model points epargne euros.

\item[\code{rev\_net\_alloue}] une valeur de type \code{numeric} correspondant au montant de revalorisation a allouer.

\item[\code{tx\_soc}] est une valeur \code{numeric} correspondant au taux de prelevement sociaux.
\end{ldescription}
\end{Arguments}
%
\begin{Details}\relax
Cette methode permet de calculer les montants de PM de fin d'annee avec une revalorisation
minimale et une revalorisation additionnelle au titre de la participation aux benefices de l'annee.
Les chargements sur encours sont egalement calcules et preleves.
Cette methode permet de gerer les contrats a taux de revalorisation net negatif.
\end{Details}
%
\begin{Value}
Une liste contenant :
\begin{description}

\item[\code{flux} : ] une liste comprenant les flux de l'annee
\item[\code{stock} : ] une liste comprenant les nombres de sorties
\item[\code{tx\_rev\_net} : ] un vecteur correspondant au taux de revalorisation net appliques
a chaque model point.

\end{description}


Le format de la liste \code{flux} est :
\begin{description}

\item[\code{rev\_stock\_brut\_ap\_pb} : ] un vecteur contenant la revalorisation
brute de l'annee appliquee au PM
\item[\code{rev\_stock\_nette\_ap\_pb} : ] un vecteur contenant la revalorisation
nette de l'annee appliquee au PM. Elle peut etre negative pour des contrats a taux negatif.
\item[\code{enc\_charg\_stock\_ap\_pb} : ] un vecteur contenant les montants de chargement sur encours
de l'annee calcules pour le stock de PM
\item[\code{soc\_stock\_ap\_pb} : ] un vecteur contenant les prelevements sociaux de l'annee

\end{description}


Le format de la liste \code{stock} est :
s\begin{description}

\item[\code{pm\_fin\_ap\_pb : }] un vecteur contenant le montant de PM en fin d'annee

\end{description}

\end{Value}
%
\begin{Author}\relax
Prim'Act
\end{Author}
%
\begin{SeeAlso}\relax
Le calcul des PM avec revalorisation minimale uniquement \code{\LinkA{calc\_pm}{calc.Rul.pm}}.
\end{SeeAlso}
\inputencoding{utf8}
\HeaderA{calc\_sur\_dec}{Calcul les surcote/decote de chaque composante d'un portefeuille obligataire.}{calc.Rul.sur.Rul.dec}
\aliasA{Oblig}{calc\_sur\_dec}{Oblig}
%
\begin{Description}\relax
\code{calc\_sur\_dec} est une methode permettant de calculer les surcotes/decotes de chaque composante d'un portefeuille obligataire.
\end{Description}
%
\begin{Usage}
\begin{verbatim}
calc_sur_dec(x)
\end{verbatim}
\end{Usage}
%
\begin{Arguments}
\begin{ldescription}
\item[\code{x}] objet de la classe Oblig (decrivant le portefeuille obligataire).
\end{ldescription}
\end{Arguments}
%
\begin{Value}
Un data.frame compose de deux colonnes : 1 ere colonne : surcotes decotes ; 2de colonne : valeurs nettes comptables.
\end{Value}
%
\begin{Author}\relax
Prim'Act
\end{Author}
\inputencoding{utf8}
\HeaderA{calc\_tra}{Calcul du taux de rendement financier}{calc.Rul.tra}
\aliasA{PortFin}{calc\_tra}{PortFin}
%
\begin{Description}\relax
\code{calc\_tra} est une methode permettant de calculer le taux de rendement financier du portefeuille.
\end{Description}
%
\begin{Usage}
\begin{verbatim}
calc_tra(plac_moy, res_fin)
\end{verbatim}
\end{Usage}
%
\begin{Arguments}
\begin{ldescription}
\item[\code{plac\_moy}] est un objet de type \code{numeric}, qui fournit la valeur moyenne
des placements de l'annee en valeur nette comptable.

\item[\code{res\_fin}] est un objet de type \code{numeric}, qui fournit le resultat financier du porfeuille.
\end{ldescription}
\end{Arguments}
%
\begin{Value}
La valeur du taux de rendement de l'actif.
\end{Value}
%
\begin{Author}\relax
Prim'Act
\end{Author}
\inputencoding{utf8}
\HeaderA{calc\_tx\_cible}{Calcul du taux cible pour des contrats epargne en euros.}{calc.Rul.tx.Rul.cible}
\aliasA{EpEuroInd}{calc\_tx\_cible}{EpEuroInd}
%
\begin{Description}\relax
\code{calc\_tx\_cible} est une methode permettant d'evaluer le taux de revalorisation cible
de chaque model point.
\end{Description}
%
\begin{Usage}
\begin{verbatim}
calc_tx_cible(x, ht, list_rd)
\end{verbatim}
\end{Usage}
%
\begin{Arguments}
\begin{ldescription}
\item[\code{x}] un objet de la classe \code{\LinkA{EpEuroInd}{EpEuroInd}} contenant les model points epargne euros.

\item[\code{ht}] un objet de la classe \code{\LinkA{HypTech}{HypTech}} contenant differentes lois de comportement.

\item[\code{list\_rd}] une liste contenant les rendements de reference. Le format de cette liste est :
\begin{description}

\item[le taux de rendement obligataire] 
\item[le taux de rendement de l'indice action de reference] 
\item[le taux de rendement de l'indice immobilier de reference] 
\item[le taux de rendement de l'indice tresorerie de reference] 

\end{description}

\end{ldescription}
\end{Arguments}
%
\begin{Value}
\code{tx\_cible\_an} : un vecteur contenant les taux cible de l'annee

\code{tx\_cible\_se} : un vecteur contenant les taux cible de l'annee sur base semestrielle
\end{Value}
%
\begin{Note}\relax
Pour les besoins des calculs a mi-annee, des taux semestriels sont produits.
\end{Note}
%
\begin{Author}\relax
Prim'Act
\end{Author}
%
\begin{SeeAlso}\relax
La recuperation des taux cibles calcules : \code{\LinkA{get\_comport}{get.Rul.comport}}.
\end{SeeAlso}
\inputencoding{utf8}
\HeaderA{calc\_tx\_cible\_ref\_marche}{Calcule le taux de revalorisation cible.}{calc.Rul.tx.Rul.cible.Rul.ref.Rul.marche}
\aliasA{ParamComport}{calc\_tx\_cible\_ref\_marche}{ParamComport}
%
\begin{Description}\relax
\code{calc\_tx\_cible\_ref\_marche} est une methode permettant de calculer le taux de revalorisation cible
en evaluant le taux de rendement des assureurs sur le marche.
\end{Description}
%
\begin{Usage}
\begin{verbatim}
calc_tx_cible_ref_marche(param_comport, list_rd, tx_cible_prec)
\end{verbatim}
\end{Usage}
%
\begin{Arguments}
\begin{ldescription}
\item[\code{param\_comport}] un objet de la classe \code{\LinkA{ParamComport}{ParamComport}} contenant les parametres
comportementaux.

\item[\code{list\_rd}] une liste contenant les rendements de reference. Le format de cette liste est :
\begin{description}

\item[le taux de rendement obligataire] 
\item[le taux de rendement de l'indice action de reference] 
\item[le taux de rendement de l'indice immobilier de reference] 
\item[le taux de rendement de l'indice tresorerie de reference] 

\end{description}


\item[\code{tx\_cible\_prec}] une valeur \code{numeric} correspondant au taux cible de la periode precedente.
\end{ldescription}
\end{Arguments}
%
\begin{Value}
La valeur du taux cible.
\end{Value}
%
\begin{Author}\relax
Prim'Act
\end{Author}
\inputencoding{utf8}
\HeaderA{calc\_tx\_min}{Calcul le taux de revalorisation contractuel minimum pour des contrats epargne en euros.}{calc.Rul.tx.Rul.min}
\aliasA{EpEuroInd}{calc\_tx\_min}{EpEuroInd}
%
\begin{Description}\relax
\code{calc\_tx\_min} est une methode permettant de calculer les taux de revalorisation minimum
sur une periode. La revalorisation minimum est le maximum entre le taux technique et
le taux minimim garanti (TMG) du contrat.
\end{Description}
%
\begin{Usage}
\begin{verbatim}
calc_tx_min(x, an)
\end{verbatim}
\end{Usage}
%
\begin{Arguments}
\begin{ldescription}
\item[\code{x}] un objet de la classe \code{\LinkA{EpEuroInd}{EpEuroInd}} contenant les model points epargne euros.

\item[\code{an}] un \code{numeric} representant l'annee de projection courante.
\end{ldescription}
\end{Arguments}
%
\begin{Value}
\code{tx\_tech\_an} : un vecteur contenant les taux de technique de l'annee

\code{tx\_tech\_se} : un vecteur contenant les taux de technique de l'annee sur base semestrielle

\code{tx\_an} : un vecteur contenant les taux de revalorisation minimum de l'annee

\code{x\_se} : un vecteur contenant les taux de revalorisation minimum de l'annee exprimes en semestriel.
\end{Value}
%
\begin{Note}\relax
Pour les besoins des calculs a mi-annee, des taux semestriels sont produits.
\end{Note}
%
\begin{Author}\relax
Prim'Act
\end{Author}
\inputencoding{utf8}
\HeaderA{calc\_tx\_sortie}{Calcul des taux de sortie pour des contrats epargne en euros.}{calc.Rul.tx.Rul.sortie}
\aliasA{EpEuroInd}{calc\_tx\_sortie}{EpEuroInd}
%
\begin{Description}\relax
\code{calc\_tx\_sortie} est une methode permettant de calculer les differents taux de sortie
sur une periode.
\end{Description}
%
\begin{Usage}
\begin{verbatim}
calc_tx_sortie(x, ht)
\end{verbatim}
\end{Usage}
%
\begin{Arguments}
\begin{ldescription}
\item[\code{x}] un objet de la classe \code{\LinkA{EpEuroInd}{EpEuroInd}} contenant les model points epargne euros.

\item[\code{ht}] un objet de la classe \code{\LinkA{HypTech}{HypTech}} contenant differentes tables de mortalite et differentes
lois de rachat.
\end{ldescription}
\end{Arguments}
%
\begin{Value}
Une matrice contenant pour chaque model points en ligne :
\begin{description}

\item[\code{qx\_rach\_tot} : ] un vecteur contenant les taux de rachats totaux
\item[\code{qx\_rach\_tot\_dyn} : ] un vecteur contenant les taux de rachats totaux dynamiques
\item[\code{qx\_dc} : ] un vecteur contenant les taux de deces
\item[\code{qx\_rach\_part} : ] un vecteur contenant les taux de rachats partiels
\item[\code{qx\_rach\_part\_dyn} : ] un vecteur contenant les taux de rachats partiels dynamiques.

\end{description}

\end{Value}
%
\begin{Author}\relax
Prim'Act
\end{Author}
%
\begin{SeeAlso}\relax
La recuperation des taux de rachat structurel : \code{\LinkA{get\_qx\_rach}{get.Rul.qx.Rul.rach}}.
La recuperation des taux de rachat dynamique : \code{\LinkA{get\_rach\_dyn}{get.Rul.rach.Rul.dyn}}.
La recuperation des taux de deces : \code{\LinkA{get\_qx\_mort}{get.Rul.qx.Rul.mort}}.
\end{SeeAlso}
\inputencoding{utf8}
\HeaderA{calc\_vm\_action}{Calcul les valeurs de marches de chaque composante du portefeuille action.}{calc.Rul.vm.Rul.action}
\aliasA{Action}{calc\_vm\_action}{Action}
%
\begin{Description}\relax
\code{calc\_vm\_action} est une methode permettant de calculer les valeurs de marche.
\end{Description}
%
\begin{Usage}
\begin{verbatim}
calc_vm_action(x, rdt)
\end{verbatim}
\end{Usage}
%
\begin{Arguments}
\begin{ldescription}
\item[\code{x}] objet de la classe \code{Action} (decrivant le portefeuille d'action).

\item[\code{rdt}] vecteur de type \code{numeric} decrivant le rendement de chacune des actions du portefeuille action de l'assureur.
Contient autant d'elements que le portefeuille action a de lignes.
\end{ldescription}
\end{Arguments}
%
\begin{Value}
L'objet \code{x} dont les valeurs de marche ont ete mises a jour.
\end{Value}
%
\begin{Author}\relax
Prim'Act
\end{Author}
\inputencoding{utf8}
\HeaderA{calc\_vm\_immo}{Calcul les valeurs de marches de chaque composante du portefeuille immobilier.}{calc.Rul.vm.Rul.immo}
\aliasA{Immo}{calc\_vm\_immo}{Immo}
%
\begin{Description}\relax
\code{calc\_vm\_immo} est une methode permettant de calculer les valeurs de marche.
\end{Description}
%
\begin{Usage}
\begin{verbatim}
calc_vm_immo(x, rdt)
\end{verbatim}
\end{Usage}
%
\begin{Arguments}
\begin{ldescription}
\item[\code{x}] objet de la classe \code{Immo} (decrivant le portefeuille d'immobilier).

\item[\code{rdt}] vecteur de type \code{numeric} decrivant le rendement de chacune des lignes d'immobilier du portefeuille immobilier de l'assureur.
Contient autant d'elements que le portefeuille immobilier a de lignes.
\end{ldescription}
\end{Arguments}
%
\begin{Value}
L'objet \code{x} dont les valeurs de marche ont ete mises a jour.
\end{Value}
%
\begin{Author}\relax
Prim'Act
\end{Author}
\inputencoding{utf8}
\HeaderA{calc\_vm\_oblig}{Calcul les valeurs de marches de chaque composante du portefeuille obligation.}{calc.Rul.vm.Rul.oblig}
\aliasA{Oblig}{calc\_vm\_oblig}{Oblig}
%
\begin{Description}\relax
\code{calc\_vm\_oblig} est une methode permettant de calculer les valeurs de marche.
\end{Description}
%
\begin{Usage}
\begin{verbatim}
calc_vm_oblig(x, yield_curve)
\end{verbatim}
\end{Usage}
%
\begin{Arguments}
\begin{ldescription}
\item[\code{x}] objet de la classe \code{Oblig} (decrivant le portefeuille d'obligation).

\item[\code{yield\_curve}] vecteur de type \code{numeric} contenant la courbe de taux 
(cf. l'attribut \code{yield\_curve} des objets de la classe \code{\LinkA{ModelPointESG}{ModelPointESG}}).
\end{ldescription}
\end{Arguments}
%
\begin{Value}
L'objet \code{x} dont les valeurs de marche ont ete mises a jour.
\end{Value}
%
\begin{Author}\relax
Prim'Act
\end{Author}
\inputencoding{utf8}
\HeaderA{calc\_vm\_treso}{Calcul les valeurs de marches de chaque composante du portefeuille treso.}{calc.Rul.vm.Rul.treso}
\aliasA{Treso}{calc\_vm\_treso}{Treso}
%
\begin{Description}\relax
\code{calc\_vm\_treso} est une methode permettant de calculer les valeurs de marche de chaque ligne du portefeuille treso.
\end{Description}
%
\begin{Usage}
\begin{verbatim}
calc_vm_treso(x, rdt, flux_milieu, flux_fin)
\end{verbatim}
\end{Usage}
%
\begin{Arguments}
\begin{ldescription}
\item[\code{x}] objet de la classe treso (decrivant le portefeuille de treso).

\item[\code{rdt}] vecteur decrivant le rendement de chacune des lignes treso du ptf.
Contient autant d'elements que le portefeuille a de lignes.

\item[\code{flux\_milieu}] vecteur decrivant les flux ( percus)entrants : positif, sortants : negatifs) en milieu d'annee, ventiles selon chacune des lignes de cash.

\item[\code{flux\_fin}] vecteur decrivant les flux (entrants : positifs, sortants : negatifs) en fin d'annee, ventiles selon chacune des lignes de cash.
\end{ldescription}
\end{Arguments}
%
\begin{Value}
L'objet x dont les valeurs de marche ont ete mises a jour.
\end{Value}
%
\begin{Author}\relax
Prim'Act
\end{Author}
\inputencoding{utf8}
\HeaderA{calc\_vnc}{Calcul les valeurs nettes comptables de chaque composante du portefeuille obligation.}{calc.Rul.vnc}
\aliasA{Oblig}{calc\_vnc}{Oblig}
%
\begin{Description}\relax
\code{calc\_vnc} est une methode permettant de calculer les valeurs de marche.
\end{Description}
%
\begin{Usage}
\begin{verbatim}
calc_vnc(x, sd_unitaire)
\end{verbatim}
\end{Usage}
%
\begin{Arguments}
\begin{ldescription}
\item[\code{x}] objet de la classe \code{Oblig} (decrivant le portefeuille d'obligation).

\item[\code{sd\_unitaire}] vecteur de type \code{numeric} decrivant la surcote decote de chacune des lignes d'obligation du portefeuille obligation de l'assureur.
Contient autant d'elements que le portefeuille a de lignes.
\end{ldescription}
\end{Arguments}
%
\begin{Value}
L'objet \code{x} dont les valeurs nettes comptables ont ete mises a jour.
\end{Value}
%
\begin{Author}\relax
Prim'Act
\end{Author}
\inputencoding{utf8}
\HeaderA{calc\_z\_spread}{Calcul les zeros spreads de chaque composante d'un portefeuille obligataire.}{calc.Rul.z.Rul.spread}
\aliasA{Oblig}{calc\_z\_spread}{Oblig}
%
\begin{Description}\relax
\code{calc\_z\_spread} est une methode permettant de calculer les zeros spread de chaque composante d'un portefeuille obligataire.
\end{Description}
%
\begin{Usage}
\begin{verbatim}
calc_z_spread(x, yield_curve)
\end{verbatim}
\end{Usage}
%
\begin{Arguments}
\begin{ldescription}
\item[\code{x}] objet de la classe Oblig (decrivant le portefeuile obligataire).

\item[\code{yield\_curve}] vecteur decrivant la courbe de taux sans risque retenue.
\end{ldescription}
\end{Arguments}
%
\begin{Value}
Un vecteur dont chaque element correspond a la valeur du zero spread de l'obligation du portefeuille obligataire.
Ce vecteur a autant d'elements que le portefeuille obligataire a de lignes.
\end{Value}
%
\begin{Author}\relax
Prim'Act
\end{Author}
\inputencoding{utf8}
\HeaderA{Canton}{La classe \code{Canton}.}{Canton}
\keyword{classes}{Canton}
%
\begin{Description}\relax
Une classe pour le canton d'un assureur. Un objet de cette classe agrege un portefeuille financier,
un portefeuille de passifs, l'ensemble des autres provisions ainsi que les parametres et donnees necessaires
a la projection de la situation d'un l'assureur.
\end{Description}
%
\begin{Section}{Slots}

\begin{description}

\item[\code{annee}] une valeur entiere correspondant a l'annee de projection.

\item[\code{ptf\_fin}] est un objet de type \code{\LinkA{PortFin}{PortFin}},
qui represente le portefeuille d'investissement d'un canton.

\item[\code{ptf\_passif}] est un objet de type \code{\LinkA{PortPassif}{PortPassif}},
qui represente le portefeuille de passif d'un canton.

\item[\code{mp\_esg}] est un objet de type \code{\LinkA{ModelPointESG}{ModelPointESG}},
qui represente la situation courante deduite de l'ESG. Cet objet traduit la situation economique
pour une annee donnee et une simulation donnee.

\item[\code{ppb}] est un objet de type \code{\LinkA{Ppb}{Ppb}},
qui represente la provision pour participation aux benefices (PPB).

\item[\code{hyp\_canton}] est un objet de type \code{\LinkA{HypCanton}{HypCanton}},
qui regroupe les hypotheses generales applicables au canton.

\item[\code{param\_alm}] est un objet de type \code{\LinkA{ParamAlmEngine}{ParamAlmEngine}},
qui contient les parametres utilises dans les methodes de gestion de l'allocation d'actifs.

\item[\code{param\_revalo}] est un objet de type \code{\LinkA{ParamRevaloEngine}{ParamRevaloEngine}},
qui contient les parametres utilises dans les methodes de gestion de la revalorisation.

\end{description}
\end{Section}
%
\begin{Author}\relax
Prim'Act
\end{Author}
%
\begin{SeeAlso}\relax
La projection du \code{Canton} sur une annee : \code{\LinkA{proj\_an}{proj.Rul.an}}.
Le calcul du resultat technique : \code{\LinkA{calc\_result\_technique\_ap\_pb}{calc.Rul.result.Rul.technique.Rul.ap.Rul.pb}}.
Le calcul des fins de projection : \code{\LinkA{calc\_fin\_proj}{calc.Rul.fin.Rul.proj}}.
\end{SeeAlso}
\inputencoding{utf8}
\HeaderA{chargement\_choc}{Permet de charger les parametres de choc de la formule standard.}{chargement.Rul.choc}
\aliasA{ChocSolvabilite2}{chargement\_choc}{ChocSolvabilite2}
%
\begin{Description}\relax
\code{chargement\_choc} est une methode permettant de charger les parametres
l'ensemble des parametres necessaires a la bonne application des chocs de marche et
de souscription au sens de la formule standard de la directive Solvabilite 2,
tels que renseignes par l'utilisateur.
\end{Description}
%
\begin{Usage}
\begin{verbatim}
chargement_choc(x, folder_chocs_address)
\end{verbatim}
\end{Usage}
%
\begin{Arguments}
\begin{ldescription}
\item[\code{x}] objet de la classe \code{\LinkA{ChocSolvabilite2}{ChocSolvabilite2}}.

\item[\code{folder\_choc\_address}] est un \code{character}. Cette chaine de caractere est construite par la methode
\code{\LinkA{set\_architecture}{set.Rul.architecture}} de la classe \code{\LinkA{Initialisation}{Initialisation}}.
Elle contient l'adresse du dossier contenant les fichiers de parametres des chocs de la formule standard a appliquer.
Ces derniers doivent etre renseignes par l'utilisateur.
\end{ldescription}
\end{Arguments}
%
\begin{Value}
\code{x} l'objet  de la classe \bsl{}code\LinkA{ChocSolvabilite2}{ChocSolvabilite2} dont les attributs \code{param\_choc\_mket}
et \code{param\_choc\_sousc} ont ete mis a jour.
\end{Value}
%
\begin{Author}\relax
Prim'Act
\end{Author}
%
\begin{SeeAlso}\relax
La creation de l'architecture de chargement des donnees et parametres renseignes par l'utilisateur
\code{\LinkA{set\_architecture}{set.Rul.architecture}},
ainsi que les classes \code{\LinkA{ParamChocMket}{ParamChocMket}} et \code{\LinkA{ParamChocSousc}{ParamChocSousc}}.
\end{SeeAlso}
\inputencoding{utf8}
\HeaderA{chargement\_ESG}{Cette methode charge les tables de simulations d'un ESG.}{chargement.Rul.ESG}
\aliasA{ESG}{chargement\_ESG}{ESG}
%
\begin{Description}\relax
\code{chargement\_ESG} est une methode permettant de charger les trajectoires simulees par le generateur de
scenarios economiques (ESG) de Prim'Act et d'alimenter un objet \code{\LinkA{ESG}{ESG}}.
\end{Description}
%
\begin{Usage}
\begin{verbatim}
chargement_ESG(folder_ESG_address, nb_simu, nb_annee_proj)
\end{verbatim}
\end{Usage}
%
\begin{Arguments}
\begin{ldescription}
\item[\code{folder\_ESG\_address}] est un \code{character}. Il correspond au chemin de reference du dossier contenant
les extractions de l'ESG Prim'Act.

\item[\code{nb\_simu}] est une valeur de type \code{integer} correspondant au nombre de trajectoire
simulees par l'ESG Prim'Act.

\item[\code{nb\_annee\_proj}] est une valeur de type \code{integer} correspondant au nombre d'annees de projection
des sorties de l'ESG Prim'Act.
\end{ldescription}
\end{Arguments}
%
\begin{Details}\relax
Les differentes adresses potentielles pour les differents ESG employes (central, hausse de taux, baisse de taux)
sont construites par la fonction \code{\LinkA{set\_architecture}{set.Rul.architecture}} de la classe \code{\LinkA{Initialisation}{Initialisation}}.
\end{Details}
%
\begin{Value}
\code{x} l'objet de la classe \code{ESG} construit.
\end{Value}
%
\begin{Author}\relax
Prim'Act
\end{Author}
\inputencoding{utf8}
\HeaderA{chargement\_PortFin}{Charge le PortFin a partir des donnees renseignees par l'utilisateur.}{chargement.Rul.PortFin}
\aliasA{PortFin}{chargement\_PortFin}{PortFin}
%
\begin{Description}\relax
\code{chargement\_PortFin} est une methode permettant de creer un objet \code{PortFin} a partir des donnees renseignees par l'utilisateur.
\end{Description}
%
\begin{Usage}
\begin{verbatim}
chargement_PortFin(folder_PortFin_address, mp_ESG)
\end{verbatim}
\end{Usage}
%
\begin{Arguments}
\begin{ldescription}
\item[\code{folder\_PortFin\_address}] est un chemin de type \code{character}, cf la methode \code{\LinkA{set\_architecture}{set.Rul.architecture}}

\item[\code{mp\_ESG}] est un objet de la classe \code{ModelPointESG}, qui fournit le resultat financier du porfeuille.
\end{ldescription}
\end{Arguments}
%
\begin{Value}
L'objet \code{PortFin} tel que precise par les donnees initiales et les parametres renseignes par l'utilisateur.
\end{Value}
%
\begin{Author}\relax
Prim'Act
\end{Author}
\inputencoding{utf8}
\HeaderA{chargement\_PortFin\_reference}{Charge le PortFin de reinvestissement a partir des donnees renseignees par l'utilisateur.}{chargement.Rul.PortFin.Rul.reference}
\aliasA{PortFin}{chargement\_PortFin\_reference}{PortFin}
%
\begin{Description}\relax
\code{chargement\_PortFin\_reference} est une methode permettant de creer un objet \code{PortFin} correspondant au portefeuille finanicer de reinvestissement
a partir des donnees renseignees par l'utilisateur.
\end{Description}
%
\begin{Usage}
\begin{verbatim}
chargement_PortFin_reference(folder_PortFin_reference_address, mp_ESG)
\end{verbatim}
\end{Usage}
%
\begin{Arguments}
\begin{ldescription}
\item[\code{folder\_PortFin\_reference\_address}] est un chemin de type \code{character}, cf la methode \code{\LinkA{set\_architecture}{set.Rul.architecture}}

\item[\code{mp\_ESG}] est un objet de la classe \code{ModelPointESG}, qui fournit le resultat financier du porfeuille.
\end{ldescription}
\end{Arguments}
%
\begin{Value}
L'objet \code{PortFin} correspondant au portefeuille financier de reinvestissement
tel que precise par les donnees initiales et les parametres renseignes par l'utilisateur.
\end{Value}
%
\begin{Author}\relax
Prim'Act
\end{Author}
\inputencoding{utf8}
\HeaderA{ChocSolvabilite2}{La classe \code{ChocSolvabilite2} instancie les parametres de chocs Marche et Souscription de la formule standard de la directive Solvabilite 2.}{ChocSolvabilite2}
%
\begin{Description}\relax
La classe \code{ChocSolvabilite2} permet de realiser les principaux des scenarios de choc initiaux
au sens de la formule standard de la directive Solvabilite 2.
\end{Description}
%
\begin{Details}\relax
Cette classe contient deux attributs
qui contiennent respectivement l'ensemble des parametres necessaires a l'application des chocs Marche et Souscription.
Cette classe contient aussi l'ensemble des methodes permettant d'appliquer chacun de ces chocs individuellement
a un objet de la classe \code{\LinkA{Canton}{Canton}}. Les chocs permis sont :
\begin{description}

\item[\code{central} : ] la situation centrale
\item[\code{taux\_up} : ] le choc de taux a la hausse
\item[\code{taux\_down} : ] le choc de taux a la baisse
\item[\code{action\_type1} : ] le choc action de type 1
\item[\code{action\_type2} : ] le choc action de type 2
\item[\code{immo} : ] le choc immobilier
\item[\code{spread} : ] le choc spread sur les obligations corporates
\item[\code{mortalite} : ] le choc mortalite sur les tables de mortalite
\item[\code{longevite} : ] le choc longevite sur les tables de mortalite
\item[\code{frais} : ] le choc depenses sur le niveau des frais et l'inflation des frais
\item[\code{rachat\_up} : ] le choc de rachat a la hausse
\item[\code{rachat\_down} : ] le choc de rachat a la baisse.

\end{description}

\end{Details}
%
\begin{Section}{Slots}

\begin{description}

\item[\code{param\_choc\_mket}] un objet de la classe \code{\LinkA{ParamChocMket}{ParamChocMket}}.

\item[\code{param\_choc\_sousc}] un objet de la classe \code{\LinkA{ParamChocSousc}{ParamChocSousc}}.

\end{description}
\end{Section}
%
\begin{Author}\relax
Prim'Act
\end{Author}
%
\begin{SeeAlso}\relax
L'application des chocs de \code{taux\_up} et \code{taux\_down} : \code{\LinkA{do\_choc\_taux}{do.Rul.choc.Rul.taux}}.
L'application des chocs de \code{action\_type1} et \code{action\_type2} : \code{\LinkA{do\_choc\_action\_type1}{do.Rul.choc.Rul.action.Rul.type1}},
\code{\LinkA{do\_choc\_action\_type2}{do.Rul.choc.Rul.action.Rul.type2}}.
L'application du choc de \code{immo} : \code{\LinkA{do\_choc\_immo}{do.Rul.choc.Rul.immo}}.
L'application du choc de \code{spread} : \code{\LinkA{do\_choc\_spread}{do.Rul.choc.Rul.spread}}.
L'application du choc de \code{mortalite} : \code{\LinkA{do\_choc\_mortalite}{do.Rul.choc.Rul.mortalite}}.
L'application du choc de \code{longevite} : \code{\LinkA{do\_choc\_longevite}{do.Rul.choc.Rul.longevite}}.
L'application du choc de \code{frais} : \code{\LinkA{do\_choc\_frais}{do.Rul.choc.Rul.frais}}, \code{\LinkA{get\_choc\_inflation\_frais}{get.Rul.choc.Rul.inflation.Rul.frais}}.
L'application des chocs de \code{rachat\_up} et \code{rachat\_down} : \code{\LinkA{do\_choc\_rachat\_up}{do.Rul.choc.Rul.rachat.Rul.up}},
\code{\LinkA{do\_choc\_rachat\_down}{do.Rul.choc.Rul.rachat.Rul.down}}.
\end{SeeAlso}
\inputencoding{utf8}
\HeaderA{create\_ptf\_bought\_action}{Ajuste les quantites d'actions a acheter.}{create.Rul.ptf.Rul.bought.Rul.action}
\aliasA{AlmEngine}{create\_ptf\_bought\_action}{AlmEngine}
%
\begin{Description}\relax
\code{create\_ptf\_bought\_action} est une methode permettant d'ajuster d'un coefficient les quantites
d'actions a acheter. Cette methode est utilisee pour l'achat de nouvelles actions.
\end{Description}
%
\begin{Usage}
\begin{verbatim}
create_ptf_bought_action(x, coefficient)
\end{verbatim}
\end{Usage}
%
\begin{Arguments}
\begin{ldescription}
\item[\code{x}] objet de la classe \code{\LinkA{Action}{Action}}, correspondant au portefeuille actions de reinvestissement.
Ce portefeuille est unitaire.

\item[\code{coefficient}] un vecteur de type \code{numeric} qui a autant d'elements
que le portefeuille de reinvestissement action a de lignes.
Il correspond au coefficient a appliquer au portefeuille de reinvestissement action
pour effectuer l'achat desire.
\end{ldescription}
\end{Arguments}
%
\begin{Value}
\code{x} un objet de la classe \code{\LinkA{Action}{Action}} correspondant a une
proportion precise du portefeuille de reinvestissement action.
\end{Value}
%
\begin{Author}\relax
Prim'Act
\end{Author}
%
\begin{SeeAlso}\relax
La classe \code{\LinkA{Action}{Action}}.
\end{SeeAlso}
\inputencoding{utf8}
\HeaderA{create\_ptf\_bought\_immo}{Ajuste les quantites d'immobilier a acheter.}{create.Rul.ptf.Rul.bought.Rul.immo}
\aliasA{AlmEngine}{create\_ptf\_bought\_immo}{AlmEngine}
%
\begin{Description}\relax
\code{create\_ptf\_bought\_immo} est une methode permettant d'ajuster d'un coefficient les quantites
d'immobilier a acheter. Cette methode est utilisee pour l'achat de nouveaux titres immobilier.
\end{Description}
%
\begin{Usage}
\begin{verbatim}
create_ptf_bought_immo(x, coefficient)
\end{verbatim}
\end{Usage}
%
\begin{Arguments}
\begin{ldescription}
\item[\code{x}] objet de la classe \code{\LinkA{Immo}{Immo}}, correspondant au portefeuille immobilier de reinvestissement.
Ce portefeuille est unitaire.

\item[\code{coefficient}] est un vecteur de type \code{numeric} qui a autant d'elements
que le portefeuille de reinvestissement immo a de lignes. Il correspond au coefficient
a appliquer au portefeuille de reinvestissement immo pour effectuer l'achat desire.
\end{ldescription}
\end{Arguments}
%
\begin{Value}
\code{x} un objet de la classe \code{\LinkA{Immo}{Immo}} correspondant a une
proportion precise du portefeuille de reinvestissement immo.
\end{Value}
%
\begin{Author}\relax
Prim'Act
\end{Author}
%
\begin{SeeAlso}\relax
La classe \code{\LinkA{Immo}{Immo}}.
\end{SeeAlso}
\inputencoding{utf8}
\HeaderA{create\_ptf\_bought\_oblig}{Ajuste les quantites d'obligations a acheter.}{create.Rul.ptf.Rul.bought.Rul.oblig}
\aliasA{AlmEngine}{create\_ptf\_bought\_oblig}{AlmEngine}
%
\begin{Description}\relax
Cette methode permet d'ajuster d'un coefficient les quantites
d'obligations a acheter. Cette methode est utilisee pour l'achat de nouveaux titres obligataires.
\end{Description}
%
\begin{Usage}
\begin{verbatim}
create_ptf_bought_oblig(x, coefficient)
\end{verbatim}
\end{Usage}
%
\begin{Arguments}
\begin{ldescription}
\item[\code{x}] objet de la classe \code{\LinkA{Oblig}{Oblig}}, correspondant au portefeuille
obligataire de reinvestissement. Ce portefeuille est unitaire.

\item[\code{coefficient}] est un vecteur de type \code{numeric} qui a autant d'elements
que le portefeuille de reinvestissement obligataire a de lignes. Il correspond au coefficient
a appliquer au portefeuille de reinvestissement obligataire pour effectuer l'achat desire.
\end{ldescription}
\end{Arguments}
%
\begin{Value}
\code{x} un objet de la classe \code{\LinkA{Oblig}{Oblig}} correspondant a une proportion
precise du portefeuille de reinvestissement obligataire.
\end{Value}
%
\begin{Author}\relax
Prim'Act
\end{Author}
%
\begin{SeeAlso}\relax
La classe \code{\LinkA{Oblig}{Oblig}}.
\end{SeeAlso}
\inputencoding{utf8}
\HeaderA{do\_calc\_nb\_sold\_action}{Calcule le nombre d'actions a vendre.}{do.Rul.calc.Rul.nb.Rul.sold.Rul.action}
\aliasA{AlmEngine}{do\_calc\_nb\_sold\_action}{AlmEngine}
%
\begin{Description}\relax
Cette methode permet de calculer pour chaque ligne d'un portefeuille
action d'un assureur le nombre d'unites a vendre afin de realiser un certain montant de vente en actions.
\end{Description}
%
\begin{Usage}
\begin{verbatim}
do_calc_nb_sold_action(x, montant_vente, method_vente)
\end{verbatim}
\end{Usage}
%
\begin{Arguments}
\begin{ldescription}
\item[\code{x}] objet de la classe \code{\LinkA{Action}{Action}}, correspondant au portefeuille action de l'assureur.

\item[\code{montant\_vente}] est un reel de type \code{numeric} correspondant a un montant de vente (en valeur de marche) totale d'action que l'assureur souhaite effectuer.

\item[\code{method\_vente}] est un element de type \code{character} correspondant a methode de vente
retenue (seule la methode proportionnelle est implementee actuellement).
\end{ldescription}
\end{Arguments}
%
\begin{Value}
\code{data.frame} contenant deux colonnes \code{(num\_mp, nb\_sold)} correspondant
respectivement au numero de model point de chaque ligne action du portefeuille
et du nombre d'unite a vendre pour chacune d'entre elles.
\end{Value}
%
\begin{Author}\relax
Prim'Act
\end{Author}
%
\begin{SeeAlso}\relax
\code{\LinkA{Action}{Action}}.
\end{SeeAlso}
\inputencoding{utf8}
\HeaderA{do\_calc\_nb\_sold\_immo}{Calcule le nombre de titres immobilier a vendre.}{do.Rul.calc.Rul.nb.Rul.sold.Rul.immo}
\aliasA{AlmEngine}{do\_calc\_nb\_sold\_immo}{AlmEngine}
%
\begin{Description}\relax
Cette methode permet de calculer pour chaque ligne d'un portefeuille
immobilier d'un assureur le nombre d'unites a vendre afin de realiser un certain montant de vente immo.
\end{Description}
%
\begin{Usage}
\begin{verbatim}
do_calc_nb_sold_immo(x, montant_vente, method_vente)
\end{verbatim}
\end{Usage}
%
\begin{Arguments}
\begin{ldescription}
\item[\code{x}] objet de la classe \code{\LinkA{Immo}{Immo}}, correspondant au portefeuille immo de l'assureur.

\item[\code{montant\_vente}] est un reel de type \code{numeric} correspondant a la vente totale de vm immo
que l'assureur souhaite effectuer.

\item[\code{method\_vente}] est un element de type \code{character} correspondant a methode de vente retenue
(seule la methode proportionnelle est implementee actuellement).
\end{ldescription}
\end{Arguments}
%
\begin{Value}
\code{data.frame} contenant deux colonnes \code{(num\_mp, nb\_sold)} correspondant respectivement
au numero de model point de chaque ligne immo du portefeuille et du nombre d'unite a vendre pour chacune
d'entre elles.
\end{Value}
%
\begin{Author}\relax
Prim'Act
\end{Author}
%
\begin{SeeAlso}\relax
La classe \code{\LinkA{Immo}{Immo}}.
\end{SeeAlso}
\inputencoding{utf8}
\HeaderA{do\_calc\_nb\_sold\_oblig}{Calcule le nombre d'obligations a vendre.}{do.Rul.calc.Rul.nb.Rul.sold.Rul.oblig}
\aliasA{AlmEngine}{do\_calc\_nb\_sold\_oblig}{AlmEngine}
%
\begin{Description}\relax
Cette methode permet de calculer pour chaque ligne d'un portefeuille
obligataire d'un assureur le nombre d'unites a vendre afin de realiser un certain montant de vente obligataire.
\end{Description}
%
\begin{Usage}
\begin{verbatim}
do_calc_nb_sold_oblig(x, montant_vente, method_vente)
\end{verbatim}
\end{Usage}
%
\begin{Arguments}
\begin{ldescription}
\item[\code{x}] objet de la classe \code{\LinkA{Oblig}{Oblig}}, correspondant au portefeuille obligataire de l'assureur.

\item[\code{montant\_vente}] est un reel de type \code{numeric} correspondant a la vente totale de vm obligataire
que l'assureur souhaite effectuer.

\item[\code{method\_vente}] est un element de type \code{character} correspondant a methode de vente retenue
(seule la methode proportionnelle est implementee actuellement).
\end{ldescription}
\end{Arguments}
%
\begin{Value}
\code{data.frame} contenant deux colonnes \code{(num\_mp, nb\_sold)} correspondant respectivement
au numero de model point de chaque ligne obligataire du portefeuille et du nombre d'unite a vendre pour
chacune d'entre elles.
\end{Value}
%
\begin{Author}\relax
Prim'Act
\end{Author}
%
\begin{SeeAlso}\relax
La classe \code{\LinkA{Oblig}{Oblig}}.
\end{SeeAlso}
\inputencoding{utf8}
\HeaderA{do\_choc\_action\_type1}{Permet a partir d'un canton initial de creer un canton choque action.}{do.Rul.choc.Rul.action.Rul.type1}
\aliasA{ChocSolvabilite2}{do\_choc\_action\_type1}{ChocSolvabilite2}
%
\begin{Description}\relax
\code{do\_choc\_action\_type1} est une methode permettant d'appliquer le choc action type 1 de la formule standard Solvabilite 2 a un canton.
\end{Description}
%
\begin{Usage}
\begin{verbatim}
do_choc_action_type1(x, canton)
\end{verbatim}
\end{Usage}
%
\begin{Arguments}
\begin{ldescription}
\item[\code{x}] objet de la classe \code{\LinkA{ChocSolvabilite2}{ChocSolvabilite2}}.

\item[\code{canton}] un objet de la classe \code{\LinkA{Canton}{Canton}}. Il correspond au canton non choque (i.e. central) de l'assureur.
\end{ldescription}
\end{Arguments}
%
\begin{Value}
\code{canton} l'objet  de la classe \code{\LinkA{Canton}{Canton}} correspondant au scenario choque action
au sens de la formule standard Solvabilite 2.
\end{Value}
%
\begin{Note}\relax
Il est possible d'appliquer des chocs actions distincts a chaque action selon l'index.
Cette parametrisation est effectuee dans les fichiers d'inputs utilisateurs.
\end{Note}
%
\begin{Author}\relax
Prim'Act
\end{Author}
\inputencoding{utf8}
\HeaderA{do\_choc\_action\_type2}{Permet a partir d'un canton initial de creer un canton choque action.}{do.Rul.choc.Rul.action.Rul.type2}
\aliasA{ChocSolvabilite2}{do\_choc\_action\_type2}{ChocSolvabilite2}
%
\begin{Description}\relax
\code{do\_choc\_action\_type2} est une  methode permettant d'appliquer le choc action type 2 de la formule
standard Solvabilite 2 a un canton.
\end{Description}
%
\begin{Usage}
\begin{verbatim}
do_choc_action_type2(x, canton)
\end{verbatim}
\end{Usage}
%
\begin{Arguments}
\begin{ldescription}
\item[\code{x}] objet de la classe \code{\LinkA{ChocSolvabilite2}{ChocSolvabilite2}}.

\item[\code{canton}] un objet de la classe \code{\LinkA{Canton}{Canton}}. Il correspond au canton non choque (i.e. central) de l'assureur.
\end{ldescription}
\end{Arguments}
%
\begin{Value}
\code{canton} l'objet  de la classe \code{\LinkA{Canton}{Canton}} correspondant au scenario choque action 
au sens de la formule standard Solvabilite 2.
\end{Value}
%
\begin{Note}\relax
Il est possible d'appliquer des chocs actions distincts a chaque action selon l'index.
Cette parametrisation est effectuee dans les fichiers d'inputs utilisateurs.
\end{Note}
%
\begin{Author}\relax
Prim'Act
\end{Author}
\inputencoding{utf8}
\HeaderA{do\_choc\_frais}{Permet a partir d'un canton initial de creer un canton choque frais.}{do.Rul.choc.Rul.frais}
\aliasA{ChocSolvabilite2}{do\_choc\_frais}{ChocSolvabilite2}
%
\begin{Description}\relax
\code{do\_choc\_frais} est une methode permettant d'appliquer le choc frais de la formule standard Solvabilite 2
a un canton.
\end{Description}
%
\begin{Usage}
\begin{verbatim}
do_choc_frais(x, canton, autres_passifs_choc)
\end{verbatim}
\end{Usage}
%
\begin{Arguments}
\begin{ldescription}
\item[\code{x}] objet de la classe \code{\LinkA{ChocSolvabilite2}{ChocSolvabilite2}}.

\item[\code{canton}] est un objet de la classe \code{\LinkA{Canton}{Canton}}. Il correspond au canton non choque (i.e. central)
de l'assureur.

\item[\code{autres\_passifs\_choc}] est un objet de la classe \code{\LinkA{AutresPassifs}{AutresPassifs}}, il correspond au chargement
des autres passifs choques.
Ces derniers ont ete renseignes par l'utilisateur en donnees.
\end{ldescription}
\end{Arguments}
%
\begin{Value}
\code{canton} l'objet  de la classe \code{\LinkA{Canton}{Canton}} correspondant au scenario choque frais
au sens de la formule standard Solvabilite 2.
\end{Value}
%
\begin{Note}\relax
La parametrisation des chocs de frais est effectuee dans les fichiers d'inputs utilisateurs.
\end{Note}
%
\begin{Author}\relax
Prim'Act
\end{Author}
\inputencoding{utf8}
\HeaderA{do\_choc\_immo}{Permet a partir d'un canton initial de creer un canton choque immobilier.}{do.Rul.choc.Rul.immo}
\aliasA{ChocSolvabilite2}{do\_choc\_immo}{ChocSolvabilite2}
%
\begin{Description}\relax
\code{do\_choc\_immo} est une methode permettant d'appliquer le choc immobilier de la formule standard Solvabilite 2
a un canton.
\end{Description}
%
\begin{Usage}
\begin{verbatim}
do_choc_immo(x, canton)
\end{verbatim}
\end{Usage}
%
\begin{Arguments}
\begin{ldescription}
\item[\code{x}] objet de la classe \code{\LinkA{ChocSolvabilite2}{ChocSolvabilite2}}.

\item[\code{canton}] est un objet de la classe \code{\LinkA{Canton}{Canton}}. Il correspond au canton non choque (i.e. central) 
de l'assureur.
\end{ldescription}
\end{Arguments}
%
\begin{Value}
\code{canton} l'objet  de la classe \code{\LinkA{Canton}{Canton}} correspondant au scenario choque immobilier au sens de la formule standard Solvabilite 2.
\end{Value}
%
\begin{Note}\relax
Il est possible d'appliquer des chocs immobiliers distincts a chaque ligne immobilier present en portefeuille
selon l'index.
Cette parametrisation est effectuee dans les fichiers d'inputs utilisateurs.
\end{Note}
%
\begin{Author}\relax
Prim'Act
\end{Author}
\inputencoding{utf8}
\HeaderA{do\_choc\_longevite}{Permet a partir d'un canton initial de creer un canton choque longevite.}{do.Rul.choc.Rul.longevite}
\aliasA{ChocSolvabilite2}{do\_choc\_longevite}{ChocSolvabilite2}
%
\begin{Description}\relax
\code{do\_choc\_longevite} est une methode permettant d'appliquer le choc longevite de la formule standard
Solvabilite 2 a un canton.
\end{Description}
%
\begin{Usage}
\begin{verbatim}
do_choc_longevite(x, canton, autres_passifs_choc)
\end{verbatim}
\end{Usage}
%
\begin{Arguments}
\begin{ldescription}
\item[\code{x}] objet de la classe \code{\LinkA{ChocSolvabilite2}{ChocSolvabilite2}}.

\item[\code{canton}] est un objet de la classe \code{\LinkA{Canton}{Canton}}. Il correspond au canton non choque (i.e. central) de l'assureur.

\item[\code{autres\_passifs\_choc}] est un objet de la classe \code{\LinkA{AutresPassifs}{AutresPassifs}}, il correspond au chargement des autres passifs choques en longevite.
Ces derniers ont ete renseignes par l'utilisateur en donnees.
\end{ldescription}
\end{Arguments}
%
\begin{Value}
\code{canton} l'objet  de la classe \code{\LinkA{Canton}{Canton}} correspondant au scenario choque longevite
au sens de la formule standard Solvabilite 2.
\end{Value}
%
\begin{Note}\relax
La parametrisation des chocs de longevite est effectuee dans les fichiers d'inputs utilisateurs.
\end{Note}
%
\begin{Author}\relax
Prim'Act
\end{Author}
\inputencoding{utf8}
\HeaderA{do\_choc\_mortalite}{Permet a partir d'un canton initial de creer un canton choque mortalite.}{do.Rul.choc.Rul.mortalite}
\aliasA{ChocSolvabilite2}{do\_choc\_mortalite}{ChocSolvabilite2}
%
\begin{Description}\relax
\code{do\_choc\_mortalite} est une methode permettant d'appliquer le choc mortalite de la formule standard
Solvabilite 2 a un canton.
\end{Description}
%
\begin{Usage}
\begin{verbatim}
do_choc_mortalite(x, canton, autres_passifs_choc)
\end{verbatim}
\end{Usage}
%
\begin{Arguments}
\begin{ldescription}
\item[\code{x}] objet de la classe \code{\LinkA{ChocSolvabilite2}{ChocSolvabilite2}}.

\item[\code{canton}] est un objet de la classe \code{\LinkA{Canton}{Canton}}. Il correspond au canton non choque (i.e. central)
de l'assureur.

\item[\code{autres\_passifs\_choc}] est un objet de la classe \code{\LinkA{AutresPassifs}{AutresPassifs}}, il correspond au chargement des autres passifs choques en mortalite.
Ces derniers ont ete renseignes par l'utilisateur en donnees.
\end{ldescription}
\end{Arguments}
%
\begin{Value}
\code{canton} l'objet  de la classe \code{canton} correspondant au scenario choque mortalite au sens de la formule standard Solvabilite 2.
\end{Value}
%
\begin{Note}\relax
La parametrisation des chocs de mortalite est effectuee dans les fichiers d'inputs utilisateurs.
\end{Note}
%
\begin{Author}\relax
Prim'Act
\end{Author}
\inputencoding{utf8}
\HeaderA{do\_choc\_rachat\_down}{Permet a partir d'un canton initial de creer un canton dont les taux de rachat sont choques a la baisse.}{do.Rul.choc.Rul.rachat.Rul.down}
\aliasA{ChocSolvabilite2}{do\_choc\_rachat\_down}{ChocSolvabilite2}
%
\begin{Description}\relax
\code{do\_choc\_rachat\_down} est une methode permettant d'appliquer le choc a la baisse des taux de rachat de la formule standard Solvabilite 2 a un canton.
\end{Description}
%
\begin{Usage}
\begin{verbatim}
do_choc_rachat_down(x, canton, autres_passifs_choc)
\end{verbatim}
\end{Usage}
%
\begin{Arguments}
\begin{ldescription}
\item[\code{x}] objet de la classe \code{\LinkA{ChocSolvabilite2}{ChocSolvabilite2}}.

\item[\code{canton}] est un objet de la classe \code{\LinkA{Canton}{Canton}}. Il correspond au canton non choque (i.e. central)
de l'assureur.

\item[\code{autres\_passifs\_choc}] est un objet de la classe \code{\LinkA{AutresPassifs}{AutresPassifs}}, il correspond au chargement
des autres passifs choques en rachat a la baisse.
Ces derniers ont ete renseignes par l'utilisateur en donnees.
\end{ldescription}
\end{Arguments}
%
\begin{Value}
\code{canton} l'objet  de la classe \code{\LinkA{Canton}{Canton}} correspondant au scenario de choc a la baisse
des taux de rachats au sens de la formule standard Solvabilite 2.
\end{Value}
%
\begin{Note}\relax
La parametrisation des chocs a la baisse des taux de rachat est effectuee dans les fichiers d'inputs
utilisateurs.
\end{Note}
%
\begin{Author}\relax
Prim'Act
\end{Author}
\inputencoding{utf8}
\HeaderA{do\_choc\_rachat\_up}{Permet a partir d'un canton initial de creer un canton dont les taux de rachat sont choques a la hausse.}{do.Rul.choc.Rul.rachat.Rul.up}
\aliasA{ChocSolvabilite2}{do\_choc\_rachat\_up}{ChocSolvabilite2}
%
\begin{Description}\relax
\code{do\_choc\_rachat\_up} est une methode permettant d'appliquer le choc a la hausse des taux de rachat
de la formule standard Solvabilite 2 a un canton.
\end{Description}
%
\begin{Usage}
\begin{verbatim}
do_choc_rachat_up(x, canton, autres_passifs_choc)
\end{verbatim}
\end{Usage}
%
\begin{Arguments}
\begin{ldescription}
\item[\code{x}] objet de la classe \code{\LinkA{ChocSolvabilite2}{ChocSolvabilite2}}.

\item[\code{canton}] est un objet de la classe \code{\LinkA{Canton}{Canton}}. Il correspond au canton non choque (i.e. central)
de l'assureur.

\item[\code{autres\_passifs\_choc}] est un objet de la classe \code{\LinkA{AutresPassifs}{AutresPassifs}}, il correspond au chargement
des autres passifs choques en rachat a la hausse.
Ces derniers ont ete renseignes par l'utilisateur en donnees.
\end{ldescription}
\end{Arguments}
%
\begin{Value}
\code{canton} l'objet  de la classe \code{\LinkA{Canton}{Canton}} correspondant au scenario de choc a la hausse
des taux de rachats au sens de la formule standard Solvabilite 2.
\end{Value}
%
\begin{Note}\relax
La parametrisation des chocs a la hausse des taux de rachat est effectuee dans les fichiers d'inputs
utilisateurs.
\end{Note}
%
\begin{Author}\relax
Prim'Act
\end{Author}
\inputencoding{utf8}
\HeaderA{do\_choc\_spread}{Permet a partir d'un canton initial de creer un canton choque spread.}{do.Rul.choc.Rul.spread}
\aliasA{ChocSolvabilite2}{do\_choc\_spread}{ChocSolvabilite2}
%
\begin{Description}\relax
\code{do\_choc\_spread} est une methode permettant d'appliquer le choc spread de la formule standard Solvabilite 2
a un canton. Cette methode s'applique uniquement aux obligations de type \code{corp}.
\end{Description}
%
\begin{Usage}
\begin{verbatim}
do_choc_spread(x, canton)
\end{verbatim}
\end{Usage}
%
\begin{Arguments}
\begin{ldescription}
\item[\code{x}] objet de la classe \code{\LinkA{ChocSolvabilite2}{ChocSolvabilite2}}.

\item[\code{canton}] est un objet de la classe \code{\LinkA{Canton}{Canton}}. Il correspond au canton non choque (i.e. central)
de l'assureur.
\end{ldescription}
\end{Arguments}
%
\begin{Value}
\code{canton} l'objet  de la classe \code{\LinkA{Canton}{Canton}} correspondant au scenario choque
spread au sens de la formule standard Solvabilite 2.
\end{Value}
%
\begin{Note}\relax
Il est possible d'appliquer des chocs de spreads distincts a chaque ligne du portefeuille obligataire
selon le numero de rating et la duration de l'obligation.
Cette parametrisation est effectuee dans les fichiers d'inputs utilisateurs.
\end{Note}
%
\begin{Author}\relax
Prim'Act
\end{Author}
%
\begin{SeeAlso}\relax
L'application du choc de spread a une ligne obligataire : \code{\LinkA{do\_choc\_spread\_unitaire}{do.Rul.choc.Rul.spread.Rul.unitaire}}.
\end{SeeAlso}
\inputencoding{utf8}
\HeaderA{do\_choc\_spread\_unitaire}{Applique le choc spread de la formule standard Solvabilite 2 a une ligne obligataire.}{do.Rul.choc.Rul.spread.Rul.unitaire}
\aliasA{ChocSolvabilite2}{do\_choc\_spread\_unitaire}{ChocSolvabilite2}
%
\begin{Description}\relax
\code{do\_choc\_spread\_unitaire} Permet a partir d'une table contenant les elements du choc de spread obligataire
Solvabilite 2 et d'une ligne obligataire d'un element \code{\LinkA{Oblig}{Oblig}} d'un portefeuille financier
\code{\LinkA{PortFin}{PortFin}}
d'appliquer le choc de spread a cette ligne obligataire.
\end{Description}
%
\begin{Usage}
\begin{verbatim}
do_choc_spread_unitaire(table_choc_spread, ligne_oblig)
\end{verbatim}
\end{Usage}
%
\begin{Arguments}
\begin{ldescription}
\item[\code{table\_choc\_spread}] un \code{data.frame} contenant la table de parametres avec les chocs de spreads.

\item[\code{ligne\_oblig}] un \code{data.frame}. Il correspond a une ligne obligataire d'un portefeuille \code{\LinkA{Oblig}{Oblig}}
d'un assureur.
\end{ldescription}
\end{Arguments}
%
\begin{Value}
\code{vm\_choquee} une valeur \code{numeric} correspondant a la valeur de marche de la ligne obligataire
suite a l'application du choc de spread a cette ligne.
\end{Value}
%
\begin{Author}\relax
Prim'Act
\end{Author}
%
\begin{SeeAlso}\relax
La classe \code{\LinkA{PortFin}{PortFin}}.
\end{SeeAlso}
\inputencoding{utf8}
\HeaderA{do\_choc\_taux}{Methode permettant d'appliquer le choc de taux a un Canton.}{do.Rul.choc.Rul.taux}
\aliasA{ChocSolvabilite2}{do\_choc\_taux}{ChocSolvabilite2}
%
\begin{Description}\relax
\code{do\_choc\_taux} est une methode permettant d'appliquer le choc de taux de la formule standard Solvabilite 2 a un canton.
\end{Description}
%
\begin{Usage}
\begin{verbatim}
do_choc_taux(canton)
\end{verbatim}
\end{Usage}
%
\begin{Arguments}
\begin{ldescription}
\item[\code{canton}] un objet de la classe \code{Canton}, correspondant au canton auquel on souhaite appliquer le choc de taux.
\end{ldescription}
\end{Arguments}
%
\begin{Value}
canton l'objet de la classe \code{Canton}, mis a jour du choc de taux.
\end{Value}
%
\begin{Author}\relax
Prim'Act
\end{Author}
\inputencoding{utf8}
\HeaderA{do\_update\_pmvl}{Met a jour l'ensemble des attributs pvl et pml d'un objet PortFin}{do.Rul.update.Rul.pmvl}
\aliasA{PortFin}{do\_update\_pmvl}{PortFin}
%
\begin{Description}\relax
\code{do\_update\_pmvl} est une methode permettant de calculer le taux de rendement financier du portefeuille.
\end{Description}
%
\begin{Usage}
\begin{verbatim}
do_update_pmvl(x)
\end{verbatim}
\end{Usage}
%
\begin{Arguments}
\begin{ldescription}
\item[\code{x}] est un objet de la classe \code{PortFin},
\end{ldescription}
\end{Arguments}
%
\begin{Value}
L'objet \code{x} de la classe \code{PortFin} dont les plus values et moins values ont ete recalculees avec les elements du \code{PortFin} renseigne en input.
\end{Value}
%
\begin{Author}\relax
Prim'Act
\end{Author}
\inputencoding{utf8}
\HeaderA{do\_update\_PRE\_val\_courante}{Mise a jour de la valeur courante de PRE.}{do.Rul.update.Rul.PRE.Rul.val.Rul.courante}
\aliasA{PRE}{do\_update\_PRE\_val\_courante}{PRE}
%
\begin{Description}\relax
\code{do\_update\_PRE\_val\_courante} est une methode permettant de calculer le montant de PRE.
\end{Description}
%
\begin{Usage}
\begin{verbatim}
do_update_PRE_val_courante(x, val_courante)
\end{verbatim}
\end{Usage}
%
\begin{Arguments}
\begin{ldescription}
\item[\code{x}] objet de la classe \code{PRE}correspondant a la PRE avant mise a jour.

\item[\code{val\_courante}] est un \code{numeric} correspondant au montant de PRE calcule par la fonction \code{\LinkA{calc\_PRE}{calc.Rul.PRE}}.
\end{ldescription}
\end{Arguments}
%
\begin{Value}
L'objet \code{PRE} mis a jour de la nouvelle valeur courante de \code{PRE}
\end{Value}
%
\begin{Author}\relax
Prim'Act
\end{Author}
%
\begin{SeeAlso}\relax
La methode de calcul de la PRE \code{\LinkA{calc\_PRE}{calc.Rul.PRE}}
\end{SeeAlso}
\inputencoding{utf8}
\HeaderA{do\_update\_PRE\_val\_debut}{Mise a jour de la valeur de debut de periode de la PRE}{do.Rul.update.Rul.PRE.Rul.val.Rul.debut}
\aliasA{PRE}{do\_update\_PRE\_val\_debut}{PRE}
%
\begin{Description}\relax
\code{do\_update\_PRE\_val\_debut} est une methode permettant de mettre a jour le montant de debut de periode de PRE.
\end{Description}
%
\begin{Usage}
\begin{verbatim}
do_update_PRE_val_debut(x, val_debut)
\end{verbatim}
\end{Usage}
%
\begin{Arguments}
\begin{ldescription}
\item[\code{x}] objet de la classe \code{PRE} correspondant a la PRE avant mise a jour.

\item[\code{val\_debut}] est un \code{numeric} correspondant au montant de debut de periode de PRE.
\end{ldescription}
\end{Arguments}
%
\begin{Value}
L'objet \code{PRE} mis a jour de la nouvelle valeur de debut de \code{PRE}
\end{Value}
%
\begin{Author}\relax
Prim'Act
\end{Author}
%
\begin{SeeAlso}\relax
La methode de calcul de la PRE \code{\LinkA{calc\_PRE}{calc.Rul.PRE}}
\end{SeeAlso}
\inputencoding{utf8}
\HeaderA{do\_update\_RC\_val\_courante}{Mise a jour de la valeur courante de RC}{do.Rul.update.Rul.RC.Rul.val.Rul.courante}
\aliasA{RC}{do\_update\_RC\_val\_courante}{RC}
%
\begin{Description}\relax
\code{do\_update\_RC\_val\_courante} est une methode permettant de calculer le montant de RC.
\end{Description}
%
\begin{Usage}
\begin{verbatim}
do_update_RC_val_courante(x, val_courante)
\end{verbatim}
\end{Usage}
%
\begin{Arguments}
\begin{ldescription}
\item[\code{x}] objet de la classe \code{RC}correspondant a la RC avant mise a jour.

\item[\code{val\_courante}] est un \code{numeric} correspondant au montant de RC calcule par la fonction \code{\LinkA{calc\_RC}{calc.Rul.RC}}.
\end{ldescription}
\end{Arguments}
%
\begin{Value}
L'objet \code{RC} mis a jour de la nouvelle valeur courante de \code{RC}
\end{Value}
%
\begin{Author}\relax
Prim'Act
\end{Author}
%
\begin{SeeAlso}\relax
La methode de calcul de la RC \code{\LinkA{calc\_RC}{calc.Rul.RC}}
\end{SeeAlso}
\inputencoding{utf8}
\HeaderA{do\_update\_RC\_val\_debut}{Mise a jour de la valeur initiale de RC}{do.Rul.update.Rul.RC.Rul.val.Rul.debut}
\aliasA{RC}{do\_update\_RC\_val\_debut}{RC}
%
\begin{Description}\relax
\code{do\_update\_RC\_val\_debut} est une methode permettant de mettre a jour le montant de debut de periode de RC.
\end{Description}
%
\begin{Usage}
\begin{verbatim}
do_update_RC_val_debut(x, val_debut)
\end{verbatim}
\end{Usage}
%
\begin{Arguments}
\begin{ldescription}
\item[\code{x}] objet de la classe \code{RC} correspondant a la RC avant mise a jour.

\item[\code{val\_debut}] est un \code{numeric} correspondant au montant de debut de periode de RC.
\end{ldescription}
\end{Arguments}
%
\begin{Value}
L'objet \code{RC} mis a jour de la nouvelle valeur de debut de \code{RC}
\end{Value}
%
\begin{Author}\relax
Prim'Act
\end{Author}
%
\begin{SeeAlso}\relax
La methode de calcul de la RC \code{\LinkA{calc\_RC}{calc.Rul.RC}}
\end{SeeAlso}
\inputencoding{utf8}
\HeaderA{do\_update\_vm\_vnc\_precedent}{Evalue et met a jour les objets constituants un PortFin.}{do.Rul.update.Rul.vm.Rul.vnc.Rul.precedent}
\aliasA{PortFin}{do\_update\_vm\_vnc\_precedent}{PortFin}
%
\begin{Description}\relax
\code{do\_update\_vm\_vnc\_precedent} est une methode permettant de calculer et mettre a jour un portefeuille financier
suite a un vieillissement.
\end{Description}
%
\begin{Usage}
\begin{verbatim}
do_update_vm_vnc_precedent(x)
\end{verbatim}
\end{Usage}
%
\begin{Arguments}
\begin{ldescription}
\item[\code{x}] objet de la classe \code{PortFin}, correspondant au portefeuille financier de l'assureur avant mise a jour de l'attribut vm\_vnc\_precedent.
\end{ldescription}
\end{Arguments}
%
\begin{Value}
L'objet de la classe \code{PortFin} renvoye correspond au portefeuille financier de l'assureur dont l'attribut vm\_vnc\_precedent a ete mis a jour.
\end{Value}
%
\begin{Author}\relax
Prim'Act
\end{Author}
\inputencoding{utf8}
\HeaderA{duration\_sensi}{Calcule la duration de chaque composante d'un portefeuille obligataire.}{duration.Rul.sensi}
\aliasA{Oblig}{duration\_sensi}{Oblig}
%
\begin{Description}\relax
\code{duration\_sensi} est une methode permettant de calculer la duration de chaque composante d'un portefeuille obligataire.
\end{Description}
%
\begin{Usage}
\begin{verbatim}
duration_sensi(x)
\end{verbatim}
\end{Usage}
%
\begin{Arguments}
\begin{ldescription}
\item[\code{x}] objet de la classe Oblig (decrivant le portefeuille obligataire).
\end{ldescription}
\end{Arguments}
%
\begin{Value}
Un data frame compose de deux colonnes : la premiere est composee de la duration de chacune des obligations du portefeuille obligataire.
La seconde est compose de la sensibilite de chacune des obligations du portefeuille obligataire.
Le dataframe de sortie a autant d'elements que le portefeuille obligataire a de lignes.
\end{Value}
%
\begin{Author}\relax
Prim'Act
\end{Author}
\inputencoding{utf8}
\HeaderA{echeancier}{Calcule les flux d'un model point ou d'un ensemble de models points obligataires.}{echeancier}
\aliasA{Oblig}{echeancier}{Oblig}
%
\begin{Description}\relax
\code{echeancier} est une methode permettant de calculer les flux jusqu'a maturite residuelle.
\end{Description}
%
\begin{Usage}
\begin{verbatim}
echeancier(coupon, maturite, zspread, nominal, yield)
\end{verbatim}
\end{Usage}
%
\begin{Arguments}
\begin{ldescription}
\item[\code{coupon}] vecteur contenant les taux de coupons

\item[\code{maturite}] vecteur d'entiers contenant les maturites residuelles

\item[\code{zspread}] vecteur contenant les zero-spreads

\item[\code{nominal}] vecteur contenant les valeurs nominales de chaque obligation

\item[\code{yield}] vecteur contenant la courbe de taux consideree (peut-etre vide)
\end{ldescription}
\end{Arguments}
%
\begin{Value}
Une matrice contenant :
\begin{description}

\item[\code{grid\_flux} : ] la matrice d'ecoulement des flux. Cette matrice a autant de colonnes
que le max du vecteur de maturite residuelle, et autant de lignes que les vecteurs d'input \code{coupon,maturite,zspread,nominal}.
Chaque ligne decrit les flux annuels a venir pour l'actif obligataire de caracteristique renseigne en input.

\end{description}

\end{Value}
%
\begin{Author}\relax
Prim'Act
\end{Author}
\inputencoding{utf8}
\HeaderA{EpEuroInd}{La classe \code{EpEuroInd}.}{EpEuroInd}
\keyword{classes}{EpEuroInd}
%
\begin{Description}\relax
Une classe pour les passifs epargne en euros.
\end{Description}
%
\begin{Section}{Slots}

\begin{description}

\item[\code{mp}] un objet \code{data.frame} au format fige contenant l'ensemble de model points epargne en euros.

\item[\code{tab}] un objet de la classe \code{\LinkA{TabEpEuroInd}{TabEpEuroInd}} dedie au stockage de variables intermediaires.

\end{description}
\end{Section}
%
\begin{Author}\relax
Prim'Act
\end{Author}
%
\begin{SeeAlso}\relax
Le calcul des primes, des prestations et des PM : \code{\LinkA{calc\_primes}{calc.Rul.primes}},
\code{\LinkA{calc\_prest}{calc.Rul.prest}}, \code{\LinkA{calc\_pm}{calc.Rul.pm}}.
Le calcul des taux de sortie, du taux minimum et des taux cible de revalorisation :
\code{\LinkA{calc\_tx\_sortie}{calc.Rul.tx.Rul.sortie}}, \code{\LinkA{calc\_tx\_min}{calc.Rul.tx.Rul.min}}, \code{\LinkA{calc\_tx\_cible}{calc.Rul.tx.Rul.cible}}.
La revalorisation des PM apres participation aux benefices : \code{\LinkA{calc\_relavo\_pm}{calc.Rul.relavo.Rul.pm}}.
Le vieillissement des model points sur une periode : \code{\LinkA{vieilli\_mp}{vieilli.Rul.mp}}.
\end{SeeAlso}
\inputencoding{utf8}
\HeaderA{ESG}{La classe ESG}{ESG}
\keyword{classes}{ESG}
%
\begin{Description}\relax
Une classe de parametres contenant les tables de simulation, generees par une generateur de scenarions
economique.
\end{Description}
%
\begin{Section}{Slots}

\begin{description}

\item[\code{nb\_simu}] un entier (\code{integer}) correspondant au nombre de simulations.

\item[\code{ind\_action}] une liste contenant les differents indices actions utilises. Chaque element de la liste contient
\code{nb\_simu} simulations de l'indice.

\item[\code{ind\_immo}] une liste contenant les differents indices immobilier utilises. Chaque element de la liste contient
\code{nb\_simu} simulations de l'indice.

\item[\code{ind\_inflation}] une liste contenant l'indice inflation utilise. L'element de la liste contient
\code{nb\_simu} simulations de l'indice.

\item[\code{yield\_curve}] une liste contenant les courbes de taux simulees a chaque date de projection. Chaque element
de la liste, correspondant a une annee de projection, contient \code{nb\_simu} simulations de la courbe des taux.

\item[\code{deflateur}] une liste contenant le deflateur stochastique a utiliser. L'element de la liste contient
\code{nb\_simu} simulations du deflateur.

\end{description}
\end{Section}
%
\begin{Author}\relax
Prim'Act
\end{Author}
%
\begin{SeeAlso}\relax
Les methodes de chargement d'un ESG \code{\LinkA{chargement\_ESG}{chargement.Rul.ESG}} et d'extraction d'un model point
ESG \code{\LinkA{extract\_ESG}{extract.Rul.ESG}}.
\end{SeeAlso}
\inputencoding{utf8}
\HeaderA{extract\_ESG}{permet de construire et charger les trajectoires simulees par le Generateur de Scenarios Economiques de Prim'Act.}{extract.Rul.ESG}
\aliasA{ESG}{extract\_ESG}{ESG}
%
\begin{Description}\relax
\code{extract\_ESG} construit l'objet de classe \code{\LinkA{ModelPoint\_ESG}{ModelPoint.Rul.ESG}} a partir d'un objet
de la classe \code{\LinkA{ESG}{ESG}}.
Le \code{ModelPointESG} ainsi construit correspond a l'extraction de donnees de l'ESG
pour une annee specifique et pour une simulation specifique.
\end{Description}
%
\begin{Usage}
\begin{verbatim}
extract_ESG(x, num_trajectoire, annee)
\end{verbatim}
\end{Usage}
%
\begin{Arguments}
\begin{ldescription}
\item[\code{x}] un objet de la classe \code{\LinkA{ESG}{ESG}}.

\item[\code{num\_trajectoire}] une valeur de type \code{integer} correspondant a la trajectoire de simulation
dont on souhaite obtenir les valeurs.

\item[\code{annee}] une valeur de type \code{integer} correspondant a l'annee d'interet pour le model point
(possibilite de selectionner les annees 0 a \code{nb\_annee\_proj}).
\end{ldescription}
\end{Arguments}
%
\begin{Value}
\code{x} l'objet de la classe \code{\LinkA{ModelPoint\_ESG}{ModelPoint.Rul.ESG}} construit.
\end{Value}
%
\begin{Author}\relax
Prim'Act
\end{Author}
%
\begin{SeeAlso}\relax
La classe \code{\LinkA{ModelPoint\_ESG}{ModelPoint.Rul.ESG}}.
\end{SeeAlso}
\inputencoding{utf8}
\HeaderA{finance\_cible\_marge}{Evalue le financement d'une revalorisation au taux cible par la marge de l'assureur}{finance.Rul.cible.Rul.marge}
\aliasA{RevaloEngine}{finance\_cible\_marge}{RevaloEngine}
%
\begin{Description}\relax
\code{finance\_cible\_marge} est une methode permettant de
determiner le financement d'une revalorisation au taux cible en comprimant la marge financiere
de l'assureur
\end{Description}
%
\begin{Usage}
\begin{verbatim}
finance_cible_marge(marge_fin, bes_cible, rev_stock_nette, marge_min)
\end{verbatim}
\end{Usage}
%
\begin{Arguments}
\begin{ldescription}
\item[\code{marge\_fin}] une valeur \code{numeric} donnant le montant courant de la marge financiere de l'assureur.

\item[\code{bes\_cible}] un vecteur \code{numeric} correspondant au besoin de financement necessaire pour atteindre
le taux cible part produit.

\item[\code{rev\_stock\_nette}] un vecteur \code{numeric} comprenant par produit
le montant de revalorisation nette au titre de le PB atteint.

\item[\code{marge\_min}] est une valeur \code{numeric} correspondant
au montant minimum de marge financiere souhaite par l'assureur.
\end{ldescription}
\end{Arguments}
%
\begin{Value}
\code{rev\_stock\_nette} la valeur de la revalorisation nette servie par produit apres reduction de marge.

\code{marge\_fin} le montant de marge de l'assureur apres reduction.
\end{Value}
%
\begin{Author}\relax
Prim'Act
\end{Author}
\inputencoding{utf8}
\HeaderA{finance\_cible\_pmvl}{Evalue le financement d'une revalorisation au taux cible par des cessions de plus-values latentes.}{finance.Rul.cible.Rul.pmvl}
\aliasA{RevaloEngine}{finance\_cible\_pmvl}{RevaloEngine}
%
\begin{Description}\relax
\code{finance\_cible\_pmvl} est une methode permettant de
determiner le financement d'une revalorisation au taux cible par une cession de plus-values latentes en actions.
\end{Description}
%
\begin{Usage}
\begin{verbatim}
finance_cible_pmvl(bes_cible, rev_stock_nette, base_fin, seuil_pmvl, tx_pb)
\end{verbatim}
\end{Usage}
%
\begin{Arguments}
\begin{ldescription}
\item[\code{bes\_cible}] un vecteur \code{numeric} correspondant au besoin de financement necessaire pour atteindre
le taux cible part produit.

\item[\code{rev\_stock\_nette}] un vecteur \code{numeric} comprenant par produit
le montant de revalorisation nette au titre de le PB atteint.

\item[\code{base\_fin}] un vecteur \code{numeric} comprenant par produit la base de produits financiers.

\item[\code{seuil\_pmvl}] une valeur \code{numeric} correspondant au montant de plus-values latentes
qui peut etre liquidee. Ce montant doit etre exprime en tenant compte de l'abattement (mise a l'echelle) realise
pour rapport aux plus-values latentes de l'actif general au passif.

\item[\code{tx\_pb}] un vecteur \code{numeric} comprenant par produit les taux de participation aux benefices contractuels.
\end{ldescription}
\end{Arguments}
%
\begin{Details}\relax
Lorsque la revalorisation nette est superieure au besoin de financement des taux cibles, on sert le taux cible
et on partage le surplus. A l'inverse, les taux cible sont finances par
les compensations entre produits lorsque certains prevoient une revalorisation superieure au taux cible, et
par une liquidation de plus-values latentes.
\end{Details}
%
\begin{Value}
\code{rev\_stock\_nette} la valeur de la revalorisation nette servie par produit apres cession.

\code{pmvl\_liq} le montant de plus-values a liquider, ramene a la valeur du passif, pour financer
la revalorisation.
\end{Value}
%
\begin{Author}\relax
Prim'Act
\end{Author}
\inputencoding{utf8}
\HeaderA{finance\_cible\_ppb}{Evalue le financement d'une revalorisation au taux cible par une reprise de PPB.}{finance.Rul.cible.Rul.ppb}
\aliasA{RevaloEngine}{finance\_cible\_ppb}{RevaloEngine}
%
\begin{Description}\relax
\code{finance\_cible\_ppb} est une methode permettant de
determiner le financement d'une revalorisation au taux cible par la reprise de provision pour participation
aux benefices (PPB). Cette methode evalue egalement si une dotation est effectue.
\end{Description}
%
\begin{Usage}
\begin{verbatim}
finance_cible_ppb(bes_cible, rev_stock_nette, ppb)
\end{verbatim}
\end{Usage}
%
\begin{Arguments}
\begin{ldescription}
\item[\code{bes\_cible}] un vecteur \code{numeric} correspondant au besoin de financement necessaire pour atteindre
le taux cible par produit.

\item[\code{rev\_stock\_nette}] un vecteur \code{numeric} comprenant par produit
le montant de revalorisation nette au titre de le PB atteint.

\item[\code{ppb}] un objet de la classe \code{\LinkA{Ppb}{Ppb}} qui renvoie l'etat courant de la PPB.
\end{ldescription}
\end{Arguments}
%
\begin{Details}\relax
Lorsque la revalorisation nette est superieure au besoin de financement des taux cibles,
on sert le taux cible et on dote le reste a la PPB dans la limite du plafond de dotation annuel.
A l'inverse, les taux cible sont finances par les compensations entre produits lorsque certains
prevoient une revalorisation superieure au taux cible, puis
par une reprise sur PPB.
\end{Details}
%
\begin{Value}
\code{rev\_stock\_nette} la valeur de la revalorisation nette servie apres une eventuelle reprise de PPB.

\code{dotation} le montant de dotation a la PPB.

\code{reprise} le montant de reprise sur la PPB.

\code{ppb} l'objet \code{ppb} mis a jour.
\end{Value}
%
\begin{Author}\relax
Prim'Act
\end{Author}
\inputencoding{utf8}
\HeaderA{finance\_contrainte\_legale}{Applique la contrainte legale de participation aux benefices.}{finance.Rul.contrainte.Rul.legale}
\aliasA{RevaloEngine}{finance\_contrainte\_legale}{RevaloEngine}
%
\begin{Description}\relax
\code{finance\_contrainte\_legale} est une methode permettant de calculer la contrainte legale
de participation aux benefices et de l'appliquer si necessaire pour accroitre la revalorisation.
\end{Description}
%
\begin{Usage}
\begin{verbatim}
finance_contrainte_legale(base_fin, base_fin_etendu, result_tech, it_stock,
  rev_stock_nette, rev_prest_nette, dot_ppb, marge_fin, ppb, param_revalo)
\end{verbatim}
\end{Usage}
%
\begin{Arguments}
\begin{ldescription}
\item[\code{base\_fin}] un vecteur \code{numeric} comprenant par produit la base de produits financiers.

\item[\code{result\_tech}] une valeur \code{numeric} comprenant le resultat technique.

\item[\code{it\_stock}] un vecteur \code{numeric} comprenant par produit les interets techniques affectes au stock.

\item[\code{rev\_stock\_nette}] un vecteur de type \code{numeric} comprenant par produit la revalorisaton nette
affectee au stock.

\item[\code{rev\_prest\_nette}] un vecteur de type \code{numeric} comprenant par produit a revalorisaton nette
affectee aux prestations.

\item[\code{dot\_ppb}] une valeur \code{numeric} comprenant la dotation de PPB financant la revalorisation
sur stock.

\item[\code{marge\_fin}] une valeur \code{numeric} comprenant la marge financiere courante de l'assureur.

\item[\code{ppb}] un objet de la classe \code{\LinkA{Ppb}{Ppb}} qui renvoie l'etat courant de la PPB.

\item[\code{param\_revalo}] un objet de la classe \code{\LinkA{ParamRevaloEngine}{ParamRevaloEngine}}.
comprenant les parametres de revalorisation.

\item[\code{base\_fin\_entendu}] une valeur \code{numeric} comprenant la base totale de produits financiers
(somme des produits modelise et des  passifs non modelises).
\end{ldescription}
\end{Arguments}
%
\begin{Details}\relax
Cette methode permet de calculer la contrainte de revalorisation imposee par la reglementation. Si cette
contrainte est verifie alors rien n'est fait, hormis la mise a jour eventuelle du solde negatif de PB. Sinon, la
revalorisation additionnelle est dote a la PPB, jusqu'au maximum de dotation possible, puis le relicat est alloue
entre les produits. La revalorisation additionelle vient diminuer la marge financiere de l'assureur.
\end{Details}
%
\begin{Value}
\code{rev\_stock\_nette} la valeur de la revalorisation nette servie apres application de la contrainte legale.

\code{marge\_fin} le montant de marge de l'assureur apres reduction.

\code{ppb} l'objet \code{ppb} mis a jour.

\code{param\_revalo} l'objet \code{param\_revalo} mis a jour (solde de PB reglementaire negatif).
\end{Value}
%
\begin{Author}\relax
Prim'Act
\end{Author}
\inputencoding{utf8}
\HeaderA{finance\_tmg}{Calcule la contribution de la PPB au financement des taux minimums garantis.}{finance.Rul.tmg}
\aliasA{RevaloEngine}{finance\_tmg}{RevaloEngine}
%
\begin{Description}\relax
\code{finance\_tmg} est une methode permettant d'evaluer le contribution de la PPB
au financement des taux minimums garantis (TMG) sur prestations et sur stock.
\end{Description}
%
\begin{Usage}
\begin{verbatim}
finance_tmg(bes_tmg_prest, bes_tmg_stock, ppb)
\end{verbatim}
\end{Usage}
%
\begin{Arguments}
\begin{ldescription}
\item[\code{bes\_tmg\_prest}] un vecteur \code{numeric} comprenant
par produit le besoin de finance des TMG sur prestations.
@param bes\_tmg\_stock un vecteur \code{numeric} comprenant
par produit le besoin de finance des TMG sur le stock.

\item[\code{ppb}] est un objet de la classe \code{\LinkA{Ppb}{Ppb}} qui renvoie l'etat courant de la PPB.
\end{ldescription}
\end{Arguments}
%
\begin{Details}\relax
Dans cette methode, il est considere que le PPB peut venir financer les TMG sur prestations et sur stock.
Par convention, la PPB finance d'abord les TMG sur prestations, puis sur stock.
\end{Details}
%
\begin{Value}
\code{contrib\_tmg\_prest} la valeur de la contribution au financement des TMG sur prestations.

\code{contrib\_tmg\_stock} la valeur de la contribution au financement des TMG sur stock.

\code{ppb} l'objet \code{ppb} mis a jour.
\end{Value}
%
\begin{Author}\relax
Prim'Act
\end{Author}
\inputencoding{utf8}
\HeaderA{FraisFin}{La classe FraisFin}{FraisFin}
\keyword{classes}{FraisFin}
%
\begin{Description}\relax
Classe pour les parametres de frais financiers d'un assureur.
\end{Description}
%
\begin{Section}{Slots}

\begin{description}

\item[\code{tx\_chargement}] est une valeur \code{numeric} correspondant au taux de frais de gestion financiere.

\item[\code{indicatrice\_inflation}] est un objet de type \code{logical}, qui permet d'indiquer si une inflation
doit etre appliquee.

\end{description}
\end{Section}
%
\begin{Author}\relax
Prim'Act
\end{Author}
%
\begin{SeeAlso}\relax
Mettre le lien vers les methodes de la classe
\end{SeeAlso}
\inputencoding{utf8}
\HeaderA{FraisPassif}{La classe \code{FraisPassif}}{FraisPassif}
\keyword{classes}{FraisPassif}
%
\begin{Description}\relax
Une classe de parametres pour les frais des produits du portefeuille de passif.
\end{Description}
%
\begin{Section}{Slots}

\begin{description}

\item[\code{mp}] un objet \code{data.frame} contenant les parametres de frais au passif par produit.

\end{description}
\end{Section}
%
\begin{Author}\relax
Prim'Act
\end{Author}
%
\begin{SeeAlso}\relax
Le calcul des frais de passif \code{\LinkA{calc\_frais}{calc.Rul.frais}}.
\end{SeeAlso}
\inputencoding{utf8}
\HeaderA{frais\_fin\_load}{Methode permettant de charger la valeur initiale des frais financiers dans un objet de type FraisFin.}{frais.Rul.fin.Rul.load}
\aliasA{FraisFin}{frais\_fin\_load}{FraisFin}
%
\begin{Description}\relax
\code{frais\_fin\_load} est une methode permettant de charger les frais financiers.
\end{Description}
%
\begin{Usage}
\begin{verbatim}
frais_fin_load(file_frais_fin_address)
\end{verbatim}
\end{Usage}
%
\begin{Arguments}
\begin{ldescription}
\item[\code{file\_frais\_fin\_address}] est un \code{character} contenant l'adresse exacte du fichier d'input utilisateur permettant de renseigner les Frais financier.
\end{ldescription}
\end{Arguments}
%
\begin{Value}
L'objet de la classe \code{FraisFin} construit a partir des inputs renseignes par l'utilisateur.
\end{Value}
%
\begin{Author}\relax
Prim'Act
\end{Author}
%
\begin{SeeAlso}\relax
La classe \code{\LinkA{Initialisation}{Initialisation}} et sa methode \code{\LinkA{set\_architecture}{set.Rul.architecture}} pour renseigner l'input.
\end{SeeAlso}
\inputencoding{utf8}
\HeaderA{frais\_passif\_load}{Methode permettant de charger la valeur des frais de passif.}{frais.Rul.passif.Rul.load}
\aliasA{FraisPassif}{frais\_passif\_load}{FraisPassif}
%
\begin{Description}\relax
\code{frais\_passif\_load} est une methode permettant de charger les donnees associees a un
objet de classe \code{\LinkA{FraisPassif}{FraisPassif}}.
\end{Description}
%
\begin{Usage}
\begin{verbatim}
frais_passif_load(file_frais_passif_address)
\end{verbatim}
\end{Usage}
%
\begin{Arguments}
\begin{ldescription}
\item[\code{file\_frais\_passif\_address}] est un \code{character} contenant l'adresse exacte du fichier d'input utilisateur
permettant de renseigner un objet \code{\LinkA{FraisPassif}{FraisPassif}}.
\end{ldescription}
\end{Arguments}
%
\begin{Value}
L'objet de la classe \code{\LinkA{FraisPassif}{FraisPassif}} construit a partir des inputs renseignes par l'utilisateur.
\end{Value}
%
\begin{Author}\relax
Prim'Act
\end{Author}
%
\begin{SeeAlso}\relax
La classe \code{\LinkA{Initialisation}{Initialisation}} et sa methode \code{\LinkA{set\_architecture}{set.Rul.architecture}}
pour renseigner l’input.
\end{SeeAlso}
\inputencoding{utf8}
\HeaderA{get\_choc\_inflation\_frais}{Applique le choc frais de la formule standard a la table de simulation pour l'indice inflation.}{get.Rul.choc.Rul.inflation.Rul.frais}
\aliasA{ESG}{get\_choc\_inflation\_frais}{ESG}
%
\begin{Description}\relax
\code{get\_choc\_inflation\_frais} est une methode permettant d'appliquer le choc frais
de la formule standard a la table de simulation pour l'indice inflation.
\end{Description}
%
\begin{Usage}
\begin{verbatim}
get_choc_inflation_frais(x, choc)
\end{verbatim}
\end{Usage}
%
\begin{Arguments}
\begin{ldescription}
\item[\code{x}] un objet de la classe \code{\LinkA{ESG}{ESG}}.

\item[\code{choc}] une valeur \code{numeric} correspondant au coefficient de
choc a appliquer en additif au taux d'inflation.
\end{ldescription}
\end{Arguments}
%
\begin{Value}
L'objet \code{x} mis a jour.
\end{Value}
%
\begin{Note}\relax
L'inflation comprise dans l'ESG est suppose etre deja capitalise, i.e.
\eqn{indice_inflation = (1 + tx inflation)^{annee}}{}. Il ne s'agit pas du taux d'inflation.
\end{Note}
%
\begin{Author}\relax
Prim'Act
\end{Author}
\inputencoding{utf8}
\HeaderA{get\_choc\_rach}{Applique les chocs de rachat de la formule standard.}{get.Rul.choc.Rul.rach}
\aliasA{HypTech}{get\_choc\_rach}{HypTech}
%
\begin{Description}\relax
\code{get\_choc\_rach} est une methode permettant d'appliquer a l'ensemble des lois de rachat
structurelle d'un objet \code{\LinkA{HypTech}{HypTech}} les chocs a la hausse ou a la baisse de la formule standard.
\end{Description}
%
\begin{Usage}
\begin{verbatim}
get_choc_rach(x, type_choc_rach, choc, choc_lim)
\end{verbatim}
\end{Usage}
%
\begin{Arguments}
\begin{ldescription}
\item[\code{type\_choc\_rach}] est un character renseignant le type de choc a applique : \code{up}
pour le choc a la hausse, et \code{down} pour le choc a la baisse.

\item[\code{choc}] une valeur \code{numeric} indiquant le taux de choc.

\item[\code{choc\_lim}] une valeur \code{numeric} indiquant la limite haute pour le choc a la hausse,
ou une limite basse pour le choc a baisse.

\item[\code{ht}] un objet de la classe \code{\LinkA{HypTech}{HypTech}} contenant differentes
lois de rachat partielles et totales.
\end{ldescription}
\end{Arguments}
%
\begin{Value}
L'objet \code{ht} apres choc.
\end{Value}
\inputencoding{utf8}
\HeaderA{get\_choc\_table}{Applique les chocs de mortalite et de longevite de la formule standard.}{get.Rul.choc.Rul.table}
\aliasA{HypTech}{get\_choc\_table}{HypTech}
%
\begin{Description}\relax
\code{get\_choc\_table} est une methode permettant d'appliquer a l'ensemble des table de mortalite
d'un objet \code{\LinkA{HypTech}{HypTech}} les chocs de mortalite ou de longevite de la formule standard.
\end{Description}
%
\begin{Usage}
\begin{verbatim}
get_choc_table(x, choc)
\end{verbatim}
\end{Usage}
%
\begin{Arguments}
\begin{ldescription}
\item[\code{choc}] une valeur \code{numeric} indiquant le taux de choc.

\item[\code{ht}] un objet de la classe \code{\LinkA{HypTech}{HypTech}} contenant differentes tables de
mortalite.
\end{ldescription}
\end{Arguments}
%
\begin{Value}
L'objet \code{ht} apres choc.
\end{Value}
%
\begin{Author}\relax
Prim'Act
\end{Author}
\inputencoding{utf8}
\HeaderA{get\_comport}{Recuperer les taux de revalorisation cible calcules.}{get.Rul.comport}
\aliasA{HypTech}{get\_comport}{HypTech}
%
\begin{Description}\relax
\code{get\_comport} est une methode permettant d'executer le calcul des taux de revalorisation cible.
\end{Description}
%
\begin{Usage}
\begin{verbatim}
get_comport(x, nom_table, list_rd, tx_cible_prec)
\end{verbatim}
\end{Usage}
%
\begin{Arguments}
\begin{ldescription}
\item[\code{x}] un objet de la classe \code{\LinkA{HypTech}{HypTech}}.

\item[\code{nom\_table}] un nom de la table de parametres de taux cible.

\item[\code{list\_rd}] une liste contenant les rendements de reference. Le format de cette liste est :
\begin{description}

\item[le taux de rendement obligataire] 
\item[le taux de rendement de l'indice action de reference] 
\item[le taux de rendement de l'indice immobilier de reference] 
\item[le taux de rendement de l'indice tresorerie de reference] 

\end{description}


\item[\code{tx\_cible\_prec}] une valeur \code{numeric} correspondant au taux cible de la periode precedente.
\end{ldescription}
\end{Arguments}
%
\begin{Value}
La valeur du taux cible.
\end{Value}
%
\begin{Author}\relax
Prim'Act
\end{Author}
%
\begin{SeeAlso}\relax
Le calcul du taux cible \code{\LinkA{calc\_tx\_cible\_ref\_marche}{calc.Rul.tx.Rul.cible.Rul.ref.Rul.marche}}.
\end{SeeAlso}
\inputencoding{utf8}
\HeaderA{get\_qx\_mort}{Recuperer les taux de deces calcules.}{get.Rul.qx.Rul.mort}
\aliasA{HypTech}{get\_qx\_mort}{HypTech}
%
\begin{Description}\relax
\code{get\_qx\_mort} est une methode permettant d'executer le calcul des taux de deces.
\end{Description}
%
\begin{Usage}
\begin{verbatim}
get_qx_mort(x, nom_table, age, gen)
\end{verbatim}
\end{Usage}
%
\begin{Arguments}
\begin{ldescription}
\item[\code{x}] un objet de la classe \code{\LinkA{HypTech}{HypTech}}.

\item[\code{nom\_table}] un nom de la table de mortalite.

\item[\code{age}] est la valeur \code{numeric} de l'age.

\item[\code{gen}] est la valeur \code{numeric} de la generation.
\end{ldescription}
\end{Arguments}
%
\begin{Value}
Le taux de deces.
\end{Value}
%
\begin{Author}\relax
Prim'Act
\end{Author}
%
\begin{SeeAlso}\relax
Le calcul du taux de deces \code{\LinkA{calc\_qx}{calc.Rul.qx}}.
\end{SeeAlso}
\inputencoding{utf8}
\HeaderA{get\_qx\_rach}{Recuperer les taux de rachat calcules.}{get.Rul.qx.Rul.rach}
\aliasA{HypTech}{get\_qx\_rach}{HypTech}
%
\begin{Description}\relax
\code{get\_qx\_rach} est une methode permettant d'executer le calcul des taux de rachat structurel. Il
peut s'agir soit de taux de rachat partiels, soit de taux de rachat totaux.
\end{Description}
%
\begin{Usage}
\begin{verbatim}
get_qx_rach(x, nom_table, age, anc)
\end{verbatim}
\end{Usage}
%
\begin{Arguments}
\begin{ldescription}
\item[\code{x}] un objet de la classe \code{\LinkA{HypTech}{HypTech}}.

\item[\code{nom\_table}] un nom de la table de rachat.

\item[\code{age}] est la valeur \code{numeric} de l'age.

\item[\code{anc}] est la valeur \code{numeric} de l'anciennete du contrat.
\end{ldescription}
\end{Arguments}
%
\begin{Details}\relax
Selon le nom de la table \code{nom\_table}, le resultat de cette fonction sera un taux
de rachat partiel ou un taux de rachat total.
\end{Details}
%
\begin{Value}
Le taux de rachat.
\end{Value}
%
\begin{Author}\relax
Prim'Act
\end{Author}
%
\begin{SeeAlso}\relax
Le calcul du taux de rachat \code{\LinkA{calc\_rach}{calc.Rul.rach}}.
\end{SeeAlso}
\inputencoding{utf8}
\HeaderA{get\_rach\_dyn}{Recuperer les taux de rachat dynamiques calcules.}{get.Rul.rach.Rul.dyn}
\aliasA{HypTech}{get\_rach\_dyn}{HypTech}
%
\begin{Description}\relax
\code{get\_rach\_dyn} est une methode permettant d'executer le calcul des taux de rachat dynamique.
\end{Description}
%
\begin{Usage}
\begin{verbatim}
get_rach_dyn(x, nom_table, tx_cible, tx_serv)
\end{verbatim}
\end{Usage}
%
\begin{Arguments}
\begin{ldescription}
\item[\code{x}] un objet de la classe \code{\LinkA{HypTech}{HypTech}}.

\item[\code{nom\_table}] un nom de jeu de paramatre de rachat dynamique.

\item[\code{tx\_cible}] est une valeur \code{numeric} correspondant taux de revalorisation cible.

\item[\code{tx\_serv}] est une valeur \code{numeric} correspondant taux de revalorisation servi.
\end{ldescription}
\end{Arguments}
%
\begin{Value}
Le taux de rachat dynamique.
\end{Value}
%
\begin{Author}\relax
Prim'Act
\end{Author}
%
\begin{SeeAlso}\relax
Le calcul du taux de rachat dynamique \code{\LinkA{calc\_rach\_dyn}{calc.Rul.rach.Rul.dyn}}.
\end{SeeAlso}
\inputencoding{utf8}
\HeaderA{HypCanton}{La classe \code{HypCanton}.}{HypCanton}
\keyword{classes}{HypCanton}
%
\begin{Description}\relax
Une class de parametres pour les parametres generaux du canton.
\end{Description}
%
\begin{Section}{Slots}

\begin{description}

\item[\code{tx\_soc}] une valeur \code{numeric} correspondant au taux de prelevements social.

\item[\code{tx\_import}] une valeur \code{numeric} correspondant au taux d'impot sur le resultat.

\item[\code{method\_taux\_cible}] un \code{character} correspond au nom de la methode de calcul du taux cible.

\end{description}
\end{Section}
%
\begin{Note}\relax
Dans la version courante, la valeur de \code{method\_taux\_cible} doit etre parametree a "Meth1".
\end{Note}
%
\begin{Author}\relax
Prim'Act
\end{Author}
\inputencoding{utf8}
\HeaderA{HypTech}{La classe \code{HypTech}.}{HypTech}
\keyword{classes}{HypTech}
%
\begin{Description}\relax
Une classe contenant les listes de tables de mortalite, de rachat, les parametres de rachat dynamique et
les parametres comportementaux qui permettent
de calculer les attentes en matiere de taux de revalorisation cible.
\end{Description}
%
\begin{Details}\relax
Chaque elements de ces liste doit avoir prealablement ete nomme.
\end{Details}
%
\begin{Section}{Slots}

\begin{description}

\item[\code{tables\_mort}] une liste contenant des tables de mortalite au format \code{\LinkA{ParamTableMort}{ParamTableMort}}.

\item[\code{tables\_rach}] une liste contenant des tables de rachat (structurel) au format \code{\LinkA{ParamTableRach}{ParamTableRach}}.

\item[\code{param\_rach\_dyn}] une liste contenant des parametres de rachat dynamique
au format \code{\LinkA{ParamRachDyn}{ParamRachDyn}}.

\item[\code{param\_comport}] une liste contenant des des parametres comportementaux au format \code{\LinkA{ParamComport}{ParamComport}}.

\end{description}
\end{Section}
%
\begin{Author}\relax
Prim'Act
\end{Author}
%
\begin{SeeAlso}\relax
Les classes de parametres contenues \code{\LinkA{ParamTableMort}{ParamTableMort}}, \code{\LinkA{ParamTableRach}{ParamTableRach}},
\code{\LinkA{ParamRachDyn}{ParamRachDyn}}, \code{\LinkA{ParamComport}{ParamComport}}.
La methode pour l'application des chocs de mortalite et de longevite : \code{\LinkA{get\_choc\_table}{get.Rul.choc.Rul.table}}.
La methode pour l'application des chocs de rachat haut et bas : \code{\LinkA{get\_choc\_rach}{get.Rul.choc.Rul.rach}}.
La methode pour la recuperation des parametres comportementaux : \code{\LinkA{get\_comport}{get.Rul.comport}}.
La methode pour la recuperation des taux de deces : \code{\LinkA{get\_qx\_mort}{get.Rul.qx.Rul.mort}}.
La methode pour la recuperation des taux de rachat structurel : \code{\LinkA{get\_qx\_rach}{get.Rul.qx.Rul.rach}}.
La methode pour la recuperation des taux de rachat dynamique : \code{\LinkA{get\_rach\_dyn}{get.Rul.rach.Rul.dyn}}.
\end{SeeAlso}
\inputencoding{utf8}
\HeaderA{hyp\_canton\_load}{Methode permettant de charger la valeur initiale des hypotheses du canton.}{hyp.Rul.canton.Rul.load}
\aliasA{HypCanton}{hyp\_canton\_load}{HypCanton}
%
\begin{Description}\relax
\code{hyp\_canton\_load} est une methode permettant de charger les parametres associees a un
objet de classe \code{\LinkA{HypCanton}{HypCanton}}.
\end{Description}
%
\begin{Usage}
\begin{verbatim}
hyp_canton_load(file_hyp_canton_address)
\end{verbatim}
\end{Usage}
%
\begin{Arguments}
\begin{ldescription}
\item[\code{file\_hyp\_canton\_address}] est un \code{character} contenant l'adresse exacte
du fichier d'input utilisateur
permettant de renseigner un objet \code{\LinkA{HypCanton}{HypCanton}}.
\end{ldescription}
\end{Arguments}
%
\begin{Value}
L'objet de la classe \code{\LinkA{HypCanton}{HypCanton}} construit a partir des inputs renseignes par l'utilisateur.
\end{Value}
%
\begin{Author}\relax
Prim'Act
\end{Author}
%
\begin{SeeAlso}\relax
La classe \code{\LinkA{Initialisation}{Initialisation}} et sa methode \code{\LinkA{set\_architecture}{set.Rul.architecture}}
pour renseigner l’input.
\end{SeeAlso}
\inputencoding{utf8}
\HeaderA{Immo}{Classe pour les actifs de type immobilier.}{Immo}
%
\begin{Description}\relax
Classe pour les actifs de type immobilier.
\end{Description}
%
\begin{Section}{Slots}

\begin{description}

\item[\code{ptf\_immo}] est un dataframe, chaque ligne represente un actif immobilier du portefeuille d'immobilier.

\end{description}
\end{Section}
%
\begin{Author}\relax
Prim'Act
\end{Author}
%
\begin{SeeAlso}\relax
Les operations d'achat vente immo  \code{\LinkA{buy\_immo}{buy.Rul.immo}} et \code{\LinkA{sell\_immo}{sell.Rul.immo}}.
\end{SeeAlso}
\inputencoding{utf8}
\HeaderA{Initialisation}{La classe \code{Initialisation}.}{Initialisation}
\keyword{classes}{Initialisation}
%
\begin{Description}\relax
Une classe permettant de gerer les parametres techniques necessaire a l'initialisation d'une etude.
\end{Description}
%
\begin{Section}{Slots}

\begin{description}

\item[\code{root\_address}] ce \code{character} doit correspondre a la racine du projet. C'est dans les sous dossiers de cet emplacement que l'ensemble des donnees, parametres et dossiers de sauvegarde doivent se situer,
en respectant l'architecture etablie par Prim'Act.

\item[\code{address}] est une liste renseignee par la fonction \code{\LinkA{set\_architecture}{set.Rul.architecture}} qui contient l'ensemble des adresses de l'architecture physique du projet
(emplacement des donnnees utilisateurs, emplacement des parametres utilisateurs, emplacement des sauvegardes temporaires et definitives).

\item[\code{nb\_simu}] est un \code{integer} correspondant aux nombres de trajectoires simulees par le jeu de donnees de l'ESG Prim'Act.

\item[\code{nb\_annee\_proj}] est un \code{integer} correspondant au nombre d'annee de projection de la modelisation.

\end{description}
\end{Section}
%
\begin{Note}\relax
Il est necessaire que l'attribut nb\_annee\_proj corresponde au nombre d'annee de projection des donnees de l'ESG Prim'Act.
\end{Note}
%
\begin{Author}\relax
Prim'Act
\end{Author}
%
\begin{SeeAlso}\relax
La mise en place de l'architecture de chargement des donnees et parametres renseignes par l'utilisateur \code{\LinkA{set\_architecture}{set.Rul.architecture}}, 
la creation et la sauvegarde du canton initial \code{\LinkA{init\_SimBEL}{init.Rul.SimBEL}}, la creation de l'architecture des scenarios central, de marche et de souscription 
de la formule standard ainsi que la creation des objets \code{\LinkA{Be}{Be}} pour chacun de ces scenarios.
\end{SeeAlso}
\inputencoding{utf8}
\HeaderA{initialisation\_load}{Chargement de certains attributs dans un objet \code{Initialisation}}{initialisation.Rul.load}
\aliasA{Initialisation}{initialisation\_load}{Initialisation}
%
\begin{Description}\relax
\code{initialisation\_load} est la methode de chargement des attributs \code{nb\_simu} et \code{nb\_annee\_proj} a partir des donnees de l'environnement utilisateur.
\end{Description}
%
\begin{Usage}
\begin{verbatim}
initialisation_load(x, file_lancement_address)
\end{verbatim}
\end{Usage}
%
\begin{Arguments}
\begin{ldescription}
\item[\code{x}] un objet de la classe \code{\LinkA{Initialisation}{Initialisation}}.

\item[\code{file\_lancement\_address}] nom complet (i.e. avec chemin d'acces et extension) du fichier contenant les parametres de lancement.
\end{ldescription}
\end{Arguments}
%
\begin{Value}
Pas de sortie.
\end{Value}
%
\begin{Note}\relax
Cette methode permet de creer l'objet \code{\LinkA{Canton}{Canton}} initial et de le sauvegarder dans le repertoire adequat de l'architecture.
\end{Note}
%
\begin{Author}\relax
Prim'Act
\end{Author}
\inputencoding{utf8}
\HeaderA{init\_create\_folder}{Creation de l'architecture de sauvegarde des scenarios et executions du code a partir de la racine renseignee.}{init.Rul.create.Rul.folder}
\aliasA{Initialisation}{init\_create\_folder}{Initialisation}
%
\begin{Description}\relax
\code{init\_create\_folder} est une methode permettant de creer l'architecture de sauvegarde des scenarios et les executions du code a partir de la racine renseignee.
\end{Description}
%
\begin{Usage}
\begin{verbatim}
init_create_folder(x)
\end{verbatim}
\end{Usage}
%
\begin{Arguments}
\begin{ldescription}
\item[\code{x}] objet de la classe \code{\LinkA{Initialisation}{Initialisation}}.
\end{ldescription}
\end{Arguments}
%
\begin{Value}
En cas de bonne execution (i.e. l'ensemble des dossiers est cree ou ecrase) la methode renvoie un \code{logical}.
\end{Value}
%
\begin{Note}\relax
Il est necessaire anterieurement a l'appel de cette fonction d'avoir dans un premier temps
cree un objet \code{\LinkA{Initialisation}{Initialisation}} en lui ayant affecte une racine,
puis dans un second temps d'avoir appele la methode \code{\LinkA{set\_architecture}{set.Rul.architecture}} a ce meme objet.
\end{Note}
%
\begin{Author}\relax
Prim'Act
\end{Author}
\inputencoding{utf8}
\HeaderA{init\_debut\_pgg\_psap}{Re-initialise un objet \code{AutresReserves} en debut d'annee.}{init.Rul.debut.Rul.pgg.Rul.psap}
\aliasA{AutresReserves}{init\_debut\_pgg\_psap}{AutresReserves}
%
\begin{Description}\relax
\code{init\_debut\_pgg\_psap} est une methode permettant de re-initialiser les montants
de PGG et de PSAP de debut de periode.
\end{Description}
%
\begin{Usage}
\begin{verbatim}
init_debut_pgg_psap(x)
\end{verbatim}
\end{Usage}
%
\begin{Arguments}
\begin{ldescription}
\item[\code{x}] objet de la classe \code{AutresReserves}.
\end{ldescription}
\end{Arguments}
%
\begin{Value}
L'objet x reinitialise.
\end{Value}
%
\begin{Author}\relax
Prim'Act
\end{Author}
\inputencoding{utf8}
\HeaderA{init\_debut\_ppb}{Re-initialise la PPB en debut d'annee.}{init.Rul.debut.Rul.ppb}
\aliasA{Ppb}{init\_debut\_ppb}{Ppb}
%
\begin{Description}\relax
\code{init\_debut\_ppb} est une methode permettant de re-initialiser les montants de
dotation ou de reprise cumules sur l'annee et
de re-initialiser le montant de PPB de debut de periode.
\end{Description}
%
\begin{Usage}
\begin{verbatim}
init_debut_ppb(x)
\end{verbatim}
\end{Usage}
%
\begin{Arguments}
\begin{ldescription}
\item[\code{x}] un objet de la classe \code{\LinkA{Ppb}{Ppb}}.
\end{ldescription}
\end{Arguments}
%
\begin{Value}
L'objet \code{x} reinitialise.
\end{Value}
%
\begin{Author}\relax
Prim'Act
\end{Author}
\inputencoding{utf8}
\HeaderA{init\_scenario}{Initialisation des scenarios : central et de chocs d'un workspace.}{init.Rul.scenario}
\aliasA{Initialisation}{init\_scenario}{Initialisation}
%
\begin{Description}\relax
\code{init\_scenario} est la methode d'initialisation.
\end{Description}
%
\begin{Usage}
\begin{verbatim}
init_scenario(x)
\end{verbatim}
\end{Usage}
%
\begin{Arguments}
\begin{ldescription}
\item[\code{x}] un objet de la classe \code{\LinkA{Initialisation}{Initialisation}}
\end{ldescription}
\end{Arguments}
%
\begin{Value}
Pas de sortie.
\end{Value}
%
\begin{Note}\relax
Cette methode cree l'architecture, puis les objets \code{\LinkA{Be}{Be}} correspondant a chacun des scenarios : central et de chocs de la formule standard.
\end{Note}
%
\begin{Author}\relax
Prim'Act
\end{Author}
\inputencoding{utf8}
\HeaderA{init\_SimBEL}{Initialisation d'un workspace.}{init.Rul.SimBEL}
\aliasA{Initialisation}{init\_SimBEL}{Initialisation}
%
\begin{Description}\relax
\code{init\_SimBEL} est la methode d'initialisation d'un workspace.
\end{Description}
%
\begin{Usage}
\begin{verbatim}
init_SimBEL(x)
\end{verbatim}
\end{Usage}
%
\begin{Arguments}
\begin{ldescription}
\item[\code{x}] un objet de la classe \code{\LinkA{Initialisation}{Initialisation}}.
\end{ldescription}
\end{Arguments}
%
\begin{Value}
Pas de sortie.
\end{Value}
%
\begin{Note}\relax
Cette methode permet de creer l'objet \code{\LinkA{Canton}{Canton}} initial et de le sauvegarder dans le repertoire adequat de l'architecture.
\end{Note}
%
\begin{Author}\relax
Prim'Act
\end{Author}
\inputencoding{utf8}
\HeaderA{load\_ht}{Methode permettant de charger la valeur des parametres techniques.}{load.Rul.ht}
\aliasA{HypTech}{load\_ht}{HypTech}
%
\begin{Description}\relax
\code{load\_ht} est une methode permettant de charger les parametres associees a un
objet de classe \code{\LinkA{HypTech}{HypTech}}.
\end{Description}
%
\begin{Usage}
\begin{verbatim}
load_ht(x)
\end{verbatim}
\end{Usage}
%
\begin{Arguments}
\begin{ldescription}
\item[\code{x}] est un objet de la classe \code{\LinkA{Initialisation}{Initialisation}} qui est utilise pour renseigner le chemin
d'acces de tous les parametres techniques.
\end{ldescription}
\end{Arguments}
%
\begin{Value}
L'objet de la classe \code{\LinkA{HypTech}{HypTech}} construit a partir des inputs renseignes par l'utilisateur.
\end{Value}
%
\begin{Author}\relax
Prim'Act
\end{Author}
%
\begin{SeeAlso}\relax
La classe \code{\LinkA{Initialisation}{Initialisation}} et sa methode \code{\LinkA{set\_architecture}{set.Rul.architecture}}
pour renseigner l’input.
\end{SeeAlso}
\inputencoding{utf8}
\HeaderA{load\_pp}{Methode permettant de charger et d'instancier un portfeuille de passif.}{load.Rul.pp}
\aliasA{PortPassif}{load\_pp}{PortPassif}
%
\begin{Description}\relax
\code{load\_pp} est une methode permettant de charger les parametres et les donnees associees a un
objet de classe \code{\LinkA{PortPassifs}{PortPassifs}}.
\end{Description}
%
\begin{Usage}
\begin{verbatim}
load_pp(x)
\end{verbatim}
\end{Usage}
%
\begin{Arguments}
\begin{ldescription}
\item[\code{x}] est un objet de la classe \code{\LinkA{Initialisation}{Initialisation}} qui est utilise pour renseigner le chemin
d'acces de tous les parametres et les donnees necessaires.
\end{ldescription}
\end{Arguments}
%
\begin{Value}
L'objet de la classe \code{\LinkA{PortPassif}{PortPassif}} construit a partir des inputs renseignes par l'utilisateur.
\end{Value}
%
\begin{Author}\relax
Prim'Act
\end{Author}
%
\begin{SeeAlso}\relax
La classe \code{\LinkA{Initialisation}{Initialisation}} et sa methode \code{\LinkA{set\_architecture}{set.Rul.architecture}}
pour renseigner l’input.
\end{SeeAlso}
\inputencoding{utf8}
\HeaderA{ModelPointESG}{La classe \code{ModelPointESG}.}{ModelPointESG}
\keyword{classes}{ModelPointESG}
%
\begin{Description}\relax
Une classe pour une extraction de l'ES pour une annee et une simulation particuliere.
\end{Description}
%
\begin{Section}{Slots}

\begin{description}

\item[\code{annee}] une valeur \code{integer} correspondant a l'annee de projection.

\item[\code{num\_traj}] une valeur \code{integer} correspondant au numero de simulation de l'ESG.

\item[\code{indice\_action}] un \code{data.frame} contenant les valeurs prises par les indices actions pour l'annee et la
simulation selectionnees.

\item[\code{indice\_immo}] un \code{data.frame} contenant les valeurs prises par les indices immobiliers pour l'annee et la
simulation selectionnees.

\item[\code{indice\_inflation}] une valeur \code{numeric} correspondant a la valeur prise par l'indice inflation
pour l'annee et la simulation selectionnees.

\item[\code{yield\_curve}] un vecteur \code{numeric} contenant la structure par terme des taux d'interets spots
pour l'annee et la simulation selectionnees. La courbe representee correspond aux valeurs des
R(k, k+i) ou i va de 1 au \code{nb\_annee\_proj}.

\item[\code{deflateur}] une valeur \code{numeric} correspondant a la valeur prise par le deflateur stochastique
pour l'annee et la simulation selectionnees.

\end{description}
\end{Section}
%
\begin{Author}\relax
Prim'Act
\end{Author}
%
\begin{SeeAlso}\relax
Les methodes de chargement d'un ESG \code{\LinkA{chargement\_ESG}{chargement.Rul.ESG}} et d'extraction d'un model point \code{\LinkA{extract\_ESG}{extract.Rul.ESG}}.
\end{SeeAlso}
\inputencoding{utf8}
\HeaderA{Oblig}{Classe pour les actifs de type obligation.}{Oblig}
%
\begin{Description}\relax
Classe pour les actifs de type obligation.
\end{Description}
%
\begin{Section}{Slots}

\begin{description}

\item[\code{ptf\_oblig}] est un dataframe, chaque ligne represente un actif obligation du portefeuille d'obligation.

\end{description}
\end{Section}
%
\begin{Author}\relax
Prim'Act
\end{Author}
%
\begin{SeeAlso}\relax
Les operations d'achat vente obligations  \code{\LinkA{buy\_oblig}{buy.Rul.oblig}} et \code{\LinkA{sell\_oblig}{sell.Rul.oblig}}.
\end{SeeAlso}
\inputencoding{utf8}
\HeaderA{ParamAlmEngine}{La classe \code{ParamAlmEngine}.}{ParamAlmEngine}
\keyword{classes}{ParamAlmEngine}
%
\begin{Description}\relax
Une classe pour le parametre ALM d'un canton.
\end{Description}
%
\begin{Section}{Slots}

\begin{description}

\item[\code{ptf\_reference}] est un objet de type \code{\LinkA{PortFin}{PortFin}}, qui represente le portefeuille
d'investissement de reference d'un canton.

\item[\code{alloc\_cible}] un vecteur de 4 elements rendant compte du pourcentage
de l'actif composant respectivement les investissements: actions, immobiliers, obligataires et de tresorerie.

\item[\code{seuil\_realisation\_PVL}] une valeur \code{numeric} correspondant au pourcentage de plus-values actions
qui peut etre liquidee chaque annee pour atteindre l'objectif de revalorisation cible des passifs.

\end{description}
\end{Section}
%
\begin{Author}\relax
Prim'Act
\end{Author}
\inputencoding{utf8}
\HeaderA{ParamBe}{La classe \code{ParamBe}.}{ParamBe}
\keyword{classes}{ParamBe}
%
\begin{Description}\relax
Une classe contenant le nombre d'annees de projection utilise
pour le calcul du best estimate d'un assureur.
\end{Description}
%
\begin{Section}{Slots}

\begin{description}

\item[\code{nb\_annee}] un entier comprenant le nombre d'annees de projection.

\end{description}
\end{Section}
%
\begin{Author}\relax
Prim'Act
\end{Author}
\inputencoding{utf8}
\HeaderA{ParamChocMket}{La classe \code{ParamChocMket}.}{ParamChocMket}
\keyword{classes}{ParamChocMket}
%
\begin{Description}\relax
Une classe contenant les parametres des chocs de marche de la formule standard.
\end{Description}
%
\begin{Section}{Slots}

\begin{description}

\item[\code{table\_choc\_action\_type1}] un \code{data.frame} contenant les parametres du choc action type 1.

\item[\code{table\_choc\_action\_type2}] un \code{data.frame} contenant les parametres du choc action type 2.

\item[\code{table\_choc\_immo}] un \code{data.frame} contenant les parametres du choc immobilier.

\item[\code{table\_choc\_spread}] un \code{data.frame} contenant les parametres du choc de spread.

\end{description}
\end{Section}
%
\begin{Author}\relax
Prim'Act
\end{Author}
\inputencoding{utf8}
\HeaderA{ParamChocSousc}{La classe \code{ParamChocSousc}.}{ParamChocSousc}
\keyword{classes}{ParamChocSousc}
%
\begin{Description}\relax
Une classe contenant les parametres des chocs souscription de la formule standard.
\end{Description}
%
\begin{Section}{Slots}

\begin{description}

\item[\code{mp}] un \code{data.frame} contenant l'ensemble des parametres necessaires a l'application des
chocs du module Souscription Vie.

\end{description}
\end{Section}
%
\begin{Author}\relax
Prim'Act
\end{Author}
\inputencoding{utf8}
\HeaderA{ParamComport}{La classe de parametres de comportement \code{ParamComport}.}{ParamComport}
\keyword{classes}{ParamComport}
%
\begin{Description}\relax
Une classe pour les parametres de comportement.
\end{Description}
%
\begin{Section}{Slots}

\begin{description}

\item[\code{mat\_oblig}] une valeur \code{numeric} correspondant a la maturite du taux de rendement obligataire pris en
reference sur le marche.

\item[\code{alloc\_mar}] un vecteur \code{numeric} correspondant a l'allocation pris en reference sur le marche.
Le format de cette liste est :
\begin{description}

\item[le taux de rendement obligataire] 
\item[le taux de rendement de l'indice action de reference] 
\item[le taux de rendement de l'indice immobilier de reference] 
\item[le taux de rendement de l'indice tresorerie de reference.] 

\end{description}


\item[\code{w\_n}] une valeur \code{numeric} correspondant au poids accorde au rendement de l'annee courante par
rapport a l'annee precedente.

\item[\code{marge\_mar}] une valeur \code{numeric} correspondant a la marge financiere pris en reference sur le marche.

\item[\code{ch\_enc\_mar}] une valeur \code{numeric} correspondant au niveau de chargement sur encours
pris en reference sur le marche.

\item[\code{ind\_ref\_action}] une valeur \code{numeric} correspondant au numero de l'indice action
pris en reference sur le marche.

\item[\code{ind\_ref\_immo}] une valeur \code{numeric} correspondant au numero de l'indice immobilier
pris en reference sur le marche.

\end{description}
\end{Section}
%
\begin{Author}\relax
Prim'Act
\end{Author}
%
\begin{SeeAlso}\relax
Le calcul du taux cible \code{\LinkA{calc\_tx\_cible\_ref\_marche}{calc.Rul.tx.Rul.cible.Rul.ref.Rul.marche}}.
\end{SeeAlso}
\inputencoding{utf8}
\HeaderA{ParamRachDyn}{La classe de parametres de rachat dynamique \code{ParamRachDyn}.}{ParamRachDyn}
\keyword{classes}{ParamRachDyn}
%
\begin{Description}\relax
Une classe pour les parametres de des lois de rachat dynamique.
\end{Description}
%
\begin{Section}{Slots}

\begin{description}

\item[\code{vec\_param}] un \code{data frame} contenant les parametres pour les rachats dynamiques.

\end{description}
\end{Section}
%
\begin{Author}\relax
Prim'Act
\end{Author}
%
\begin{SeeAlso}\relax
Le calcul du taux de rachat dynamique \code{\LinkA{calc\_rach\_dyn}{calc.Rul.rach.Rul.dyn}}.
\end{SeeAlso}
\inputencoding{utf8}
\HeaderA{ParamRevaloEngine}{La classe \code{ParamRevaloEngine}. Une Classe pour les parametres utilises pour la gestion de la revalorisation.}{ParamRevaloEngine}
\keyword{classes}{ParamRevaloEngine}
%
\begin{Description}\relax
La classe \code{ParamRevaloEngine}.
Une Classe pour les parametres utilises pour la gestion de la revalorisation.
\end{Description}
%
\begin{Section}{Slots}

\begin{description}

\item[\code{taux\_pb\_fi}] une valeur \code{numeric} correspondant au taux de participation applique au resultat financier.

\item[\code{taux\_pb\_tech}] une valeur \code{numeric} correspondant au taux de participation applique au resultat technique.

\item[\code{tx\_marge\_min}] une valeur \code{numeric} correspondant au taux de marge minimal auquel s'attend l'assureur.

\item[\code{solde\_pb\_regl}] une valeur \code{numeric} correspondant au solde deficitaire de participation aux
benefices reglementaire. Cette valeur doit etre negative.

\end{description}
\end{Section}
%
\begin{Author}\relax
Prim'Act
\end{Author}
\inputencoding{utf8}
\HeaderA{ParamTableMort}{La classe de parametres pour les table de rachat \code{ParamTableRach}.}{ParamTableMort}
\keyword{classes}{ParamTableMort}
%
\begin{Description}\relax
Une classe de parametres pour les tables de rachat.

Une classe de parametres pour les tables de mortalite.
\end{Description}
%
\begin{Details}\relax
Une table de rachat peut etre une table de rachat partiel ou une table de rachat total.
Pour une table de rachat partiel, les taux de rachat sont exprimes en pourcentage de l'encours.
Pour une table de rachat total, les taux de rachat sont exprimes en pourcentage du nombre de contrats.
\end{Details}
%
\begin{Section}{Slots}

\begin{description}

\item[\code{age\_min}] un entier correspondant a l'age minimal de la table.

\item[\code{age\_max}] un entier correspondant a l'age maximal de la table.

\item[\code{anc\_min}] un entier correspondant a la premiere anciennete de la table.

\item[\code{anc\_max}] un entier correspondant a la derniere anciennete de la table.

\item[\code{table}] un \code{data frame} contenant la table de rachat.

\item[\code{age\_min}] un entier correspondant a l'age minimal de la table.

\item[\code{age\_max}] un entier correspondant a l'age maximal de la table.

\item[\code{gen\_min}] un entier correspondant a la premiere generation de la table.

\item[\code{gen\_max}] un entier correspondant a la derniere generation de la table.

\item[\code{table}] un \code{data frame} contenant la table de mortalite.

\end{description}
\end{Section}
%
\begin{Note}\relax
Les tables de mortalite doivent contenir des effectifs sous risque par age (Lx).
\end{Note}
%
\begin{Author}\relax
Prim'Act

Prim'Act
\end{Author}
%
\begin{SeeAlso}\relax
Le calcul du taux de rachat \code{\LinkA{calc\_rach}{calc.Rul.rach}}.

Le calcul du taux de deces \code{\LinkA{calc\_qx}{calc.Rul.qx}}.
\end{SeeAlso}
\inputencoding{utf8}
\HeaderA{param\_alm\_engine\_load}{Chargement des attributs d'un objet \code{ParamAlmEngine} a partir des donnees utilisateurs.}{param.Rul.alm.Rul.engine.Rul.load}
\aliasA{ParamAlmEngine}{param\_alm\_engine\_load}{ParamAlmEngine}
%
\begin{Description}\relax
\code{param\_alm\_engine\_load} est la methode de chargement des attributs d'un objet \code{\LinkA{ParamAlmEngine}{ParamAlmEngine}}
a partir des donnees de l'environnement utilisateur et d'un portefeuille financier de reference (charge par la fonction \code{\LinkA{chargement\_PortFin\_ref}{chargement.Rul.PortFin.Rul.ref}}.
\end{Description}
%
\begin{Usage}
\begin{verbatim}
param_alm_engine_load(file_alm_address, ptf_fin_ref)
\end{verbatim}
\end{Usage}
%
\begin{Arguments}
\begin{ldescription}
\item[\code{file\_alm\_address}] un \code{character} contenant l'adresse exacte
du fichier d'input utilisateur.

\item[\code{ptf\_fin\_ref}] un objet de la classe \code{\LinkA{PortFin}{PortFin}} correspondant au portefeuille de reinvestissement.
\end{ldescription}
\end{Arguments}
%
\begin{Value}
L'objet de la classe \code{\LinkA{ParamAlmEngine}{ParamAlmEngine}} construit a partir des inputs renseignes par l'utilisateur.
\end{Value}
%
\begin{Author}\relax
Prim'Act
\end{Author}
\inputencoding{utf8}
\HeaderA{param\_revalo\_load}{Chargement des attributs d'un objet \code{ParamRevaloEngine} a partir des donnees utilisateurs.}{param.Rul.revalo.Rul.load}
\aliasA{ParamRevaloEngine}{param\_revalo\_load}{ParamRevaloEngine}
%
\begin{Description}\relax
\code{param\_revalo\_load} est la methode de chargement des attributs d'un objet \code{\LinkA{ParamRevaloEngine}{ParamRevaloEngine}}
a partir des donnees de l'environnement utilisateur.
\end{Description}
%
\begin{Usage}
\begin{verbatim}
param_revalo_load(file_revalo_address)
\end{verbatim}
\end{Usage}
%
\begin{Arguments}
\begin{ldescription}
\item[\code{file\_revalo\_address}] un \code{character} contenant l'adresse exacte
du fichier d'input utilisateur.
\end{ldescription}
\end{Arguments}
%
\begin{Value}
L'objet de la classe \code{\LinkA{ParamRevaloEngine}{ParamRevaloEngine}} construit a partir des inputs renseignes par l'utilisateur.
\end{Value}
%
\begin{Author}\relax
Prim'Act
\end{Author}
\inputencoding{utf8}
\HeaderA{pb\_contr}{Calcule la PB contractuelle.}{pb.Rul.contr}
\aliasA{RevaloEngine}{pb\_contr}{RevaloEngine}
%
\begin{Description}\relax
\code{pb\_contr} est une methode permettant de calculer la participation aux benefices contractuelle par produit.
\end{Description}
%
\begin{Usage}
\begin{verbatim}
pb_contr(base_fin, tx_pb, rev_stock_brut, ch_enc_th, tx_enc_moy)
\end{verbatim}
\end{Usage}
%
\begin{Arguments}
\begin{ldescription}
\item[\code{base\_fin}] un vecteur \code{numeric} comprenant par produit la base de produits financiers.

\item[\code{tx\_pb}] un vecteur \code{numeric} comprenant par produit les taux de participation aux benefices contractuels.

\item[\code{rev\_stock\_brut}] un vecteur de type \code{numeric} comprenant par produit la revalorisation
appliquee sur le stock au taux minimum.

\item[\code{ch\_enc\_th}] est un vecteur de type \code{numeric} comprenant par produit le montant total
des chargements sur encours appliques au stock et revalorises au taux minimum. Il s'agit ici des chargements
qui pourraient theoriquement etre preleves.

\item[\code{tx\_enc\_moy}] un vecteur \code{numeric} comprenant par produit
les taux de chargements sur encours moyens.
\end{ldescription}
\end{Arguments}
%
\begin{Details}\relax
Le montant des chargements \code{ch\_enc\_th} est theorique et peut
conduire a l'application d'une revalorisation nette negative.
\end{Details}
%
\begin{Value}
\code{ch\_enc\_ap\_pb\_contr} un vecteur comprenant par produit les chargements sur encours appliques

\code{rev\_stock\_nette\_contr} un vecteur comprenant par produit la
revalorisation contractuelle nette.
\end{Value}
%
\begin{Author}\relax
Prim'Act
\end{Author}
\inputencoding{utf8}
\HeaderA{PortFin}{La classe PortFin}{PortFin}
\keyword{classes}{PortFin}
%
\begin{Description}\relax
Classe pour le portefeuille global d'actif
\end{Description}
%
\begin{Section}{Slots}

\begin{description}

\item[\code{ptf\_action}] est un objet de type \code{\LinkA{Action}{Action}}, qui represente le portefeuille d'action d'un canton.

\item[\code{ptf\_immo}] est un objet de type \code{\LinkA{Immo}{Immo}}, qui represente le portefeuille immobilier d'un canton.

\item[\code{ptf\_oblig}] est un objet de type \code{\LinkA{Oblig}{Oblig}}, qui represente le portefeuille obligataire d'un canton.

\item[\code{ptf\_treso}] est un objet de type \code{\LinkA{Treso}{Treso}}, qui represente le portefeuille monetaire d'un canton.

\item[\code{pre}] est un objet de type \code{\LinkA{PRE}{PRE}}, qui represente la PRE d'un canton.

\item[\code{rc}] est un objet de type \code{\LinkA{RC}{RC}}, qui represente la RC d'un canton.

\item[\code{frais\_fin}] est un objet de type \code{\LinkA{FraisFin}{FraisFin}}, qui represente les frais financiers d'un canton.

\item[\code{pvl\_action}] est un \code{numeric}, qui correspond a la somme des plus values latentes des actifs Actions qui sont en situation de plus values latentes.

\item[\code{pvl\_immo}] est un \code{numeric}, qui correspond a la somme des plus values latentes des actifs Immo qui sont en situation de plus values latentes.

\item[\code{pvl\_oblig}] est un \code{numeric}, qui correspond a la somme des plus values latentes des actifs Obligs qui sont en situation de plus values latentes.

\item[\code{mvl\_action}] est un \code{numeric}, qui correspond a la somme des moins values latentes des actifs Actions qui sont en situation de moins values latentes.

\item[\code{mvl\_immo}] est un \code{numeric}, qui correspond a la somme des moins values latentes des actifs Immos qui sont en situation de moins values latentes.

\item[\code{mvl\_oblig}] est un \code{numeric}, qui correspond a la somme des moins values latentes des actifs Obligs qui sont en situation de moins values latentes.

\item[\code{vm\_vnc\_precedent}] est une liste composee de deux elements : la \code{vm\_precedente} et la
\code{vnc\_precedente}, correspondant respectivement a la valeur de marche
et a la valeur nette comptable en debut d'annee de l'objet PortFin.

\end{description}
\end{Section}
%
\begin{Author}\relax
Prim'Act
\end{Author}
\inputencoding{utf8}
\HeaderA{PortPassif}{La classe \code{PortPassif}.}{PortPassif}
\keyword{classes}{PortPassif}
%
\begin{Description}\relax
Une classe regroupant l'ensemble des donnees de passifs et les hypotheses correspondantes.
\end{Description}
%
\begin{Section}{Slots}

\begin{description}

\item[\code{annee}] une valeur entiere correspondant a l'annee de projection.

\item[\code{eei}] une liste d'objets de la classe \code{\LinkA{EpEuroInd}{EpEuroInd}} contenant l'ensemble
des produits de type epargne en euros.

\item[\code{names\_class\_prod}] un vecteur \code{character} indiquant les noms de classes de produits.

\item[\code{ht}] un objet de classe \code{\LinkA{HypTech}{HypTech}} contenant les hypotheses techniques.

\item[\code{fp}] un objet de classe \code{\LinkA{FraisPassif}{FraisPassif}} contenant les hypotheses de frais de passif
par produit.

\item[\code{tx\_pb}] un objet de classe \code{\LinkA{TauxPB}{TauxPB}} contenant les taux contractuel de participation
aux benefices par produit.

\item[\code{autres\_passifs}] un objet de classe \code{\LinkA{AutresPassifs}{AutresPassifs}}.

\item[\code{autres\_reserves}] un objet de classe \code{\LinkA{AutresReserves}{AutresReserves}}.

\end{description}
\end{Section}
%
\begin{Author}\relax
Prim'Act
\end{Author}
%
\begin{SeeAlso}\relax
La projection des produits sur l'annee avant attributiuon de participation
aux benefices : \code{\LinkA{proj\_annee\_av\_pb}{proj.Rul.annee.Rul.av.Rul.pb}}.
Le vieillissement des model points de passifs avant et apres attributiuon de participation
aux benefices : \code{\LinkA{vieillissement\_av\_pb}{vieillissement.Rul.av.Rul.pb}}, \code{\LinkA{vieillissement\_ap\_pb}{vieillissement.Rul.ap.Rul.pb}}.
\end{SeeAlso}
\inputencoding{utf8}
\HeaderA{Ppb}{La classe \code{Ppb}.}{Ppb}
\keyword{classes}{Ppb}
%
\begin{Description}\relax
Classe pour la provision pour participation aux benefices (PPB)
\end{Description}
%
\begin{Section}{Slots}

\begin{description}

\item[\code{valeur\_ppb}] est la valeur courante \code{numeric} prise par la PPB.

\item[\code{ppb\_debut}] est la valeur prise \code{numeric} par la PPB en debut d'annee.

\item[\code{seuil\_rep}] est une valeur \code{numeric} correspond a la proportion de PPB de debut d'annee
que l'on peut reprendre sur une periode.

\item[\code{seuil\_dot}] est une valeur \code{numeric} correspond a la montant maximal de dotation possible sur la PPB
sur une periode, exprimee comme une fraction de la PPB de debut d'annee.

\item[\code{compte\_rep}] est une valeur \code{numeric} qui totalise les montants de reprise effectuee sur une periode.

\item[\code{compte\_dot}] est une valeur \code{numeric} qui totalise les montants de dotation effectuee sur une periode.

\end{description}
\end{Section}
%
\begin{Author}\relax
Prim'Act
\end{Author}
%
\begin{SeeAlso}\relax
La dotation et la reprise de PPB : \code{\LinkA{calc\_dotation\_ppb}{calc.Rul.dotation.Rul.ppb}}, \code{\LinkA{calc\_reprise\_ppb}{calc.Rul.reprise.Rul.ppb}}.
\end{SeeAlso}
\inputencoding{utf8}
\HeaderA{ppb\_load}{Methode permettant de charger les valeurs des hypotheses et des donnees de PPB}{ppb.Rul.load}
\aliasA{HypCanton}{ppb\_load}{HypCanton}
%
\begin{Description}\relax
\code{ppb\_load} est une methode permettant de charger les parametres associees a un
objet de classe \code{\LinkA{Ppb}{Ppb}}.
\end{Description}
%
\begin{Usage}
\begin{verbatim}
ppb_load(file_ppb_address)
\end{verbatim}
\end{Usage}
%
\begin{Arguments}
\begin{ldescription}
\item[\code{file\_ppb\_address}] est un \code{character} contenant l'adresse exacte
du fichier d'input utilisateur
permettant de renseigner un objet \code{\LinkA{Ppb}{Ppb}}.
\end{ldescription}
\end{Arguments}
%
\begin{Value}
L'objet de la classe \code{\LinkA{Ppb}{Ppb}} construit a partir des inputs renseignes par l'utilisateur.
\end{Value}
%
\begin{Author}\relax
Prim'Act
\end{Author}
%
\begin{SeeAlso}\relax
La classe \code{\LinkA{Initialisation}{Initialisation}} et sa methode \code{\LinkA{set\_architecture}{set.Rul.architecture}}
pour renseigner l’input.
\end{SeeAlso}
\inputencoding{utf8}
\HeaderA{PRE}{La classe PRE}{PRE}
\keyword{classes}{PRE}
%
\begin{Description}\relax
Classe pour la gestion de la provision pour risque d'exigibilite (PRE).
\end{Description}
%
\begin{Section}{Slots}

\begin{description}

\item[\code{val\_debut}] est une valeur \code{numeric} correspondant a la valeur de la PRE en debut d'annee.

\item[\code{val\_courante}] est une valeur \code{numeric} correspondant a la valeur courante de la PRE.

\end{description}
\end{Section}
%
\begin{Author}\relax
Prim'Act
\end{Author}
%
\begin{SeeAlso}\relax
Les methodes de calcul de la PRE \code{\LinkA{calc\_PRE}{calc.Rul.PRE}}, et de mises a jour des PRE initiales et courantes \code{\LinkA{do\_update\_PRE\_val\_courante}{do.Rul.update.Rul.PRE.Rul.val.Rul.courante}}, \code{\LinkA{do\_update\_PRE\_val\_debut}{do.Rul.update.Rul.PRE.Rul.val.Rul.debut}}.
\end{SeeAlso}
\inputencoding{utf8}
\HeaderA{pre\_load}{Chargement de la valeur initiale de la PRE}{pre.Rul.load}
\aliasA{PRE}{pre\_load}{PRE}
%
\begin{Description}\relax
\code{pre\_load} est une methode permettant de charger la valeur de PRE initiale dans un objet de type PRE.
\end{Description}
%
\begin{Usage}
\begin{verbatim}
pre_load(file_PRE_address)
\end{verbatim}
\end{Usage}
%
\begin{Arguments}
\begin{ldescription}
\item[\code{file\_PRE\_address}] est un \code{character} correspondant a l'adresse du fichier d'input renseignant les donnees de PRE
\end{ldescription}
\end{Arguments}
%
\begin{Value}
Un objet de la classe \code{PRE} charge a partir des donnees du fichier dont le nom est precise en input.
\end{Value}
%
\begin{Author}\relax
Prim'Act
\end{Author}
\inputencoding{utf8}
\HeaderA{print\_alloc}{Calcul le poids de chaque composante du portefeuille action.}{print.Rul.alloc}
\aliasA{PortFin}{print\_alloc}{PortFin}
%
\begin{Description}\relax
\code{pint\_alloc} est une methode permettant de calculer l'allocation absolue et relative du portefeuille.
\end{Description}
%
\begin{Usage}
\begin{verbatim}
print_alloc(x)
\end{verbatim}
\end{Usage}
%
\begin{Arguments}
\begin{ldescription}
\item[\code{x}] objet de la classe PortFin.
\end{ldescription}
\end{Arguments}
%
\begin{Value}
Un data frame compose de quatre colonnes et cinq lignes.
La colonne \begin{description}

\item[\code{alloc\_valeur} : ] decrit le montant alloue en valeur de marche par poche d'actif.
\item[\code{alloc\_proportion} : ] decrit la proportion allouee en valeur de marche par poche d'actif.
\item[\code{alloc\_valeur\_vnc} : ] decrit le montant alloue en valeur nette comptable par poche d'actif.
\item[\code{alloc\_proportion\_vnc} : ] decrit la proportion allouee en valeur nette comptable par poche d'actif.

\end{description}

Les lignes correspondent aux classes d'actifs : (Action / Immobilier / Obligation / Tresorerie / Actifs cumules)
\end{Value}
%
\begin{Author}\relax
Prim'Act
\end{Author}
\inputencoding{utf8}
\HeaderA{proj\_an}{Projette un canton sur une periode.}{proj.Rul.an}
\aliasA{Canton}{proj\_an}{Canton}
%
\begin{Description}\relax
\code{proj\_an} est une methode permettant de projeter un canton sur une annee. Cette methode calcule
les flux de best estimate des passifs et fait vieillir d'une annee les elements du canton.
\end{Description}
%
\begin{Usage}
\begin{verbatim}
proj_an(x, annee_fin, pre_on)
\end{verbatim}
\end{Usage}
%
\begin{Arguments}
\begin{ldescription}
\item[\code{x}] est un objet de type \code{\LinkA{Canton}{Canton}}.

\item[\code{annee\_fin}] est une valeur \code{numeric} correpondant a l'annee de fin de projection.

\item[\code{pre\_on}] est une valeur \code{logical} qui lorsqu'elle vaut \code{TRUE} prend en compte la variation
de PRE dans le resultat technique, utilisee pour le calcul de la participation aux benefices reglementaires.
\end{ldescription}
\end{Arguments}
%
\begin{Details}\relax
Cette methode est la procedure central du package \code{SimBEL} puisqu'elle cohorde les interactions entre
les actifs et les passifs, declenche l'algorithme de revalorisation, calcule le resultat comptable et evalue les 
flux de best estimate.
\end{Details}
%
\begin{Value}
\code{canton} l'objet  \code{x} vieilli d'une annee.

\code{annee} l'annee de projection.

\code{nom\_produit} le nom des produits de passifs consideres.

\code{output\_produit} une liste comprenant les variables de flux, les variables de stocks et les resultats
des passifs non-modelises.

\code{output\_be} une liste comprenant les flux utilises pour le calcul du best estimate par produit.

\code{result\_tech} la valeur du resultat technique.

\code{result\_fin} la valeur du resultat financier.

\code{tra} la valeur du taux de rendement de l'actif.

\code{result\_brut} la valeur du resultat brut d'impot.

\code{result\_net} la valeur du resultat net d'impot.
\end{Value}
%
\begin{Author}\relax
Prim'Act
\end{Author}
%
\begin{SeeAlso}\relax
Le viellissement du portefeuille de passif avant PB : \code{\LinkA{viellissement\_av\_pb}{viellissement.Rul.av.Rul.pb}}.
Le viellissement du portefeuille financier : \code{\LinkA{update\_PortFin}{update.Rul.PortFin}}, \code{\LinkA{update\_PortFin\_reference}{update.Rul.PortFin.Rul.reference}}.
L'affiche de l'etat courant du portefeuille financier : \code{\LinkA{print\_alloc}{print.Rul.alloc}}.
Le calcul des frais financier : \code{\LinkA{calc\_frais\_fin}{calc.Rul.frais.Rul.fin}}.
La reallocation du portefeuille financier : \code{\LinkA{reallocate}{reallocate}}.
Le calcul de la PRE : \code{\LinkA{calc\_PRE}{calc.Rul.PRE}}.
Le calcul du resultat technique : \code{\LinkA{calc\_result\_technique}{calc.Rul.result.Rul.technique}}, \code{\LinkA{calc\_result\_technique\_ap\_pb}{calc.Rul.result.Rul.technique.Rul.ap.Rul.pb}}.
Le calcul du resultat financier et du TRA : \code{\LinkA{calc\_resultat\_fin}{calc.Rul.resultat.Rul.fin}}, \code{\LinkA{calc\_tra}{calc.Rul.tra}}.
L'application de l'algorithme d'attribution de la participation aux benefices : \code{\LinkA{calc\_revalo}{calc.Rul.revalo}}.
Le viellissement du portefeuille de passif apres PB : \code{\LinkA{viellissement\_ap\_pb}{viellissement.Rul.ap.Rul.pb}}.
Les autres methodes de vieillissement des actifs et de passifs: \code{\LinkA{sell\_pvl\_action}{sell.Rul.pvl.Rul.action}}, 
\code{\LinkA{do\_update\_pmvl}{do.Rul.update.Rul.pmvl}}, \code{\LinkA{do\_update\_PRE\_val\_courante}{do.Rul.update.Rul.PRE.Rul.val.Rul.courante}},
\code{\LinkA{do\_update\_vm\_vnc\_precedent}{do.Rul.update.Rul.vm.Rul.vnc.Rul.precedent}}, \code{\LinkA{init\_debut\_ppb}{init.Rul.debut.Rul.ppb}}, \code{\LinkA{do\_update\_RC\_val\_debut}{do.Rul.update.Rul.RC.Rul.val.Rul.debut}},
\code{\LinkA{do\_update\_PRE\_val\_debut}{do.Rul.update.Rul.PRE.Rul.val.Rul.debut}}, \code{\LinkA{init\_debut\_pgg\_psap}{init.Rul.debut.Rul.pgg.Rul.psap}}.
Le calcul des fins de projection : \code{\LinkA{calc\_fin\_proj}{calc.Rul.fin.Rul.proj}}.
\end{SeeAlso}
\inputencoding{utf8}
\HeaderA{proj\_annee\_autres\_passifs}{Extrait les flux et les PM des produits non modelises}{proj.Rul.annee.Rul.autres.Rul.passifs}
\aliasA{AutresPassifs}{proj\_annee\_autres\_passifs}{AutresPassifs}
%
\begin{Description}\relax
\code{proj\_annee\_autres\_passifs} est une methode permettant de calculer les PM et les flux sur une annee pour
des passif non modelises.
Cette methode calcule applique une inflation au frais.
\end{Description}
%
\begin{Usage}
\begin{verbatim}
proj_annee_autres_passifs(an, x, coef_inf)
\end{verbatim}
\end{Usage}
%
\begin{Arguments}
\begin{ldescription}
\item[\code{an}] est l'annee de projection.

\item[\code{x}] un objet de la classe \code{AutresPassifs} contenant l'ensemble des donnees de passifs non modelises.

\item[\code{coef\_inf}] un \code{numeric} correpodant au coefficient d'inflation a appliquer sur les frais.
\end{ldescription}
\end{Arguments}
%
\begin{Value}
Un \code{data.frame} contenant les flux des passifs de l'annee.
\end{Value}
%
\begin{Author}\relax
Prim'Act
\end{Author}
\inputencoding{utf8}
\HeaderA{proj\_annee\_av\_pb}{Calcule les flux et les PM des produits modelises}{proj.Rul.annee.Rul.av.Rul.pb}
\aliasA{PortPassif}{proj\_annee\_av\_pb}{PortPassif}
%
\begin{Description}\relax
\code{proj\_annee\_av\_pb} est une methode permettant de calculer les PM et les flux sur une annee avant PB.
Cette methode calcule egalement les frais sur flux et sur primes.
\end{Description}
%
\begin{Usage}
\begin{verbatim}
proj_annee_av_pb(an, x, tx_soc, coef_inf, list_rd)
\end{verbatim}
\end{Usage}
%
\begin{Arguments}
\begin{ldescription}
\item[\code{an}] une valeur \code{numeric} correspondant a l'annee de projection.

\item[\code{x}] un objet de la classe \code{\LinkA{PortPassif}{PortPassif}} contenant l'ensemble des produits de passifs.

\item[\code{tx\_soc}] une valeur \code{numeric} correspondant au taux de charges sociales.

\item[\code{coef\_inf}] une valeur \code{numeric} correspondant au coefficient d'inflation
considere pour le traitement des frais.

\item[\code{list\_rd}] une liste contenant les rendements de reference. Le format de cette liste est :
\begin{description}

\item[le taux de rendement obligataire] 
\item[le taux de rendement de l'indice action de reference] 
\item[le taux de rendement de l'indice immobilier de reference] 
\item[le taux de rendement de l'indice tresorerie de reference] 

\end{description}

\end{ldescription}
\end{Arguments}
%
\begin{Details}\relax
L'annee de projection est utilisee pour gerer les produits dont les clauses dependent de l'annee.
Cette methode calcule deux fois les prestations et les PM pour permettre de calculer le montant de FDB.
\end{Details}
%
\begin{Value}
\code{x} l'objet pour lequel les tableaux de resultats des objets \code{\LinkA{EpEuroInd}{EpEuroInd}} sont mis a jour.

\code{nom\_produit} un vecteur de \code{character} contenant les noms des produits.

\code{flux\_agg} une matrice contenant les flux aggreges par produits.

\code{stock\_agg} une matrice contenant les stocks aggreges par produits.
\end{Value}
%
\begin{Author}\relax
Prim'Act
\end{Author}
%
\begin{SeeAlso}\relax
La classe \code{\LinkA{EpEuroInd}{EpEuroInd}} et ses methodes.
La classe \code{\LinkA{FraisPassif}{FraisPassif}} et ses methodes.
\end{SeeAlso}
\inputencoding{utf8}
\HeaderA{RC}{La classe RC}{RC}
\keyword{classes}{RC}
%
\begin{Description}\relax
Classe pour la gestion de la reserve de capitalisation (RC).
\end{Description}
%
\begin{Section}{Slots}

\begin{description}

\item[\code{val\_debut}] est une valeur \code{numeric} correspondant a la valeur de la RC en debut d'annee.

\item[\code{val\_courante}] est une valeur \code{numeric} correspondant a la valeur courante de la RC.

\end{description}
\end{Section}
%
\begin{Author}\relax
Prim'Act
\end{Author}
%
\begin{SeeAlso}\relax
Les methodes de calcul de la RC \code{\LinkA{calc\_RC}{calc.Rul.RC}}, et de mises a jour des RC initiales et courantes \code{\LinkA{do\_update\_RC\_val\_courante}{do.Rul.update.Rul.RC.Rul.val.Rul.courante}}, \code{\LinkA{do\_update\_RC\_val\_debut}{do.Rul.update.Rul.RC.Rul.val.Rul.debut}}.
\end{SeeAlso}
\inputencoding{utf8}
\HeaderA{rc\_load}{Chargement de la valeur initiale de la RC}{rc.Rul.load}
\aliasA{RC}{rc\_load}{RC}
%
\begin{Description}\relax
\code{rc\_load} est une methode permettant de charger la valeur de RC initiale dans un objet de type RC.
\end{Description}
%
\begin{Usage}
\begin{verbatim}
rc_load(file_RC_address)
\end{verbatim}
\end{Usage}
%
\begin{Arguments}
\begin{ldescription}
\item[\code{file\_RC\_address}] est un \code{character} correspondant a l'adresse du fichier d'input renseignant les donnees de RC
\end{ldescription}
\end{Arguments}
%
\begin{Value}
Un objet de la classe \code{RC} charge a partir des donnees du fichier dont le nom est precise en input.
\end{Value}
%
\begin{Author}\relax
Prim'Act
\end{Author}
\inputencoding{utf8}
\HeaderA{reallocate}{Realise les operations d'achats ventes}{reallocate}
\aliasA{AlmEngine}{reallocate}{AlmEngine}
%
\begin{Description}\relax
\code{reallocate} est une methode permettant d'ajuster l'allocation du \code{\LinkA{PortFin}{PortFin}} de l'assureur.
\end{Description}
%
\begin{Usage}
\begin{verbatim}
reallocate(x, ptf_reference, alloc_cible)
\end{verbatim}
\end{Usage}
%
\begin{Arguments}
\begin{ldescription}
\item[\code{x}] objet de la classe \code{\LinkA{PortFin}{PortFin}}.

\item[\code{ptf\_reference}] est le portefeuille de reinvestissement. C'est un objet de la classe \code{\LinkA{PortFin}{PortFin}}.

\item[\code{alloc\_cible}] est un vecteur de type \code{numeric} constitue de 4 elements, il contient les proportions cibles d'allocations
action, immobilier, obligataire et de tresorerie.
\end{ldescription}
\end{Arguments}
%
\begin{Value}
\code{portFin} l'objet initial de la classe \code{\LinkA{PortFin}{PortFin}} realloue a l'allocation cible.

\code{pmvr} le montant total des plus ou moins values realisess.

\code{pmvr\_oblig} le montant des plus ou moins values obligataires realisees lors de la reallocation.

\code{pmvr\_action} le montant des plus ou moins values action realisees lors de l'etape de reallocation.

\code{pmvr\_immo}  le montant des plus ou moins values immobilieres realisees lors de l'etape de reallocation.

\code{var\_rc} la variation de la reserve de capitalisation induite par la reallocation.

\code{var\_pre} la variation de la provision pour risque d'exigibilite induite par la reallocation.

\code{plac\_moy\_vm} la valeur de marche moyenne des placements de l'assureur au cours de l'operation de reallocation.

\code{plac\_moy\_vnc} la valeur nette comptable moyenne des placements de l'assureur au cours de l'operation de reallocation.
\end{Value}
%
\begin{Note}\relax
Les operations d'achat/vente sont effectuees en termes de nombre d'unite d'achat/vente.
\end{Note}
%
\begin{Author}\relax
Prim'Act
\end{Author}
%
\begin{SeeAlso}\relax
La classe \code{\LinkA{PortFin}{PortFin}}.
\end{SeeAlso}
\inputencoding{utf8}
\HeaderA{resultat\_fin}{Calcul de resultat financier}{resultat.Rul.fin}
\aliasA{PortFin}{resultat\_fin}{PortFin}
%
\begin{Description}\relax
\code{calc\_resultat\_fin} est une methode permettant de calculer le resultat financier du portfeuille.
\end{Description}
%
\begin{Usage}
\begin{verbatim}
calc_resultat_fin(revenu, produit, frais_fin, var_rc)
\end{verbatim}
\end{Usage}
%
\begin{Arguments}
\begin{ldescription}
\item[\code{revenu}] est un objet de type \code{numeric}, qui fournit les revenus du portefeuille financier.

\item[\code{produit}] est un objet de type \code{numeric}, qui fournit le produit (ou la perte) des cessions.

\item[\code{frais\_fin}] est un objet de type \code{numeric}, qui fournit le montant des frais financiers.

\item[\code{var\_rc}] est un objet de type\code{numeric}, donnant la variation de la reserve de capitalisation.
\end{ldescription}
\end{Arguments}
%
\begin{Value}
La valeur du result financier.
\end{Value}
%
\begin{Author}\relax
Prim'Act
\end{Author}
\inputencoding{utf8}
\HeaderA{RevaloEngine}{La classe \code{RevaloEngine}.}{RevaloEngine}
\keyword{classes}{RevaloEngine}
%
\begin{Description}\relax
Une classe comprenant les methodes pour l'application de la revalorisation des passifs.
\end{Description}
%
\begin{Section}{Slots}

\begin{description}

\item[\code{param\_revalo}] est objet de type \code{\LinkA{ParamRevalo}{ParamRevalo}} comprenant
les parametres utilises pour la revalorisation des contrats.

\end{description}
\end{Section}
%
\begin{Author}\relax
Prim'Act
\end{Author}
\inputencoding{utf8}
\HeaderA{revalo\_action}{Calcul les valeurs de marches de chaque composante du portefeuille action.}{revalo.Rul.action}
\aliasA{Action}{revalo\_action}{Action}
%
\begin{Description}\relax
\code{revalo\_action} est une methode permettant de calculer les valeurs de marche.
\end{Description}
%
\begin{Usage}
\begin{verbatim}
revalo_action(x, S, S_prev)
\end{verbatim}
\end{Usage}
%
\begin{Arguments}
\begin{ldescription}
\item[\code{x}] objet de la classe \code{Action} (decrivant le portefeuille d'action).

\item[\code{S}] vecteur \code{numeric} de valeur de chaque stock du ptf en milieu d'annee N (date de versement des dividendes)

\item[\code{S\_prev}] vecteur \code{numeric}  de valeur de chaque stock du ptf en milieu d'annee N-1.
\end{ldescription}
\end{Arguments}
%
\begin{Value}
Un data frame compose de deux colonnes et autant de lignes que le portefeuille action a de lignes.
La premiere colonne decrit de le rendement annuel de chacune des actions composants le portefeuille action.
La seconde colonne decrit les dividendes annuelles percues au titre de chacune des actions composants le portefeuille action.
\end{Value}
%
\begin{Author}\relax
Prim'Act
\end{Author}
\inputencoding{utf8}
\HeaderA{revalo\_immo}{Calcul les valeurs de marches de chaque composante du portefeuille immobilier.}{revalo.Rul.immo}
\aliasA{Immo}{revalo\_immo}{Immo}
%
\begin{Description}\relax
\code{revalo\_immo} est une methode permettant de calculer les valeurs de marche.
\end{Description}
%
\begin{Usage}
\begin{verbatim}
revalo_immo(x, S, S_prev)
\end{verbatim}
\end{Usage}
%
\begin{Arguments}
\begin{ldescription}
\item[\code{x}] objet de la classe \code{Immo} (decrivant le portefeuille d'immobilier).

\item[\code{S}] vecteur \code{numeric} de valeur de chaque stock du ptf en milieu d'annee N (date de versement des dividendes)

\item[\code{S\_prev}] vecteur \code{numeric}  de valeur de chaque stock du ptf en milieu d'annee N-1.
\end{ldescription}
\end{Arguments}
%
\begin{Value}
Un data frame compose de deux colonnes et autant de lignes que le portefeuille immobilier a de lignes.
La premiere colonne decrit de le rendement annuel de chacune des lignes d'immobilier composants le portefeuille immobilier.
La seconde colonne decrit les dividendes annuelles percues au titre de chacune des lignes d'immobilier composants le portefeuille immobilier.
\end{Value}
%
\begin{Author}\relax
Prim'Act
\end{Author}
\inputencoding{utf8}
\HeaderA{revalo\_treso}{Calcul les valeurs de marches de chaque composante du portefeuille treso.}{revalo.Rul.treso}
\aliasA{Treso}{revalo\_treso}{Treso}
%
\begin{Description}\relax
\code{revalo\_treso} est une methode permettant de calculer les valeurs de marche.
\end{Description}
%
\begin{Usage}
\begin{verbatim}
revalo_treso(Rt, Rt_prev)
\end{verbatim}
\end{Usage}
%
\begin{Arguments}
\begin{ldescription}
\item[\code{S}] vecteur de valeur de chaque ligne du ptf en milieu d'annee N (date de calcul des flux).

\item[\code{S\_prev}] vecteur de valeur de chaque ligne du ptf en milieu d'annee N-1.
\end{ldescription}
\end{Arguments}
%
\begin{Value}
Un vecteur ayant autant d elements que les vecteurs inputs. Chaque element correspondant au rendement annuel d'une lige de tresorerie.
\end{Value}
%
\begin{Author}\relax
Prim'Act
\end{Author}
\inputencoding{utf8}
\HeaderA{revenu\_treso}{Calcul le revenu tresorerie.}{revenu.Rul.treso}
\aliasA{Treso}{revenu\_treso}{Treso}
%
\begin{Description}\relax
\code{revenu\_treso} est une methode permettant de calculer les valeurs de marche.
\end{Description}
%
\begin{Usage}
\begin{verbatim}
revenu_treso(x, rdt, flux_milieu)
\end{verbatim}
\end{Usage}
%
\begin{Arguments}
\begin{ldescription}
\item[\code{x}] est un objet de la classe Treso en debut d'annee

\item[\code{rdt}] est le rendement de la classe Treso au cours de l'annee (i.e. en fin d'annee)

\item[\code{flux\_milieu}] est le flux du milieu de l'annee en cours (i.e. ulterieur a l'objet Treso renseigne)
\end{ldescription}
\end{Arguments}
%
\begin{Value}
Le montant du revenu.
\end{Value}
%
\begin{Author}\relax
Prim'Act
\end{Author}
\inputencoding{utf8}
\HeaderA{run\_be}{Calcul d'un BE.}{run.Rul.be}
\aliasA{Be}{run\_be}{Be}
%
\begin{Description}\relax
\code{run\_be} est une methode permettant de calculer un best estimate pour un canton.
\end{Description}
%
\begin{Usage}
\begin{verbatim}
run_be(x, pre_on)
\end{verbatim}
\end{Usage}
%
\begin{Arguments}
\begin{ldescription}
\item[\code{x}] un objet de type \code{\LinkA{Be}{Be}}.

\item[\code{pre\_on}] une valeur \code{logical} qui lorsqu'elle vaut \code{TRUE} prend en compte la variation
de PRE dans le resultat technique utilisee pour le calcul de la participation aux benefices reglementaires.
\end{ldescription}
\end{Arguments}
%
\begin{Details}\relax
Il s'agit de la methode principale du package \code{SimBEL}. Cette methode requiert le chargement
d'un objet \code{\LinkA{Be}{Be}} deja parametre et alimente en donnees. La methode \code{\LinkA{init\_scenario}{init.Rul.scenario}}
permet d'alimenter un objet \code{\LinkA{Be}{Be}} dans la situation "centrale" de la formule standard et en situation
de choc.
\end{Details}
%
\begin{Value}
\code{be} l'objet \code{x} mis a jour : l'attribut \code{tab\_be} contient le best estimate et sa
decomposition, l'attribut \code{tab\_flux} contient les flux moyens du best estimate et ses
composantes.

\code{err\_simu} un vecteur contenant la liste des simulations qui ont generes des erreurs et qui n'ont pu
etre utilisees pour le calcul du best estimate.
\end{Value}
%
\begin{Author}\relax
Prim'Act
\end{Author}
%
\begin{SeeAlso}\relax
Le calcul du best estimate pour une simulation : \code{\LinkA{run\_be\_simu}{run.Rul.be.Rul.simu}}.
L'initialisation d'un best estimate : \code{\LinkA{init\_scenario}{init.Rul.scenario}}.
La classe \code{\LinkA{Be}{Be}}.
La sortie des resultats au format ".csv" : \code{\LinkA{write\_be\_results}{write.Rul.be.Rul.results}}.
\end{SeeAlso}
\inputencoding{utf8}
\HeaderA{run\_be\_simu}{Calcul d'un BE par une simulation.}{run.Rul.be.Rul.simu}
\aliasA{Be}{run\_be\_simu}{Be}
%
\begin{Description}\relax
\code{run\_be\_simu} est une methode permettant de calculer un best estimate
pour une simulation donnee.
\end{Description}
%
\begin{Usage}
\begin{verbatim}
run_be_simu(x, i, pre_on)
\end{verbatim}
\end{Usage}
%
\begin{Arguments}
\begin{ldescription}
\item[\code{x}] un objet de type \code{Be}.

\item[\code{i}] un entier (\code{integer}) correspondant au numero de la simulation.

\item[\code{pre\_on}] une valeur \code{logical} qui lorsqu'elle vaut \code{TRUE} prend en compte la variation
de PRE dans le resultat technique utilisee pour le calcul de la participation aux benefices reglementaires.
\end{ldescription}
\end{Arguments}
%
\begin{Details}\relax
Pour une simulation donnee, cette methode projette un \code{\LinkA{Canton}{Canton}} jusqu'au terme, parametre dans
l'objet \code{x}.
\end{Details}
%
\begin{Value}
\code{nom\_produit} un vecteur contenant le liste des noms de produits.

\code{prime} une matrice contenant les flux de primes par produit.

\code{prestation} une matrice contenant les flux de prestations par produit.

\code{prestation\_fdb} une matrice contenant les flux de prestations discretionnaires par produit.

\code{frais} une matrice contenant les flux de frais par produit.

\code{flux\_be} une matrice contenant les flux de best estimate par produit.

\code{prime\_actu} une matrice contenant la valeur des primes actualisees par produit.

\code{prestation\_actu} une matrice contenant la valeur des prestations actualisees par produit.

\code{prestation\_fdb\_actu} une matrice contenant la valeur des prestations
discretionnaires actualisees par produit.

\code{frais\_actu} une matrice contenant la valeur des frais actualisees par produit.

\code{be} une matrice contenant la valeur du best estimate par produit.
\end{Value}
%
\begin{Author}\relax
Prim'Act
\end{Author}
%
\begin{SeeAlso}\relax
La methode de projection d'un \code{\LinkA{Canton}{Canton}} : \code{\LinkA{proj\_an}{proj.Rul.an}}.
L'extraction d'une simulation de l'\code{\LinkA{ESG}{ESG}} :\code{\LinkA{extract\_ESG}{extract.Rul.ESG}}.
La classe \code{\LinkA{Be}{Be}}.
\end{SeeAlso}
\inputencoding{utf8}
\HeaderA{sell\_action}{Mise a jour de chaque composante du portefeuille action suite a la vente de tout ou partie de ce portefeuille.}{sell.Rul.action}
\aliasA{Action}{sell\_action}{Action}
%
\begin{Description}\relax
\code{sell\_action} est une methode permettant de mettre a jour chaque composante d'un portefeuille action suite a la vente
de tout ou partie de ce portefeuille.
\end{Description}
%
\begin{Usage}
\begin{verbatim}
sell_action(x, num_sold, nb_sold)
\end{verbatim}
\end{Usage}
%
\begin{Arguments}
\begin{ldescription}
\item[\code{x}] objet de la classe \code{Action} (decrivant le portefeuille action en detention).

\item[\code{num\_sold}] vecteur de type \code{numeric} contenant le numero de model point action du portefeuille que l'on souhaite vendre.

\item[\code{nb\_sold}] vecteur de type \code{numeric} contenant le nombre d'unite que l'on souhaite vendre (a autant de ligne que le vecteur num\_sold).
\end{ldescription}
\end{Arguments}
%
\begin{Value}
L'objet \code{x} mis a jour de l'operation de vente (suppression des lignes vendues).
\end{Value}
%
\begin{Author}\relax
Prim'Act
\end{Author}
\inputencoding{utf8}
\HeaderA{sell\_immo}{Mise a jour de chaque composante du portefeuille immobilier suite a la vente de tout ou partie de ce portefeuille.}{sell.Rul.immo}
\aliasA{Immo}{sell\_immo}{Immo}
%
\begin{Description}\relax
\code{sell\_immo} est une methode permettant de mettre a jour chaque composante d'un portefeuille immobilier suite a la vente
de tout ou partie de ce portefeuille.
\end{Description}
%
\begin{Usage}
\begin{verbatim}
sell_immo(x, num_sold, nb_sold)
\end{verbatim}
\end{Usage}
%
\begin{Arguments}
\begin{ldescription}
\item[\code{x}] objet de la classe \code{immo} (decrivant le portefeuille immobilier en detention).

\item[\code{num\_sold}] vecteur de type \code{numeric} contenant le numero de model point immobilier du portefeuille que l'on souhaite vendre.

\item[\code{nb\_sold}] vecteur de type \code{numeric} contenant le nombre d'unite que l'on souhaite vendre (a autant de ligne que le vecteur num\_sold).
\end{ldescription}
\end{Arguments}
%
\begin{Value}
L'objet \code{x} mis a jour de l'operation de vente (suppression des lignes vendues).
\end{Value}
%
\begin{Author}\relax
Prim'Act
\end{Author}
\inputencoding{utf8}
\HeaderA{sell\_oblig}{Mise a jour de chaque composante du portefeuille obligation suite a la vente de tout ou partie de ce portefeuille.}{sell.Rul.oblig}
\aliasA{Oblig}{sell\_oblig}{Oblig}
%
\begin{Description}\relax
\code{sell\_oblig} est une methode permettant de mettre a jour chaque composante d'un portefeuille obligation suite a la vente
de tout ou partie de ce portefeuille.
\end{Description}
%
\begin{Usage}
\begin{verbatim}
sell_oblig(x, num_sold, nb_sold)
\end{verbatim}
\end{Usage}
%
\begin{Arguments}
\begin{ldescription}
\item[\code{x}] objet de la classe \code{Oblig} (decrivant le portefeuille obligation en detention).

\item[\code{num\_sold}] vecteur de type \code{numeric} contenant le numero de model point obligation du portefeuille que l'on souhaite vendre.

\item[\code{nb\_sold}] vecteur de type \code{numeric} contenant le nombre d'unite que l'on souhaite vendre (a autant de ligne que le vecteur num\_sold).
\end{ldescription}
\end{Arguments}
%
\begin{Value}
L'objet \code{x} mis a jour de l'operation de vente (suppression des lignes vendues).
\end{Value}
%
\begin{Author}\relax
Prim'Act
\end{Author}
\inputencoding{utf8}
\HeaderA{sell\_pvl\_action}{Mise a jour de chaque composante du portefeuille action suite a une realisation d'un montant de plus values latentes action.}{sell.Rul.pvl.Rul.action}
\aliasA{Action}{sell\_pvl\_action}{Action}
%
\begin{Description}\relax
\code{sell\_pvl\_action} est une methode permettant de mettre a jour chaque composante d'un portefeuille action suite a la vente
de tout ou partie de ce portefeuille afin de realiser un montant de plus values latentes.
\end{Description}
%
\begin{Usage}
\begin{verbatim}
sell_pvl_action(x, montant)
\end{verbatim}
\end{Usage}
%
\begin{Arguments}
\begin{ldescription}
\item[\code{x}] objet de la classe \code{Action} (decrivant le portefeuille action en detention).

\item[\code{montant}] reel de type \code{numeric} contient le montant de plus value latente que l'on souhaite realiser.
\end{ldescription}
\end{Arguments}
%
\begin{Value}
L'objet \code{x} mis a jour de l'operation de vente (suppression des lignes vendues) et pmvr le montant de plus value realisees.
\end{Value}
%
\begin{Author}\relax
Prim'Act
\end{Author}
\inputencoding{utf8}
\HeaderA{set\_architecture}{Definition de l'architecture d'un workspace.}{set.Rul.architecture}
\aliasA{Initialisation}{set\_architecture}{Initialisation}
%
\begin{Description}\relax
\code{set\_architecture}.
\end{Description}
%
\begin{Usage}
\begin{verbatim}
set_architecture(x)
\end{verbatim}
\end{Usage}
%
\begin{Arguments}
\begin{ldescription}
\item[\code{x}] un objet de la classe \code{\LinkA{Initialisation}{Initialisation}}.
\end{ldescription}
\end{Arguments}
%
\begin{Value}
Objet mis a jour de l'ensemble des chemins du workspace,
ceux ci sont stockes sous forme de liste dans l'attribut \code{address} 
de l'objet \code{\LinkA{Initialisation}{Initialisation}} renseigne en input.
\end{Value}
%
\begin{Author}\relax
Prim'Act
\end{Author}
\inputencoding{utf8}
\HeaderA{SimBEL}{SimBEL: Un package de calcul du best estimate epargne sous Solvabilite 2.}{SimBEL}
\aliasA{SimBEL-package}{SimBEL}{SimBEL.Rdash.package}
%
\begin{Description}\relax
SimBEL fourni un ensemble de fonctionnalites pour permettre l'evaluation d'un best
estimate epargne sous Solvabilite 2. L'utilisation de ce package necessite au prealable
de disposer de donnees stockees dans un repertoire dont le format est predetermine par
la societe Prim'Act. Ce package est developpe a partir d'objet de type S4.
\end{Description}
%
\begin{Details}\relax
Ce package comprends :
\begin{itemize}

\item une modelisation d'un canton auquel est relie un portefeuille d'actifs et un portefeuille de passif.
SimBEL gere les interactions entre ces deux objets.
\item une modelisation du best estimate pour des produits d'epargne en euros.
\item d'appliquer les principaux chocs de la formule standard.

\end{itemize}

\end{Details}
\inputencoding{utf8}
\HeaderA{TabEpEuroInd}{La classe \code{TabEpEuroInd}.}{TabEpEuroInd}
\keyword{classes}{TabEpEuroInd}
%
\begin{Description}\relax
Une classe pour le stockage en memoire de variable de calcul au niveau du model point \code{\LinkA{EpEuroInd}{EpEuroInd}}.
\end{Description}
%
\begin{Section}{Slots}

\begin{description}

\item[\code{tab}] un objet \code{list} au format fige contenant l'ensemble des variables stockees.

\end{description}
\end{Section}
%
\begin{Author}\relax
Prim'Act
\end{Author}
\inputencoding{utf8}
\HeaderA{TauxPB}{La classe \code{TauxPB}.}{TauxPB}
\keyword{classes}{TauxPB}
%
\begin{Description}\relax
Une classe pour le stockage des parametres de taux de participation contractuelle par produit.
\end{Description}
%
\begin{Section}{Slots}

\begin{description}

\item[\code{mp}] un \code{data frame} contenant les parametres des taux de participation contractuelle par produit.

\end{description}
\end{Section}
%
\begin{Author}\relax
Prim'Act
\end{Author}
\inputencoding{utf8}
\HeaderA{Treso}{La classe Treso}{Treso}
\keyword{classes}{Treso}
%
\begin{Description}\relax
Classe pour les actifs de type Tresorerie
\end{Description}
%
\begin{Section}{Slots}

\begin{description}

\item[\code{ptf\_treso}] est un dataframe, chaque ligne represente un actif de tresorerie du portefeuille de monetaire.

\end{description}
\end{Section}
%
\begin{Author}\relax
Prim'Act
\end{Author}
%
\begin{SeeAlso}\relax
Les methodes de calcul des valeurs \code{\LinkA{calc\_vm\_treso}{calc.Rul.vm.Rul.treso}}, 
de calcul des revenus de la tresorerie \code{\LinkA{revenu\_treso}{revenu.Rul.treso}}, 
de calcul de la revalorisation de la tresorerie \code{\LinkA{revalo\_treso}{revalo.Rul.treso}},
de mise a jour de la tresorerie \code{\LinkA{update\_treso}{update.Rul.treso}}.
\end{SeeAlso}
\inputencoding{utf8}
\HeaderA{update\_cc\_oblig}{Mise a jour des coupons courus d'un portefeuille obligataire.}{update.Rul.cc.Rul.oblig}
\aliasA{Oblig}{update\_cc\_oblig}{Oblig}
%
\begin{Description}\relax
\code{update\_cc\_oblig} est une methode permettant de mettre a jour les coupons courus des composantes d'un portefeuille obligataire.
\end{Description}
%
\begin{Usage}
\begin{verbatim}
update_cc_oblig(x, coupon)
\end{verbatim}
\end{Usage}
%
\begin{Arguments}
\begin{ldescription}
\item[\code{x}] objet de la classe \code{Oblig} (decrivant le portefeuille obligataire en detention).

\item[\code{coupon}] un vecteur de \code{numeric} a assigner a l'objet \code{Obligation}.
\end{ldescription}
\end{Arguments}
%
\begin{Value}
L'objet \code{x} dont les coupons courus ont ete mis a jour
\end{Value}
%
\begin{Author}\relax
Prim'Act
\end{Author}
\inputencoding{utf8}
\HeaderA{update\_dur\_det\_action}{Mise a jour des durees de detention d'un portefeuille action.}{update.Rul.dur.Rul.det.Rul.action}
\aliasA{Action}{update\_dur\_det\_action}{Action}
%
\begin{Description}\relax
\code{update\_dur\_det\_action} est une methode permettant de mettre a jour la duree de detention des composantes d'un portefeuille Action.
\end{Description}
%
\begin{Usage}
\begin{verbatim}
update_dur_det_action(x)
\end{verbatim}
\end{Usage}
%
\begin{Arguments}
\begin{ldescription}
\item[\code{x}] objet de la classe \code{Action} (decrivant le portefeuille action en detention).
\end{ldescription}
\end{Arguments}
%
\begin{Value}
L'objet \code{x} mis a jour du vieillissement de la duree de detention.
\end{Value}
%
\begin{Author}\relax
Prim'Act
\end{Author}
\inputencoding{utf8}
\HeaderA{update\_dur\_det\_immo}{Mise a jour des durees de detention des composantes d'un portefeuille immobilier.}{update.Rul.dur.Rul.det.Rul.immo}
\aliasA{Immo}{update\_dur\_det\_immo}{Immo}
%
\begin{Description}\relax
\code{update\_dur\_det\_immo} est une methode permettant de mettre a jour la duree de detention des composantes d'un portefeuille immobilier.
\end{Description}
%
\begin{Usage}
\begin{verbatim}
update_dur_det_immo(x)
\end{verbatim}
\end{Usage}
%
\begin{Arguments}
\begin{ldescription}
\item[\code{x}] objet de la classe \code{Immo} (decrivant le portefeuille immo en detention).
\end{ldescription}
\end{Arguments}
%
\begin{Value}
L'objet \code{x} mis a jour du vieillissement de la duree de detention.
\end{Value}
%
\begin{Author}\relax
Prim'Act
\end{Author}
\inputencoding{utf8}
\HeaderA{update\_dur\_oblig}{Mise a jour des duration d'un portefeuille obligataire.}{update.Rul.dur.Rul.oblig}
\aliasA{Oblig}{update\_dur\_oblig}{Oblig}
%
\begin{Description}\relax
\code{update\_dur\_oblig} est une methode permettant de mettre a jour la duration des composantes d'un portefeuille obligataire.
\end{Description}
%
\begin{Usage}
\begin{verbatim}
update_dur_oblig(x, duration)
\end{verbatim}
\end{Usage}
%
\begin{Arguments}
\begin{ldescription}
\item[\code{x}] objet de la classe \code{Oblig} (decrivant le portefeuille obligataire en detention).

\item[\code{duration}] un vecteur de \code{numeric} a assigner a l'objet \code{Obligation}.
\end{ldescription}
\end{Arguments}
%
\begin{Value}
L'objet \code{x} dont les durations ont ete mises a jour.
\end{Value}
%
\begin{Author}\relax
Prim'Act
\end{Author}
\inputencoding{utf8}
\HeaderA{update\_mat\_res}{Mise a jour de la maturite residuelle et de la duree de detention de chaque composante d'un portefeuille obligataire.}{update.Rul.mat.Rul.res}
\aliasA{Oblig}{update\_mat\_res}{Oblig}
%
\begin{Description}\relax
\code{update\_mat\_res} est une methode permettant de mettre a jour la maturite residuelle et la duree de detention
de chaque composante d'un portefeuille obligataire.
\end{Description}
%
\begin{Usage}
\begin{verbatim}
update_mat_res(x)
\end{verbatim}
\end{Usage}
%
\begin{Arguments}
\begin{ldescription}
\item[\code{x}] objet de la classe Oblig (decrivant le portefeuille obligataire).
\end{ldescription}
\end{Arguments}
%
\begin{Value}
L'objet x dont
\begin{description}

\item[\code{mat\_res} : ] est diminuee d'une unite (une unite correspond a un an)
\item[\code{dur\_det} : ] est augmentee d'une unite (une unite correspond a un an)

\end{description}

\end{Value}
%
\begin{Author}\relax
Prim'Act
\end{Author}
\inputencoding{utf8}
\HeaderA{update\_PortFin}{Evalue et met a jour les objets constituants un PortFin.}{update.Rul.PortFin}
\aliasA{PortFin}{update\_PortFin}{PortFin}
%
\begin{Description}\relax
\code{update\_PortFin} est une methode permettant de calculer et mettre a jour un portefeuille financier
suite a un vieillissement.
\end{Description}
%
\begin{Usage}
\begin{verbatim}
update_PortFin(an, x, new_mp_ESG, flux_milieu, flux_fin)
\end{verbatim}
\end{Usage}
%
\begin{Arguments}
\begin{ldescription}
\item[\code{an}] \code{numeric} correspond a l'annee de projection du portefeuille financier.

\item[\code{x}] objet de la classe \code{PortFin}, correspondant au portefeuille financier de l'assureur avant l'etape de vieillissement.

\item[\code{new\_mp\_ESG}] est un objet de la classe \code{ModelPointESG}, decrivant les conditions economiques permettant d'effectuer le vieillissement du portefeuille financier.

\item[\code{flux\_milieu}] est une valeur \code{numeric} correspondant a la somme des flux percus en milieu d'annee (coupons des obligations, loyers immobiliers, dividendes des actions, revenus de la tresorerie).

\item[\code{flux\_fin}] est une valeur \code{numeric}  correspondant a la somme des flux percus en fin d'annee (tombee d'echeance d'obligation).
\end{ldescription}
\end{Arguments}
%
\begin{Value}
Le format de la liste renvoyee est :
\begin{description}

\item[\code{ptf} : ] un vecteur contenant les flux de sortie en echeance de l'annee
\item[\code{revenu\_fin} : ] les revenus realises au cours de la periode (coupons, tombees d'echeance, dividendes et loyers).
\item[\code{var\_vnc\_oblig} : ] la variation de valeur nette comptable obligataire.
\end{description}

\end{Value}
%
\begin{Author}\relax
Prim'Act
\end{Author}
%
\begin{SeeAlso}\relax
La fonction de mise a jour specifique au portefeuille de reinvestissement \code{\LinkA{update\_PortFin\_reference}{update.Rul.PortFin.Rul.reference}}.
\end{SeeAlso}
\inputencoding{utf8}
\HeaderA{update\_PortFin\_reference}{Evalue et met a jour les objets constituants un PortFin\_reference.}{update.Rul.PortFin.Rul.reference}
\aliasA{PortFin}{update\_PortFin\_reference}{PortFin}
%
\begin{Description}\relax
\code{update\_PortFin\_reference} est une methode permettant de calculer et mettre a jour un portefeuille financier
de reinvestissement suite a un vieillissement.
\end{Description}
%
\begin{Usage}
\begin{verbatim}
update_PortFin_reference(an, x, mp_ESG)
\end{verbatim}
\end{Usage}
%
\begin{Arguments}
\begin{ldescription}
\item[\code{an}] \code{numeric} correspond a l'annee de projection du portefeuille financier de reinvestissement.

\item[\code{x}] objet de la classe \code{PortFin}, correspondant au portefeuille financier de reinvestissement avant l'etape de vieillissement.

\item[\code{mp\_ESG}] est un objet de la classe \code{ModelPointESG}, decrivant les conditions economiques permettant d'effectuer le vieillissement du portefeuille financier de reinvestissement.
\end{ldescription}
\end{Arguments}
%
\begin{Value}
L'objet de la classe \code{PortFin} renvoye correspond au portefeuille financier de reinvesitssement veilli d'une annee.
\end{Value}
%
\begin{Author}\relax
Prim'Act
\end{Author}
%
\begin{SeeAlso}\relax
La fonction de mise a jour specifique au portefeuille \code{\LinkA{update\_PortFin}{update.Rul.PortFin}}.
\end{SeeAlso}
\inputencoding{utf8}
\HeaderA{update\_reserves}{Evalue et met a jour la valeur des autres reserves.}{update.Rul.reserves}
\aliasA{AutresReserves}{update\_reserves}{AutresReserves}
%
\begin{Description}\relax
\code{update\_reserves} est une methode permettant de calculer la valeur de la nouvelle PGG et de la nouvelle
PSAP et les met a jour.
\end{Description}
%
\begin{Usage}
\begin{verbatim}
update_reserves(x, prest_ep, prest_autres, pm_ep, pm_autres)
\end{verbatim}
\end{Usage}
%
\begin{Arguments}
\begin{ldescription}
\item[\code{x}] objet de la classe \code{AutresReserves}.

\item[\code{prest\_ep}] est une valeur \code{numeric} correspondant a la somme des prestations nettes de chargement et
de charges sociales sur epargne.

\item[\code{prest\_autres}] est une valeur \code{numeric} correspondant a la somme des prestations nettes
de chargements et de charges sociales sur autres passifs.

\item[\code{pm\_ep}] est une valeur \code{numeric}  correspondant a la somme des PM nettes de chargements et
de charges sociales sur epargne.

\item[\code{pm\_autres}] est une valeur \code{numeric} correspondant a la somme des PM nettes de chargement et
de charges sociales sur autres passifs.
\end{ldescription}
\end{Arguments}
%
\begin{Value}
\code{x} l'objet  \code{AutresReserves} mis a jour.

\code{var\_psap} une valeur \code{numeric} correspondant a la variation de PSAP.

\code{var\_gg} une valeur \code{numeric} correspondant a la variation de PGG.
\end{Value}
%
\begin{Note}\relax
Il s'agit d'une methode simplifiee.
\end{Note}
%
\begin{Author}\relax
Prim'Act
\end{Author}
\inputencoding{utf8}
\HeaderA{update\_sd\_oblig}{Mise a jour des surcotes decotes d'un portefeuille obligataire.}{update.Rul.sd.Rul.oblig}
\aliasA{Oblig}{update\_sd\_oblig}{Oblig}
%
\begin{Description}\relax
\code{update\_sd\_oblig} est une methode permettant de mettre a jour la surcotes decotes des composantes d'un portefeuille obligataire.
\end{Description}
%
\begin{Usage}
\begin{verbatim}
update_sd_oblig(x, sd)
\end{verbatim}
\end{Usage}
%
\begin{Arguments}
\begin{ldescription}
\item[\code{x}] objet de la classe \code{Oblig} (decrivant le portefeuille obligataire en detention).

\item[\code{sd}] un vecteur de \code{numeric} a assigner a l'objet \code{Obligation}.
\end{ldescription}
\end{Arguments}
%
\begin{Value}
L'objet \code{x} dont les surcotes decotes ont ete mises a jour.
\end{Value}
%
\begin{Author}\relax
Prim'Act
\end{Author}
\inputencoding{utf8}
\HeaderA{update\_treso}{Permet d'integrer un flux (entrant ou sortant) au compte de tresorerie d'un Portefeuille financier.}{update.Rul.treso}
\aliasA{Treso}{update\_treso}{Treso}
%
\begin{Description}\relax
\code{update\_treso} est une methode permettant d'integrer un flux au compte de tresorerie.
\end{Description}
%
\begin{Usage}
\begin{verbatim}
update_treso(x, flux)
\end{verbatim}
\end{Usage}
%
\begin{Arguments}
\begin{ldescription}
\item[\code{x}] objet de la classe \code{Treso}, correspondant a l'actif Tresorerie d'un assureur anterieur a integration d'un flux.

\item[\code{flux}] est un \code{numeric} correspondant a un flux.
Si il est positif, le flux est entrant.
Si il est negatif, le flux est sortant.
\end{ldescription}
\end{Arguments}
%
\begin{Value}
L'objet \code{Treso} mis a jour du flux precise en input.
\end{Value}
%
\begin{Author}\relax
Prim'Act
\end{Author}
\inputencoding{utf8}
\HeaderA{update\_vm\_action}{Mise a jour de la valeur de marche de chaque composante d'un portefeuille action.}{update.Rul.vm.Rul.action}
\aliasA{Action}{update\_vm\_action}{Action}
%
\begin{Description}\relax
\code{update\_vm\_action} est une methode permettant de mettre a jour la valeur de marche des composantes d'un portefeuille Action.
\end{Description}
%
\begin{Usage}
\begin{verbatim}
update_vm_action(x, vm)
\end{verbatim}
\end{Usage}
%
\begin{Arguments}
\begin{ldescription}
\item[\code{x}] objet de la classe \code{Action} (decrivant le portefeuille action en detention).

\item[\code{vm}] un vecteur de \code{numeric} ayant la meme longueur que le portefeuille action a de lignes et correspondant aux nouvelles valeurs de marche du portefeuille action.
\end{ldescription}
\end{Arguments}
%
\begin{Value}
L'objet \code{x} mis a jour du vieillissement de la duree de detention.
\end{Value}
%
\begin{Author}\relax
Prim'Act
\end{Author}
\inputencoding{utf8}
\HeaderA{update\_vm\_immo}{Mise a jour des valeurs de marche de chaque composante d'un portefeuille immobilier.}{update.Rul.vm.Rul.immo}
\aliasA{Immo}{update\_vm\_immo}{Immo}
%
\begin{Description}\relax
\code{update\_vm\_immo} est une methode permettant de mettre a jour les valeurs de marche des composantes d'un portefeuille immobilier.
\end{Description}
%
\begin{Usage}
\begin{verbatim}
update_vm_immo(x, vm)
\end{verbatim}
\end{Usage}
%
\begin{Arguments}
\begin{ldescription}
\item[\code{x}] objet de la classe \code{Immo} (decrivant le portefeuille immobilier en detention).

\item[\code{vm}] un vecteur de \code{numeric} ayant la meme longueur que le portefeuille immobilier a de lignes et correspondant aux nouvelles valeurs de marche du portefeuille immobilier.
\end{ldescription}
\end{Arguments}
%
\begin{Value}
L'objet \code{x} mis a jour du vieillissement de la duree de detention.
\end{Value}
%
\begin{Author}\relax
Prim'Act
\end{Author}
\inputencoding{utf8}
\HeaderA{update\_vm\_oblig}{Mise a jour des valeurs de marche d'un portefeuille obligataire.}{update.Rul.vm.Rul.oblig}
\aliasA{Oblig}{update\_vm\_oblig}{Oblig}
%
\begin{Description}\relax
\code{update\_vm\_oblig} est une methode permettant de mettre a jour les valeurs de marche des composantes d'un portefeuille obligataire.
\end{Description}
%
\begin{Usage}
\begin{verbatim}
update_vm_oblig(x, vm)
\end{verbatim}
\end{Usage}
%
\begin{Arguments}
\begin{ldescription}
\item[\code{x}] objet de la classe \code{Oblig} (decrivant le portefeuille obligataire en detention).

\item[\code{vm}] un vecteur de \code{numeric} a assigner a l'objet \code{Obligation}.
\end{ldescription}
\end{Arguments}
%
\begin{Value}
L'objet \code{x} dont les valeurs de marche ont ete mises a jour.
\end{Value}
%
\begin{Author}\relax
Prim'Act
\end{Author}
\inputencoding{utf8}
\HeaderA{update\_vnc\_oblig}{Mise a jour des valeurs nettes comptables d'un portefeuille obligataire.}{update.Rul.vnc.Rul.oblig}
\aliasA{Oblig}{update\_vnc\_oblig}{Oblig}
%
\begin{Description}\relax
\code{update\_vnc\_oblig} est une methode permettant de mettre a jour les valeurs nettes comptables des composantes d'un portefeuille obligataire.
\end{Description}
%
\begin{Usage}
\begin{verbatim}
update_vnc_oblig(x, vnc)
\end{verbatim}
\end{Usage}
%
\begin{Arguments}
\begin{ldescription}
\item[\code{x}] objet de la classe \code{Oblig} (decrivant le portefeuille obligataire en detention).

\item[\code{vnc}] un vecteur de \code{numeric} a assigner a l'objet \code{Obligation}.
\end{ldescription}
\end{Arguments}
%
\begin{Value}
L'objet \code{x} dont les valeurs nettes comptables ont ete mis a jour
\end{Value}
%
\begin{Author}\relax
Prim'Act
\end{Author}
\inputencoding{utf8}
\HeaderA{update\_zsp\_oblig}{Mise a jour des zspreads d'un portefeuille obligataire.}{update.Rul.zsp.Rul.oblig}
\aliasA{Oblig}{update\_zsp\_oblig}{Oblig}
%
\begin{Description}\relax
\code{update\_zsp\_oblig} est une methode permettant de mettre a jour les zspreads des composantes d'un portefeuille obligataire.
\end{Description}
%
\begin{Usage}
\begin{verbatim}
update_zsp_oblig(x, zspread)
\end{verbatim}
\end{Usage}
%
\begin{Arguments}
\begin{ldescription}
\item[\code{x}] objet de la classe \code{Oblig} (decrivant le portefeuille obligataire en detention).

\item[\code{zspread}] un vecteur de \code{numeric} a assigner a l'objet \code{Obligation}.
\end{ldescription}
\end{Arguments}
%
\begin{Value}
L'objet \code{x} dont les zspreads ont ete mis a jour
\end{Value}
%
\begin{Author}\relax
Prim'Act
\end{Author}
\inputencoding{utf8}
\HeaderA{vieillissement\_action\_PortFin}{Effectue le vieillissement/la projection du portefeuille action d'un portefeuille financier.}{vieillissement.Rul.action.Rul.PortFin}
\aliasA{PortFin}{vieillissement\_action\_PortFin}{PortFin}
%
\begin{Description}\relax
\code{vieillissement\_action\_PortFin} est une methode permettant de projeter la composante action d'un portefeuille financier.
suite a un vieillissement.
\end{Description}
%
\begin{Usage}
\begin{verbatim}
vieillissement_action_PortFin(x, table_rdt)
\end{verbatim}
\end{Usage}
%
\begin{Arguments}
\begin{ldescription}
\item[\code{x}] objet de la classe \code{PortFin}, correspondant au portefeuille financier de l'assureur avant l'etape de vieillissement de son atribut \code{ptf\_action} de la classe \code{Action}.

\item[\code{table\_rdt}] est une \code{liste}, construite par la fonction \code{\LinkA{calc\_rdt}{calc.Rul.rdt}}.
Cette table contient les tables d'evolution des cours et rendements sur l'annee consideree de chacune des classes d'actif.
Les tables sont constuites a partir des extractions du Generateur de Scenario Economique de Prim'Act.
\end{ldescription}
\end{Arguments}
%
\begin{Value}
Le format de la liste renvoyee est :
\begin{description}

\item[\code{portFin} : ] le portefeuille financier dont l'attribut \code{ptf\_action} a ete vieilli d'une annee.
\item[\code{dividende} : ] le montant de dividende percus en milieu d'annee suite au vieillissement du portefeuille action.

\end{description}

\end{Value}
%
\begin{Author}\relax
Prim'Act
\end{Author}
%
\begin{SeeAlso}\relax
La fonction de calcul des rendements des actifs \code{\LinkA{calc\_rdt}{calc.Rul.rdt}}.
\end{SeeAlso}
\inputencoding{utf8}
\HeaderA{vieillissement\_av\_pb}{Vieillissement du portefeuille sur l'annee avant attribution de participation aux benefices.}{vieillissement.Rul.av.Rul.pb}
\aliasA{PortPassif}{vieillissement\_av\_pb}{PortPassif}
%
\begin{Description}\relax
\code{viellissement\_av\_pb} est une methode permettant de vieillir l'objet \code{\LinkA{PortPassif}{PortPassif}}
sur l'annee avant attribution de participation aux benefices.
\end{Description}
%
\begin{Usage}
\begin{verbatim}
viellissement_av_pb(an, x, coef_inf, list_rd, tx_soc)
\end{verbatim}
\end{Usage}
%
\begin{Arguments}
\begin{ldescription}
\item[\code{an}] une valeur \code{numeric} correspondant a l'annee de projection.

\item[\code{x}] un objet de la classe \code{\LinkA{PortPassif}{PortPassif}} contenant l'ensemble des produits de passifs.

\item[\code{coef\_inf}] une valeur \code{numeric} correspondant au coefficient d'inflation
considere pour le traitement des frais.

\item[\code{list\_rd}] une liste contenant les rendements de reference. Le format de cette liste est :
\begin{description}

\item[le taux de rendement obligataire] 
\item[le taux de rendement de l'indice action de reference] 
\item[le taux de rendement de l'indice immobilier de reference] 
\item[le taux de rendement de l'indice tresorerie de reference] 

\end{description}


\item[\code{tx\_soc}] une valeur \code{numeric} correspondant au taux de charges sociales.
\end{ldescription}
\end{Arguments}
%
\begin{Value}
Une liste comprenant  :
\begin{description}

\item[\code{ptf} : ] Le portefeuille \code{x} mis a jour.
\item[\code{result\_av\_pb} : ] Une liste dont le premier element designe les noms des produits,
puis deux matrices de resultats aggreges : une pour les flux et une pour le stock. Le format de cette sortie
decoule de celui de la methode \code{\LinkA{proj\_annee\_av\_pb}{proj.Rul.annee.Rul.av.Rul.pb}}.
\item[\code{result\_autres\_passifs} : ] un vecteur contenant les resultats des passifs non modelises.
\item[\code{var\_psap} : ] la variation de PSAP sur l'annee.
\item[\code{var\_pgg} : ] la variation de PGG sur l'annee.
\item[\code{flux\_milieu} : ] les flux de milieu d'annee entrant en tresorerie en milieu de periode.
\item[\code{flux\_fin} : ] les flux de fin d'annee entrant en tresorerie en fin de periode.

\end{description}

\end{Value}
%
\begin{Author}\relax
Prim'Act
\end{Author}
%
\begin{SeeAlso}\relax
La projection des passifs sur un an avant PB : \code{\LinkA{proj\_annee\_av\_pb}{proj.Rul.annee.Rul.av.Rul.pb}}.
La projection des autres passifs : \code{\LinkA{proj\_annee\_autres\_passifs}{proj.Rul.annee.Rul.autres.Rul.passifs}}.
La mise a jour des autres reserves : \code{\LinkA{update\_reserves}{update.Rul.reserves}}.
\end{SeeAlso}
\inputencoding{utf8}
\HeaderA{vieillissement\_immo\_PortFin}{Effectue le vieillissement/la projection du portefeuille immo d'un portefeuille financier.}{vieillissement.Rul.immo.Rul.PortFin}
\aliasA{PortFin}{vieillissement\_immo\_PortFin}{PortFin}
%
\begin{Description}\relax
\code{vieillissement\_immo\_PortFin} est une methode permettant de projeter la composante immobilier d'un portefeuille financier.
\end{Description}
%
\begin{Usage}
\begin{verbatim}
vieillissement_immo_PortFin(x, table_rdt)
\end{verbatim}
\end{Usage}
%
\begin{Arguments}
\begin{ldescription}
\item[\code{x}] objet de la classe \code{PortFin}, correspondant au portefeuille financier de l'assureur avant l'etape de vieillissement de son atribut \code{ptf\_immo} de la classe \code{Immo}.

\item[\code{table\_rdt}] est une \code{liste}, construite par la fonction \code{\LinkA{calc\_rdt}{calc.Rul.rdt}}.
Cette table contient les tables d'evolution des cours et rendements sur l'annee consideree de chacune des classes d'actif.
Les tables sont constuites a partir des extractions du Generateur de Scenario Economique de Prim'Act.
\end{ldescription}
\end{Arguments}
%
\begin{Value}
Le format de la liste renvoyee est :
\begin{description}

\item[\code{portFin} : ] le portefeuille financier dont l'attribut \code{ptf\_immo} a ete vieilli d'une annee.
\item[\code{loyer} : ] le montant de loyer percus en milieu d'annee suite au vieillissement du portefeuille immobilier.

\end{description}

\end{Value}
%
\begin{Author}\relax
Prim'Act
\end{Author}
%
\begin{SeeAlso}\relax
La fonction de calcul des rendements des actifs \code{\LinkA{calc\_rdt}{calc.Rul.rdt}}.
\end{SeeAlso}
\inputencoding{utf8}
\HeaderA{vieillissement\_oblig\_PortFin}{Effectue le vieillissement/la projection du portefeuille obligataire d'un portefeuille financier.}{vieillissement.Rul.oblig.Rul.PortFin}
\aliasA{PortFin}{vieillissement\_oblig\_PortFin}{PortFin}
%
\begin{Description}\relax
\code{vieillissement\_oblig\_PortFin} est une methode permettant de projeter la composante obligataire d'un portefeuille financier.
\end{Description}
%
\begin{Usage}
\begin{verbatim}
vieillissement_oblig_PortFin(x, new_mp_ESG)
\end{verbatim}
\end{Usage}
%
\begin{Arguments}
\begin{ldescription}
\item[\code{x}] objet de la classe \code{PortFin}, correspondant au portefeuille financier de l'assureur avant l'etape de vieillissement de son atribut \code{ptf\_oblig} de la classe \code{Oblig}.

\item[\code{new\_mp\_ESG}] est un objet de type \code{ModelPointESG}, correspondant aux conditions economiques de l'annee du vieillissement.
\end{ldescription}
\end{Arguments}
%
\begin{Value}
Le format de la liste renvoyee est :
\begin{description}

\item[\code{portFin} : ] le portefeuille financier dont l'attribut \code{ptf\_oblig} a ete vieilli d'une annee.
\item[\code{loyer} : ] le montant de loyer percus en milieu d'annee suite au vieillissement du portefeuille obligataire.

\end{description}

\end{Value}
%
\begin{Author}\relax
Prim'Act
\end{Author}
%
\begin{SeeAlso}\relax
La fonction de calcul des rendements des actifs \code{\LinkA{calc\_rdt}{calc.Rul.rdt}}.
\end{SeeAlso}
\inputencoding{utf8}
\HeaderA{vieillissement\_treso\_PortFin}{Effectue le vieillissement/la projection du portefeuille tresorerie d'un portefeuille financier.}{vieillissement.Rul.treso.Rul.PortFin}
\aliasA{PortFin}{vieillissement\_treso\_PortFin}{PortFin}
%
\begin{Description}\relax
\code{vieillissement\_treso\_PortFin} est une methode permettant de projeter la composante obligataire d'un portefeuille financier.
\end{Description}
%
\begin{Usage}
\begin{verbatim}
vieillissement_treso_PortFin(x, flux_milieu, flux_fin, table_rdt)
\end{verbatim}
\end{Usage}
%
\begin{Arguments}
\begin{ldescription}
\item[\code{x}] objet de la classe \code{PortFin}, correspondant au portefeuille financier de l'assureur avant l'etape de vieillissement de son atribut \code{ptf\_treso} de la classe \code{Treso}.

\item[\code{flux\_milieu}] est un \code{numeric} correspondant aux revenus percus en milieu d'annee (coupons obligataires, loyers, dividendes).

\item[\code{flux\_fin}] est un \code{numeric} correspondant aux revenus percus en fin d'annee (tombees d'echeances et revenus de tresorerie).

\item[\code{table\_rdt}] est une \code{liste}, construite par la fonction \code{\LinkA{calc\_rdt}{calc.Rul.rdt}}.
Cette table contient les tables d'evolution des cours et rendements sur l'annee consideree de chacune des classes d'actif.
Les tables sont constuites a partir des extractions du Generateur de Scenario Economique de Prim'Act.
\end{ldescription}
\end{Arguments}
%
\begin{Value}
L'objet renvoye de la classe \code{PortFin} correspond au portefeuille financier initial dont l'attribut \code{ptf\_treso} a ete vieilli d'une annee.
\end{Value}
%
\begin{Author}\relax
Prim'Act
\end{Author}
%
\begin{SeeAlso}\relax
La fonction de calcul des rendements des actifs \code{\LinkA{calc\_rdt}{calc.Rul.rdt}}.
\end{SeeAlso}
\inputencoding{utf8}
\HeaderA{vieillissment\_ap\_pb}{Vieillissement du portefeuille sur l'annee apres attribution de participation aux benefices.}{vieillissment.Rul.ap.Rul.pb}
\aliasA{PortPassif}{vieillissment\_ap\_pb}{PortPassif}
%
\begin{Description}\relax
\code{vieillissment\_ap\_pb} est une methode permettant de calculer les PM et les flux sur une annee apres PB.
Cette methode vieilli le portefeuille de passifs apres attribution de PB.
\end{Description}
%
\begin{Usage}
\begin{verbatim}
vieillissment_ap_pb(x, rev_nette_alloue, tx_soc)
\end{verbatim}
\end{Usage}
%
\begin{Arguments}
\begin{ldescription}
\item[\code{x}] un objet de la classe \code{\LinkA{PortPassif}{PortPassif}} contenant l'ensemble des produits de passifs.

\item[\code{rev\_nette\_alloue}] un vecteur \code{numeric} contenant par produit
le supplement de revalorisation par rapport au taux minimum.

\item[\code{tx\_soc}] une valeur \code{numeric} correspondant au taux de charges sociales.
\end{ldescription}
\end{Arguments}
%
\begin{Value}
\code{x} l'objet \code{x} mis a jour.

\code{nom\_produit} un vecteur de \code{character} contenant les noms des produits.

\code{flux\_agg} une matrice contenant les flux aggreges par produits.

\code{stock\_agg} une matrice contenant les stocks aggreges par produits.
\end{Value}
%
\begin{Author}\relax
Prim'Act
\end{Author}
%
\begin{SeeAlso}\relax
L'attribution de la revalorisation par model point : \code{\LinkA{calc\_revalo\_pm}{calc.Rul.revalo.Rul.pm}}
Le viellissement des model points : \code{\LinkA{vieilli\_mp}{vieilli.Rul.mp}}.
\end{SeeAlso}
\inputencoding{utf8}
\HeaderA{vieilli\_mp}{Veillissement d'un an des contrats epargne en euros.}{vieilli.Rul.mp}
\aliasA{EpEuroInd}{vieilli\_mp}{EpEuroInd}
%
\begin{Description}\relax
\code{vieilli\_mp} est une methode permettant de vieillir
les model points epargne en euros d'une peridoe.
\end{Description}
%
\begin{Usage}
\begin{verbatim}
vieilli_mp(x, pm_fin_ap_pb, tx_revalo)
\end{verbatim}
\end{Usage}
%
\begin{Arguments}
\begin{ldescription}
\item[\code{x}] un objet de la classe \code{\LinkA{EpEuroInd}{EpEuroInd}} contenant les model points epargne euros.

\item[\code{pm\_fin\_ap\_pb}] un vecteur de type \code{numeric} contenant par model point
les montants de PM revalorises apres participation aux benefices.

\item[\code{tx\_revalo}] un vecteur de type \code{numeric} contenant par model point
les taux de revalorisation nets appliques.
\end{ldescription}
\end{Arguments}
%
\begin{Value}
l'objet \code{x} vieilli d'une periode.
\end{Value}
%
\begin{Author}\relax
Prim'Act
\end{Author}
%
\begin{SeeAlso}\relax
Calcul de la revalorisation des PM \code{\LinkA{calc\_revalo\_pm}{calc.Rul.revalo.Rul.pm}}.
\end{SeeAlso}
\inputencoding{utf8}
\HeaderA{write\_be\_results}{Enregistre les resultats d'une evaluation best estimate}{write.Rul.be.Rul.results}
\aliasA{Be}{write\_be\_results}{Be}
%
\begin{Description}\relax
\code{write\_be\_results} est une methode permettant d'enregistrer en \code{.cvs} les resultats
d'une evaluation best estimate.
\end{Description}
%
\begin{Usage}
\begin{verbatim}
write_be_results(nom_run, path, x)
\end{verbatim}
\end{Usage}
%
\begin{Arguments}
\begin{ldescription}
\item[\code{nom\_run}] est un objet de type \code{character} utilise pour nommer le fichier de resultats.

\item[\code{path}] est un objet de type \code{character} utilise pour indiquer le chemin d'enregistrement des resultats.

\item[\code{x}] est un objet de type \code{Be}.
\end{ldescription}
\end{Arguments}
%
\begin{Author}\relax
Prim'Act
\end{Author}
\inputencoding{utf8}
\HeaderA{yield\_to\_maturity}{Calcul les yield to maturity de chaque composante d'un portefeuille obligataire.}{yield.Rul.to.Rul.maturity}
\aliasA{Oblig}{yield\_to\_maturity}{Oblig}
%
\begin{Description}\relax
\code{yield\_to\_maturity} est une methode permettant de calculer les yield to maturity de chaque composante d'un portefeuille obligataire.
\end{Description}
%
\begin{Usage}
\begin{verbatim}
yield_to_maturity(x)
\end{verbatim}
\end{Usage}
%
\begin{Arguments}
\begin{ldescription}
\item[\code{x}] objet de la classe Oblig (decrivant le portefeuile obligataire).
\end{ldescription}
\end{Arguments}
%
\begin{Value}
Un vecteur dont chaque element correspond au yield to maturity de l'obligation correspondante du portefeuille obligataire.
Ce vecteur a autant d'elements que le portefeuille obligataire a de lignes.
\end{Value}
%
\begin{Author}\relax
Prim'Act
\end{Author}
\printindex{}
\end{document}
